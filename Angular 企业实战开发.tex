% Options for packages loaded elsewhere
\PassOptionsToPackage{unicode}{hyperref}
\PassOptionsToPackage{hyphens}{url}
%
\documentclass[
]{article}
\usepackage{lmodern}
\usepackage{amssymb,amsmath}
\usepackage{ifxetex,ifluatex}
\ifnum 0\ifxetex 1\fi\ifluatex 1\fi=0 % if pdftex
  \usepackage[T1]{fontenc}
  \usepackage[utf8]{inputenc}
  \usepackage{textcomp} % provide euro and other symbols
\else % if luatex or xetex
  \usepackage{unicode-math}
  \defaultfontfeatures{Scale=MatchLowercase}
  \defaultfontfeatures[\rmfamily]{Ligatures=TeX,Scale=1}
\fi
% Use upquote if available, for straight quotes in verbatim environments
\IfFileExists{upquote.sty}{\usepackage{upquote}}{}
\IfFileExists{microtype.sty}{% use microtype if available
  \usepackage[]{microtype}
  \UseMicrotypeSet[protrusion]{basicmath} % disable protrusion for tt fonts
}{}
\makeatletter
\@ifundefined{KOMAClassName}{% if non-KOMA class
  \IfFileExists{parskip.sty}{%
    \usepackage{parskip}
  }{% else
    \setlength{\parindent}{0pt}
    \setlength{\parskip}{6pt plus 2pt minus 1pt}}
}{% if KOMA class
  \KOMAoptions{parskip=half}}
\makeatother
\usepackage{xcolor}
\IfFileExists{xurl.sty}{\usepackage{xurl}}{} % add URL line breaks if available
\IfFileExists{bookmark.sty}{\usepackage{bookmark}}{\usepackage{hyperref}}
\hypersetup{
  hidelinks,
  pdfcreator={LaTeX via pandoc}}
\urlstyle{same} % disable monospaced font for URLs
\usepackage{color}
\usepackage{fancyvrb}
\newcommand{\VerbBar}{|}
\newcommand{\VERB}{\Verb[commandchars=\\\{\}]}
\DefineVerbatimEnvironment{Highlighting}{Verbatim}{commandchars=\\\{\}}
% Add ',fontsize=\small' for more characters per line
\newenvironment{Shaded}{}{}
\newcommand{\AlertTok}[1]{\textcolor[rgb]{1.00,0.00,0.00}{\textbf{#1}}}
\newcommand{\AnnotationTok}[1]{\textcolor[rgb]{0.38,0.63,0.69}{\textbf{\textit{#1}}}}
\newcommand{\AttributeTok}[1]{\textcolor[rgb]{0.49,0.56,0.16}{#1}}
\newcommand{\BaseNTok}[1]{\textcolor[rgb]{0.25,0.63,0.44}{#1}}
\newcommand{\BuiltInTok}[1]{#1}
\newcommand{\CharTok}[1]{\textcolor[rgb]{0.25,0.44,0.63}{#1}}
\newcommand{\CommentTok}[1]{\textcolor[rgb]{0.38,0.63,0.69}{\textit{#1}}}
\newcommand{\CommentVarTok}[1]{\textcolor[rgb]{0.38,0.63,0.69}{\textbf{\textit{#1}}}}
\newcommand{\ConstantTok}[1]{\textcolor[rgb]{0.53,0.00,0.00}{#1}}
\newcommand{\ControlFlowTok}[1]{\textcolor[rgb]{0.00,0.44,0.13}{\textbf{#1}}}
\newcommand{\DataTypeTok}[1]{\textcolor[rgb]{0.56,0.13,0.00}{#1}}
\newcommand{\DecValTok}[1]{\textcolor[rgb]{0.25,0.63,0.44}{#1}}
\newcommand{\DocumentationTok}[1]{\textcolor[rgb]{0.73,0.13,0.13}{\textit{#1}}}
\newcommand{\ErrorTok}[1]{\textcolor[rgb]{1.00,0.00,0.00}{\textbf{#1}}}
\newcommand{\ExtensionTok}[1]{#1}
\newcommand{\FloatTok}[1]{\textcolor[rgb]{0.25,0.63,0.44}{#1}}
\newcommand{\FunctionTok}[1]{\textcolor[rgb]{0.02,0.16,0.49}{#1}}
\newcommand{\ImportTok}[1]{#1}
\newcommand{\InformationTok}[1]{\textcolor[rgb]{0.38,0.63,0.69}{\textbf{\textit{#1}}}}
\newcommand{\KeywordTok}[1]{\textcolor[rgb]{0.00,0.44,0.13}{\textbf{#1}}}
\newcommand{\NormalTok}[1]{#1}
\newcommand{\OperatorTok}[1]{\textcolor[rgb]{0.40,0.40,0.40}{#1}}
\newcommand{\OtherTok}[1]{\textcolor[rgb]{0.00,0.44,0.13}{#1}}
\newcommand{\PreprocessorTok}[1]{\textcolor[rgb]{0.74,0.48,0.00}{#1}}
\newcommand{\RegionMarkerTok}[1]{#1}
\newcommand{\SpecialCharTok}[1]{\textcolor[rgb]{0.25,0.44,0.63}{#1}}
\newcommand{\SpecialStringTok}[1]{\textcolor[rgb]{0.73,0.40,0.53}{#1}}
\newcommand{\StringTok}[1]{\textcolor[rgb]{0.25,0.44,0.63}{#1}}
\newcommand{\VariableTok}[1]{\textcolor[rgb]{0.10,0.09,0.49}{#1}}
\newcommand{\VerbatimStringTok}[1]{\textcolor[rgb]{0.25,0.44,0.63}{#1}}
\newcommand{\WarningTok}[1]{\textcolor[rgb]{0.38,0.63,0.69}{\textbf{\textit{#1}}}}
\usepackage{graphicx}
\makeatletter
\def\maxwidth{\ifdim\Gin@nat@width>\linewidth\linewidth\else\Gin@nat@width\fi}
\def\maxheight{\ifdim\Gin@nat@height>\textheight\textheight\else\Gin@nat@height\fi}
\makeatother
% Scale images if necessary, so that they will not overflow the page
% margins by default, and it is still possible to overwrite the defaults
% using explicit options in \includegraphics[width, height, ...]{}
\setkeys{Gin}{width=\maxwidth,height=\maxheight,keepaspectratio}
% Set default figure placement to htbp
\makeatletter
\def\fps@figure{htbp}
\makeatother
\setlength{\emergencystretch}{3em} % prevent overfull lines
\providecommand{\tightlist}{%
  \setlength{\itemsep}{0pt}\setlength{\parskip}{0pt}}
\setcounter{secnumdepth}{-\maxdimen} % remove section numbering

\author{}
\date{}

\begin{document}

\hypertarget{angular-ux4f01ux4e1aux5b9eux6218ux5f00ux53d1}{%
\subsection{Angular
企业实战开发}\label{angular-ux4f01ux4e1aux5b9eux6218ux5f00ux53d1}}

\hypertarget{1--ux6982ux8ff0}{%
\subsubsection{1. 概述}\label{1--ux6982ux8ff0}}

Angular 是一个使用 HTML、CSS、TypeScript
构建客户端应用的框架,用来构建单页应用程序。

Angular 是一个重量级的框架,内部集成了大量开箱即用的功能模块。

Angular
为大型应用开发而设计,提供了干净且松耦合的代码组织方式,使应用程序整洁更易于维护。

\href{https://angular.io/}{Angular} \href{https://angular.cn/}{Angular
中文} \href{https://cli.angular.io/}{Angular CLI}

\hypertarget{2-ux67b6ux6784ux9884ux89c8}{%
\subsubsection{2. 架构预览}\label{2-ux67b6ux6784ux9884ux89c8}}

\begin{figure}
\centering
\includegraphics{C:/Users/ZSH/Desktop/ng/ppt/images/1.png}
\caption{}
\end{figure}

\hypertarget{21-ux6a21ux5757}{%
\paragraph{2.1 模块}\label{21-ux6a21ux5757}}

Angular 应用是由一个个模块组成的,此模块指的不是ESModule,而是 NgModule
即 Angular 模块。

NgModule
是一组相关功能的集合,专注于某个应用领域,可以将组件和一组相关代码关联起来,是应用组织代码结构的一种方式。

在 Angular 应用中至少要有一个根模块,用于启动应用程序。

NgModule 可以从其它 NgModule 中导入功能,前提是目标 NgModule
导出了该功能。

NgModule 是由 NgModule 装饰器函数装饰的类。

\begin{Shaded}
\begin{Highlighting}[]
\ImportTok{import}\NormalTok{ \{ BrowserModule \} }\ImportTok{from} \StringTok{\textquotesingle{}@angular/platform{-}browser\textquotesingle{}}\OperatorTok{;}
\ImportTok{import}\NormalTok{ \{ NgModule \} }\ImportTok{from} \StringTok{\textquotesingle{}@angular/core\textquotesingle{}}\OperatorTok{;}

\NormalTok{@}\FunctionTok{NgModule}\NormalTok{(\{}
  \DataTypeTok{imports}\OperatorTok{:}\NormalTok{ [}
\NormalTok{    BrowserModule}
\NormalTok{  ]}
\NormalTok{\})}
\ImportTok{export} \KeywordTok{class}\NormalTok{ AppModule \{ \}}
\end{Highlighting}
\end{Shaded}

\hypertarget{22-ux7ec4ux4ef6}{%
\paragraph{2.2 组件}\label{22-ux7ec4ux4ef6}}

组件用来描述用户界面,它由三部分组成,组件类、组件模板、组件样式,它们可以被集成在组件类文件中,也可以是三个不同的文件。

组件类用来编写和组件直接相关的界面逻辑,在组件类中要关联该组件的组件模板和组件样式。

组件模板用来编写组件的 HTML 结构,通过数据绑定标记将应用中数据和 DOM
进行关联。

组件样式用来编写组件的组件的外观,组件样式可以采用
CSS、LESS、SCSS、Stylus

在 Angular 应用中至少要有一个根组件,用于应用程序的启动。

组件类是由 Component 装饰器函数装饰的类。

\begin{Shaded}
\begin{Highlighting}[]
\ImportTok{import}\NormalTok{ \{ Component \} }\ImportTok{from} \StringTok{"@angular/core"}

\NormalTok{@}\FunctionTok{Component}\NormalTok{(\{}
  \DataTypeTok{selector}\OperatorTok{:} \StringTok{"app{-}root"}\OperatorTok{,}
  \DataTypeTok{templateUrl}\OperatorTok{:} \StringTok{"./app.component.html"}\OperatorTok{,}
  \DataTypeTok{styleUrls}\OperatorTok{:}\NormalTok{ [}\StringTok{"./app.component.css"}\NormalTok{]}
\NormalTok{\})}
\ImportTok{export} \KeywordTok{class}\NormalTok{ AppComponent \{}
\NormalTok{  title }\OperatorTok{=} \StringTok{"angular{-}test"}
\NormalTok{\}}
\end{Highlighting}
\end{Shaded}

NgModule 为组件提供了编译的上下文环境。

\begin{Shaded}
\begin{Highlighting}[]
\ImportTok{import}\NormalTok{ \{ NgModule \} }\ImportTok{from} \StringTok{\textquotesingle{}@angular/core\textquotesingle{}}\OperatorTok{;}
\ImportTok{import}\NormalTok{ \{ AppComponent \} }\ImportTok{from} \StringTok{\textquotesingle{}./app.component\textquotesingle{}}\OperatorTok{;}

\NormalTok{@}\FunctionTok{NgModule}\NormalTok{(\{}
  \DataTypeTok{declarations}\OperatorTok{:}\NormalTok{ [}
\NormalTok{    AppComponent}
\NormalTok{  ]}\OperatorTok{,}
  \DataTypeTok{bootstrap}\OperatorTok{:}\NormalTok{ [AppComponent]}
\NormalTok{\})}
\ImportTok{export} \KeywordTok{class}\NormalTok{ AppModule \{ \}}
\end{Highlighting}
\end{Shaded}

\hypertarget{23-ux670dux52a1}{%
\paragraph{2.3 服务}\label{23-ux670dux52a1}}

服务用于放置和特定组件无关并希望跨组件共享的数据或逻辑。

服务出现的目的在于解耦组件类中的代码,是组件类中的代码干净整洁。

服务是由 Injectable 装饰器装饰的类。

\begin{Shaded}
\begin{Highlighting}[]
\ImportTok{import}\NormalTok{ \{ Injectable \} }\ImportTok{from} \StringTok{\textquotesingle{}@angular/core\textquotesingle{}}\OperatorTok{;}

\NormalTok{@}\FunctionTok{Injectable}\NormalTok{(\{\})}
\ImportTok{export} \KeywordTok{class}\NormalTok{ AppService \{ \}}
\end{Highlighting}
\end{Shaded}

在使用服务时不需要在组件类中通过 new
的方式创建服务实例对象获取服务中提供的方法,以下写法错误,切记切记!!!

\begin{Shaded}
\begin{Highlighting}[]
\ImportTok{import}\NormalTok{ \{ AppService \} }\ImportTok{from} \StringTok{"./AppService"}

\ImportTok{export} \KeywordTok{class}\NormalTok{ AppComponent \{}
  \KeywordTok{let}\NormalTok{ appService }\OperatorTok{=} \KeywordTok{new} \FunctionTok{AppService}\NormalTok{()}
\NormalTok{\}}
\end{Highlighting}
\end{Shaded}

服务的实例对象由 Angular
框架中内置的依赖注入系统创建和维护。服务是依赖需要被注入到组件中。

在组件中需要通过 constructor 构造函数的参数来获取服务的实例对象。

涉及参数就需要考虑参数的顺序问题,因为在 Angular
应用中会有很多服务,一个组件又不可能会使用到所有服务,如果组件要使用到最后一个服务实例对象,难道要将前面的所有参数都写上吗
? 这显然不合理。

在组件中获取服务实例对象要结合 TypeScript 类型,写法如下。

\begin{Shaded}
\begin{Highlighting}[]
\ImportTok{import}\NormalTok{ \{ AppService \} }\ImportTok{from} \StringTok{"./AppService"}

\ImportTok{export} \KeywordTok{class}\NormalTok{ AppComponent \{}
  \FunctionTok{constructor}\NormalTok{ (}
  	\KeywordTok{private} \DataTypeTok{appService}\OperatorTok{:}\NormalTok{ AppService}
\NormalTok{  ) \{\}}
\NormalTok{\}}
\end{Highlighting}
\end{Shaded}

Angular
会根据你指定的服务的类型来传递你想要使用的服务实例对象,这样就解决了参数的顺序问题。

在 Angular
中服务被设计为单例模式,这也正是为什么服务可以被用来在组件之间共享数据和逻辑的原因。

\hypertarget{3-ux5febux901fux5f00ux59cb}{%
\subsubsection{3. 快速开始}\label{3-ux5febux901fux5f00ux59cb}}

\hypertarget{31-ux521bux5efaux5e94ux7528}{%
\paragraph{3.1 创建应用}\label{31-ux521bux5efaux5e94ux7528}}

\begin{enumerate}
\def\labelenumi{\arabic{enumi}.}
\item
  安装 angular-cli:\texttt{npm\ install\ @angular/cli\ -g}
\item
  创建应用:\texttt{ng\ new\ angular-test\ -\/-minimal\ -\/-inlineTemplate\ false}

  \begin{enumerate}
  \def\labelenumii{\arabic{enumii}.}
  \item
    -\/-skipGit=true
  \item
    -\/-minimal=true
  \item
    -\/-skip-install
  \item
    -\/-style=css
  \item
    -\/-routing=false
  \item
    -\/-inlineTemplate
  \item
    -\/-inlineStyle
  \item
    -\/-prefix
  \end{enumerate}

  \begin{figure}
  \centering
  \includegraphics{C:/Users/ZSH/Desktop/ng/ppt/images/58.png}
  \caption{}
  \end{figure}

  \begin{figure}
  \centering
  \includegraphics{C:/Users/ZSH/Desktop/ng/ppt/images/59.png}
  \caption{}
  \end{figure}

  \begin{figure}
  \centering
  \includegraphics{C:/Users/ZSH/Desktop/ng/ppt/images/60.png}
  \caption{}
  \end{figure}
\item
  运行应用:\texttt{ng\ serve}

  \begin{enumerate}
  \def\labelenumii{\arabic{enumii}.}
  \item
    -\/-open=true 应用构建完成后在浏览器中运行
  \item
    -\/-hmr=true 开启热更新
  \item
    hmrWarning=false 禁用热更新警告
  \item
    -\/-port 更改应用运行端口
  \end{enumerate}
\item
  访问应用:\texttt{localhost:4200}

  \begin{figure}
  \centering
  \includegraphics{C:/Users/ZSH/Desktop/ng/ppt/images/3.png}
  \caption{}
  \end{figure}
\end{enumerate}

\hypertarget{32-ux9ed8ux8ba4ux4ee3ux7801ux89e3ux6790}{%
\paragraph{3.2
默认代码解析}\label{32-ux9ed8ux8ba4ux4ee3ux7801ux89e3ux6790}}

\hypertarget{321-maints}{%
\subparagraph{3.2.1 main.ts}\label{321-maints}}

\begin{Shaded}
\begin{Highlighting}[]
\CommentTok{// enableProdMode 方法调用后将会开启生产模式}
\ImportTok{import}\NormalTok{ \{ enableProdMode \} }\ImportTok{from} \StringTok{"@angular/core"}
\CommentTok{// Angular 应用程序的启动在不同的平台上是不一样的}
\CommentTok{// 在浏览器中启动时需要用到 platformBrowserDynamic 方法, 该方法返回平台实例对象}
\ImportTok{import}\NormalTok{ \{ platformBrowserDynamic \} }\ImportTok{from} \StringTok{"@angular/platform{-}browser{-}dynamic"}
\CommentTok{// 引入根模块 用于启动应用程序}
\ImportTok{import}\NormalTok{ \{ AppModule \} }\ImportTok{from} \StringTok{"./app/app.module"}
\CommentTok{// 引入环境变量对象 \{ production: false \}}
\ImportTok{import}\NormalTok{ \{ environment \} }\ImportTok{from} \StringTok{"./environments/environment"}

\CommentTok{// 如果当前为生产环境}
\ControlFlowTok{if}\NormalTok{ (environment}\OperatorTok{.}\AttributeTok{production}\NormalTok{) \{}
  \CommentTok{// 开启生产模式}
  \FunctionTok{enableProdMode}\NormalTok{()}
\NormalTok{\}}
\CommentTok{// 启动应用程序}
\FunctionTok{platformBrowserDynamic}\NormalTok{()}
  \OperatorTok{.}\FunctionTok{bootstrapModule}\NormalTok{(AppModule)}
  \OperatorTok{.}\FunctionTok{catch}\NormalTok{(err }\KeywordTok{=\textgreater{}} \BuiltInTok{console}\OperatorTok{.}\FunctionTok{error}\NormalTok{(err))}
\end{Highlighting}
\end{Shaded}

\begin{figure}
\centering
\includegraphics{C:/Users/ZSH/Desktop/ng/ppt/images/61.png}
\caption{}
\end{figure}

\hypertarget{322-environmentts}{%
\subparagraph{3.2.2 environment.ts}\label{322-environmentts}}

\begin{Shaded}
\begin{Highlighting}[]
\CommentTok{// 在执行 \textasciigrave{}ng build {-}{-}prod\textasciigrave{} 时, environment.prod.ts 文件会替换 environment.ts 文件}
\CommentTok{// 该项配置可以在 angular.json 文件中找到, projects {-}\textgreater{} angular{-}test {-}\textgreater{} architect {-}\textgreater{} configurations {-}\textgreater{} production {-}\textgreater{} fileReplacements}

\ImportTok{export} \KeywordTok{const}\NormalTok{ environment }\OperatorTok{=}\NormalTok{ \{}
  \DataTypeTok{production}\OperatorTok{:} \KeywordTok{false}
\NormalTok{\}}
\end{Highlighting}
\end{Shaded}

\hypertarget{323--environmentprodts}{%
\subparagraph{3.2.3 environment.prod.ts}\label{323--environmentprodts}}

\begin{Shaded}
\begin{Highlighting}[]
\ImportTok{export} \KeywordTok{const}\NormalTok{ environment }\OperatorTok{=}\NormalTok{ \{}
  \DataTypeTok{production}\OperatorTok{:} \KeywordTok{true}
\NormalTok{\}}
\end{Highlighting}
\end{Shaded}

\hypertarget{324-appmodulets}{%
\subparagraph{3.2.4 app.module.ts}\label{324-appmodulets}}

\begin{Shaded}
\begin{Highlighting}[]
\CommentTok{// BrowserModule 提供了启动和运行浏览器应用所必需的服务}
\CommentTok{// CommonModule 提供各种服务和指令, 例如 ngIf 和 ngFor, 与平台无关}
\CommentTok{// BrowserModule 导入了 CommonModule, 又重新导出了 CommonModule, 使其所有指令都可用于导入 BrowserModule 的任何模块 }
\ImportTok{import}\NormalTok{ \{ BrowserModule \} }\ImportTok{from} \StringTok{"@angular/platform{-}browser"}
\CommentTok{// NgModule: Angular 模块装饰器}
\ImportTok{import}\NormalTok{ \{ NgModule \} }\ImportTok{from} \StringTok{"@angular/core"}
\CommentTok{// 根组件}
\ImportTok{import}\NormalTok{ \{ AppComponent \} }\ImportTok{from} \StringTok{"./app.component"}
\CommentTok{// 调用 NgModule 装饰器, 告诉 Angular 当前类表示的是 Angular 模块}
\NormalTok{@}\FunctionTok{NgModule}\NormalTok{(\{}
  \CommentTok{// 声明当前模块拥有哪些组件}
  \DataTypeTok{declarations}\OperatorTok{:}\NormalTok{ [AppComponent]}\OperatorTok{,}
  \CommentTok{// 声明当前模块依赖了哪些其他模块}
  \DataTypeTok{imports}\OperatorTok{:}\NormalTok{ [BrowserModule]}\OperatorTok{,}
  \CommentTok{// 声明服务的作用域, 数组中接收服务类, 表示该服务只能在当前模块的组件中使用}
  \DataTypeTok{providers}\OperatorTok{:}\NormalTok{ []}\OperatorTok{,}
  \CommentTok{// 可引导组件, Angular 会在引导过程中把它加载到 DOM 中}
  \DataTypeTok{bootstrap}\OperatorTok{:}\NormalTok{ [AppComponent]}
\NormalTok{\})}
\ImportTok{export} \KeywordTok{class}\NormalTok{ AppModule \{\}}
\end{Highlighting}
\end{Shaded}

\hypertarget{325-appcomponentts}{%
\subparagraph{3.2.5 app.component.ts}\label{325-appcomponentts}}

\begin{Shaded}
\begin{Highlighting}[]
\ImportTok{import}\NormalTok{ \{ Component \} }\ImportTok{from} \StringTok{"@angular/core"}

\NormalTok{@}\FunctionTok{Component}\NormalTok{(\{}
  \CommentTok{// 指定组件的使用方式, 当前为标记形式}
  \CommentTok{// app{-}home   =\textgreater{}  \textless{}app{-}home\textgreater{}\textless{}/app{-}home\textgreater{}}
	\CommentTok{// [app{-}home] =\textgreater{}  \textless{}div app{-}home\textgreater{}\textless{}/div\textgreater{}}
  \CommentTok{// .app{-}home  =\textgreater{}  \textless{}div class="app{-}home"\textgreater{}\textless{}/div\textgreater{}}
  \DataTypeTok{selector}\OperatorTok{:} \StringTok{"app{-}root"}\OperatorTok{,}
  \CommentTok{// 关联组件模板文件}
  \CommentTok{// templateUrl:\textquotesingle{}组件模板文件路径\textquotesingle{}}
	\CommentTok{// template:\textasciigrave{}组件模板字符串\textasciigrave{}}
  \DataTypeTok{templateUrl}\OperatorTok{:} \StringTok{"./app.component.html"}\OperatorTok{,}
  \CommentTok{// 关联组件样式文件}
  \CommentTok{// styleUrls : [\textquotesingle{}组件样式文件路径\textquotesingle{}]}
	\CommentTok{// styles : [\textasciigrave{}组件样式\textasciigrave{}]}
  \DataTypeTok{styleUrls}\OperatorTok{:}\NormalTok{ [}\StringTok{"./app.component.css"}\NormalTok{]}
\NormalTok{\})}
\ImportTok{export} \KeywordTok{class}\NormalTok{ AppComponent \{\}}
\end{Highlighting}
\end{Shaded}

\hypertarget{326-indexhtml}{%
\subparagraph{3.2.6 index.html}\label{326-indexhtml}}

\begin{Shaded}
\begin{Highlighting}[]
\DataTypeTok{\textless{}!doctype }\NormalTok{html}\DataTypeTok{\textgreater{}}
\KeywordTok{\textless{}html}\OtherTok{ lang=}\StringTok{"en"}\KeywordTok{\textgreater{}}
\KeywordTok{\textless{}head\textgreater{}}
  \KeywordTok{\textless{}meta}\OtherTok{ charset=}\StringTok{"utf{-}8"}\KeywordTok{\textgreater{}}
  \KeywordTok{\textless{}title\textgreater{}}\NormalTok{AngularTest}\KeywordTok{\textless{}/title\textgreater{}}
  \KeywordTok{\textless{}base}\OtherTok{ href=}\StringTok{"/"}\KeywordTok{\textgreater{}}
  \KeywordTok{\textless{}meta}\OtherTok{ name=}\StringTok{"viewport"}\OtherTok{ content=}\StringTok{"width=device{-}width, initial{-}scale=1"}\KeywordTok{\textgreater{}}
  \KeywordTok{\textless{}link}\OtherTok{ rel=}\StringTok{"icon"}\OtherTok{ type=}\StringTok{"image/x{-}icon"}\OtherTok{ href=}\StringTok{"favicon.ico"}\KeywordTok{\textgreater{}}
\KeywordTok{\textless{}/head\textgreater{}}
\KeywordTok{\textless{}body\textgreater{}}
  \KeywordTok{\textless{}app{-}root\textgreater{}\textless{}/app{-}root\textgreater{}}
\KeywordTok{\textless{}/body\textgreater{}}
\KeywordTok{\textless{}/html\textgreater{}}
\end{Highlighting}
\end{Shaded}

\begin{figure}
\centering
\includegraphics{C:/Users/ZSH/Desktop/ng/ppt/images/2.jpg}
\caption{}
\end{figure}

\hypertarget{33-ux5171ux4eabux6a21ux5757}{%
\paragraph{3.3 共享模块}\label{33-ux5171ux4eabux6a21ux5757}}

共享模块当中放置的是 Angular 应用中模块级别的需要共享的组件或逻辑。

\begin{enumerate}
\def\labelenumi{\arabic{enumi}.}
\item
  创建共享模块: \texttt{ng\ g\ m\ shared}
\item
  创建共享组件:\texttt{ng\ g\ c\ shared/components/Layout}
\item
  在共享模块中导出共享组件

\begin{Shaded}
\begin{Highlighting}[]
\NormalTok{@}\FunctionTok{NgModule}\NormalTok{(\{}
  \DataTypeTok{declarations}\OperatorTok{:}\NormalTok{ [LayoutComponent]}\OperatorTok{,}
  \DataTypeTok{exports}\OperatorTok{:}\NormalTok{ [LayoutComponent]}
\NormalTok{\})}
\ImportTok{export} \KeywordTok{class}\NormalTok{ SharedModule \{\}}
\end{Highlighting}
\end{Shaded}
\item
  在根模块中导入共享模块

\begin{Shaded}
\begin{Highlighting}[]
\NormalTok{@}\FunctionTok{NgModule}\NormalTok{(\{}
  \DataTypeTok{declarations}\OperatorTok{:}\NormalTok{ [AppComponent]}\OperatorTok{,}
  \DataTypeTok{imports}\OperatorTok{:}\NormalTok{ [SharedModule]}\OperatorTok{,}
  \DataTypeTok{bootstrap}\OperatorTok{:}\NormalTok{ [AppComponent]}
\NormalTok{\})}
\ImportTok{export} \KeywordTok{class}\NormalTok{ AppModule \{\}}
\end{Highlighting}
\end{Shaded}
\item
  在根组件中使用 Layout 组件

\begin{Shaded}
\begin{Highlighting}[]
\NormalTok{@}\FunctionTok{Component}\NormalTok{(\{}
  \DataTypeTok{selector}\OperatorTok{:} \StringTok{"app{-}root"}\OperatorTok{,}
  \DataTypeTok{template}\OperatorTok{:} \VerbatimStringTok{\textasciigrave{}}
\VerbatimStringTok{    \textless{}div\textgreater{}App works\textless{}/div\textgreater{}}
\VerbatimStringTok{    \textless{}app{-}layout\textgreater{}\textless{}/app{-}layout\textgreater{}}
\VerbatimStringTok{  \textasciigrave{}}\OperatorTok{,}
  \DataTypeTok{styles}\OperatorTok{:}\NormalTok{ []}
\NormalTok{\})}
\ImportTok{export} \KeywordTok{class}\NormalTok{ AppComponent \{ \}}
\end{Highlighting}
\end{Shaded}
\end{enumerate}

\hypertarget{4-ux7ec4ux4ef6ux6a21ux677f}{%
\subsubsection{4. 组件模板}\label{4-ux7ec4ux4ef6ux6a21ux677f}}

\hypertarget{41-ux6570ux636eux7ed1ux5b9a}{%
\paragraph{4.1 数据绑定}\label{41-ux6570ux636eux7ed1ux5b9a}}

数据绑定就是将组件类中的数据显示在组件模板中,当组件类中的数据发生变化时会自动被同步到组件模板中(数据驱动
DOM )。

在 Angular 中使用差值表达式进行数据绑定,即 \{\{ \}\} 大胡子语法。

\begin{Shaded}
\begin{Highlighting}[]
\KeywordTok{\textless{}h2\textgreater{}}\NormalTok{\{\{message\}\}}\KeywordTok{\textless{}/h2\textgreater{}}
\KeywordTok{\textless{}h2\textgreater{}}\NormalTok{\{\{getInfo()\}\}}\KeywordTok{\textless{}/h2\textgreater{}}
\KeywordTok{\textless{}h2\textgreater{}}\NormalTok{\{\{a == b ? \textquotesingle{}相等\textquotesingle{}: \textquotesingle{}不等\textquotesingle{}\}\}}\KeywordTok{\textless{}/h2\textgreater{}}
\KeywordTok{\textless{}h2\textgreater{}}\NormalTok{\{\{\textquotesingle{}Hello Angular\textquotesingle{}\}\}}\KeywordTok{\textless{}/h2\textgreater{}}
\KeywordTok{\textless{}p}\OtherTok{ [innerHTML]=}\StringTok{"htmlSnippet"}\KeywordTok{\textgreater{}\textless{}/p\textgreater{}} \CommentTok{\textless{}!{-}{-} 对数据中的代码进行转义 {-}{-}\textgreater{}}
\end{Highlighting}
\end{Shaded}

\hypertarget{42-ux5c5eux6027ux7ed1ux5b9a}{%
\paragraph{4.2 属性绑定}\label{42-ux5c5eux6027ux7ed1ux5b9a}}

\hypertarget{421-ux666eux901aux5c5eux6027}{%
\subparagraph{4.2.1 普通属性}\label{421-ux666eux901aux5c5eux6027}}

属性绑定分为两种情况,绑定 DOM 对象属性和绑定HTML标记属性。

\begin{enumerate}
\def\labelenumi{\arabic{enumi}.}
\item
  使用 {[}属性名称{]} 为元素绑定 DOM 对象属性。
\end{enumerate}

\begin{Shaded}
\begin{Highlighting}[]
\KeywordTok{\textless{}img}\OtherTok{ [src]=}\StringTok{"imgUrl"}\KeywordTok{/\textgreater{}}
\end{Highlighting}
\end{Shaded}

\begin{enumerate}
\def\labelenumi{\arabic{enumi}.}
\item
  使用 {[}attr.属性名称{]} 为元素绑定 HTML 标记属性
\end{enumerate}

\begin{Shaded}
\begin{Highlighting}[]
\KeywordTok{\textless{}td}\OtherTok{ [attr.colspan]=}\StringTok{"colSpan"}\KeywordTok{\textgreater{}\textless{}/td\textgreater{}} 
\end{Highlighting}
\end{Shaded}

在大多数情况下,DOM 对象属性和 HTML
标记属性是对应的关系,所以使用第一种情况。但是某些属性只有 HTML
标记存在,DOM 对象中不存在,此时需要使用第二种情况,比如 colspan
属性,在 DOM 对象中就没有,或者自定义 HTML 属性也需要使用第二种情况。

\hypertarget{422-class-ux5c5eux6027}{%
\subparagraph{4.2.2 class 属性}\label{422-class-ux5c5eux6027}}

\begin{Shaded}
\begin{Highlighting}[]
\KeywordTok{\textless{}button}\OtherTok{ class=}\StringTok{"btn btn{-}primary"}\OtherTok{ [class.active]=}\StringTok{"isActive"}\KeywordTok{\textgreater{}}\NormalTok{按钮}\KeywordTok{\textless{}/button\textgreater{}}
\KeywordTok{\textless{}div}\OtherTok{ [ngClass]=}\StringTok{"\{\textquotesingle{}active\textquotesingle{}: true, \textquotesingle{}error\textquotesingle{}: true\}"}\KeywordTok{\textgreater{}\textless{}/div\textgreater{}}
\end{Highlighting}
\end{Shaded}

\hypertarget{423-style-ux5c5eux6027}{%
\subparagraph{4.2.3 style 属性}\label{423-style-ux5c5eux6027}}

\begin{Shaded}
\begin{Highlighting}[]
\KeywordTok{\textless{}button}\OtherTok{ [style.backgroundColor]=}\StringTok{"isActive ? \textquotesingle{}blue\textquotesingle{}: \textquotesingle{}red\textquotesingle{}"}\KeywordTok{\textgreater{}}\NormalTok{按钮}\KeywordTok{\textless{}/button\textgreater{}}
\KeywordTok{\textless{}button}\OtherTok{ [ngStyle]=}\StringTok{"\{\textquotesingle{}backgroundColor\textquotesingle{}: \textquotesingle{}red\textquotesingle{}\}"}\KeywordTok{\textgreater{}}\NormalTok{按钮}\KeywordTok{\textless{}/button\textgreater{}}
\end{Highlighting}
\end{Shaded}

\hypertarget{43-ux4e8bux4ef6ux7ed1ux5b9a}{%
\paragraph{4.3 事件绑定}\label{43-ux4e8bux4ef6ux7ed1ux5b9a}}

\begin{Shaded}
\begin{Highlighting}[]
\KeywordTok{\textless{}button}\OtherTok{ (click)=}\StringTok{"onSave($event)"}\KeywordTok{\textgreater{}}\NormalTok{按钮}\KeywordTok{\textless{}/button\textgreater{}}
\CommentTok{\textless{}!{-}{-} 当按下回车键抬起的时候执行函数 {-}{-}\textgreater{}}
\KeywordTok{\textless{}input}\OtherTok{ type=}\StringTok{"text"}\OtherTok{ (keyup.enter)=}\StringTok{"onKeyUp()"}\KeywordTok{/\textgreater{}}
\end{Highlighting}
\end{Shaded}

\begin{Shaded}
\begin{Highlighting}[]
\ImportTok{export} \KeywordTok{class}\NormalTok{ AppComponent \{}
\NormalTok{  title }\OperatorTok{=} \StringTok{"test"}
  \FunctionTok{onSave}\NormalTok{(}\DataTypeTok{event}\OperatorTok{:} \BuiltInTok{Event}\NormalTok{) \{}
    \CommentTok{// this 指向组件类的实例对象}
    \KeywordTok{this}\OperatorTok{.}\AttributeTok{title} \CommentTok{// "test"}
\NormalTok{  \}}
\NormalTok{\}}
\end{Highlighting}
\end{Shaded}

\hypertarget{44-ux83b7ux53d6ux539fux751f-dom-ux5bf9ux8c61}{%
\paragraph{4.4 获取原生 DOM
对象}\label{44-ux83b7ux53d6ux539fux751f-dom-ux5bf9ux8c61}}

\hypertarget{441-ux5728ux7ec4ux4ef6ux6a21ux677fux4e2dux83b7ux53d6}{%
\subparagraph{4.4.1
在组件模板中获取}\label{441-ux5728ux7ec4ux4ef6ux6a21ux677fux4e2dux83b7ux53d6}}

\begin{Shaded}
\begin{Highlighting}[]
\KeywordTok{\textless{}input}\OtherTok{ type=}\StringTok{"text"}\OtherTok{ (keyup.enter)=}\StringTok{"onKeyUp(username.value)"}\OtherTok{ \#username}\KeywordTok{/\textgreater{}}
\end{Highlighting}
\end{Shaded}

\hypertarget{442-ux5728ux7ec4ux4ef6ux7c7bux4e2dux83b7ux53d6}{%
\subparagraph{4.4.2
在组件类中获取}\label{442-ux5728ux7ec4ux4ef6ux7c7bux4e2dux83b7ux53d6}}

使用 ViewChild 装饰器获取一个元素

\begin{Shaded}
\begin{Highlighting}[]
\KeywordTok{\textless{}p}\OtherTok{ \#paragraph}\KeywordTok{\textgreater{}}\NormalTok{home works!}\KeywordTok{\textless{}/p\textgreater{}}
\end{Highlighting}
\end{Shaded}

\begin{Shaded}
\begin{Highlighting}[]
\ImportTok{import}\NormalTok{ \{ AfterViewInit}\OperatorTok{,}\NormalTok{ ElementRef}\OperatorTok{,}\NormalTok{ ViewChild \} }\ImportTok{from} \StringTok{"@angular/core"}

\ImportTok{export} \KeywordTok{class}\NormalTok{ HomeComponent }\KeywordTok{implements}\NormalTok{ AfterViewInit \{}
\NormalTok{  @}\FunctionTok{ViewChild}\NormalTok{(}\StringTok{"paragraph"}\NormalTok{) }\DataTypeTok{paragraph}\OperatorTok{:}\NormalTok{ ElementRef}\OperatorTok{\textless{}}\BuiltInTok{HTMLParagraphElement}\OperatorTok{\textgreater{}} \OperatorTok{|} \KeywordTok{undefined}
  \FunctionTok{ngAfterViewInit}\NormalTok{() \{}
    \BuiltInTok{console}\OperatorTok{.}\FunctionTok{log}\NormalTok{(}\KeywordTok{this}\OperatorTok{.}\AttributeTok{paragraph}\OperatorTok{?.}\AttributeTok{nativeElement}\NormalTok{)}
\NormalTok{  \}}
\NormalTok{\}}
\end{Highlighting}
\end{Shaded}

使用 ViewChildren 获取一组元素

\begin{Shaded}
\begin{Highlighting}[]
\KeywordTok{\textless{}ul\textgreater{}}
  \KeywordTok{\textless{}li}\OtherTok{ \#items}\KeywordTok{\textgreater{}}\NormalTok{a}\KeywordTok{\textless{}/li\textgreater{}}
  \KeywordTok{\textless{}li}\OtherTok{ \#items}\KeywordTok{\textgreater{}}\NormalTok{b}\KeywordTok{\textless{}/li\textgreater{}}
  \KeywordTok{\textless{}li}\OtherTok{ \#items}\KeywordTok{\textgreater{}}\NormalTok{c}\KeywordTok{\textless{}/li\textgreater{}}
\KeywordTok{\textless{}/ul\textgreater{}}
\end{Highlighting}
\end{Shaded}

\begin{Shaded}
\begin{Highlighting}[]
\ImportTok{import}\NormalTok{ \{ AfterViewInit}\OperatorTok{,}\NormalTok{ QueryList}\OperatorTok{,}\NormalTok{ ViewChildren \} }\ImportTok{from} \StringTok{"@angular/core"}

\NormalTok{@}\FunctionTok{Component}\NormalTok{(\{}
  \DataTypeTok{selector}\OperatorTok{:} \StringTok{"app{-}home"}\OperatorTok{,}
  \DataTypeTok{templateUrl}\OperatorTok{:} \StringTok{"./home.component.html"}\OperatorTok{,}
  \DataTypeTok{styles}\OperatorTok{:}\NormalTok{ []}
\NormalTok{\})}
\ImportTok{export} \KeywordTok{class}\NormalTok{ HomeComponent }\KeywordTok{implements}\NormalTok{ AfterViewInit \{}
\NormalTok{  @}\FunctionTok{ViewChildren}\NormalTok{(}\StringTok{"items"}\NormalTok{) }\DataTypeTok{items}\OperatorTok{:}\NormalTok{ QueryList}\OperatorTok{\textless{}}\BuiltInTok{HTMLLIElement}\OperatorTok{\textgreater{}} \OperatorTok{|} \KeywordTok{undefined}
  \FunctionTok{ngAfterViewInit}\NormalTok{() \{}
    \BuiltInTok{console}\OperatorTok{.}\FunctionTok{log}\NormalTok{(}\KeywordTok{this}\OperatorTok{.}\AttributeTok{items}\OperatorTok{?.}\FunctionTok{toArray}\NormalTok{())}
\NormalTok{  \}}
\NormalTok{\}}
\end{Highlighting}
\end{Shaded}

\hypertarget{45-ux53ccux5411ux6570ux636eux7ed1ux5b9a}{%
\paragraph{4.5
双向数据绑定}\label{45-ux53ccux5411ux6570ux636eux7ed1ux5b9a}}

数据在组件类和组件模板中双向同步。

Angular 将双向数据绑定功能放在了 @angular/forms
模块中,所以要实现双向数据绑定需要依赖该模块。

\begin{Shaded}
\begin{Highlighting}[]
\ImportTok{import}\NormalTok{ \{ FormsModule \} }\ImportTok{from} \StringTok{"@angular/forms"}

\NormalTok{@}\FunctionTok{NgModule}\NormalTok{(\{}
  \DataTypeTok{imports}\OperatorTok{:}\NormalTok{ [FormsModule]}\OperatorTok{,}
\NormalTok{\})}
\ImportTok{export} \KeywordTok{class}\NormalTok{ AppModule \{\}}
\end{Highlighting}
\end{Shaded}

\begin{Shaded}
\begin{Highlighting}[]
\KeywordTok{\textless{}input}\OtherTok{ type=}\StringTok{"text"}\OtherTok{ [}\ErrorTok{(ngModel)]}\OtherTok{=}\StringTok{"username"} \KeywordTok{/\textgreater{}}
\KeywordTok{\textless{}button}\OtherTok{ (click)=}\StringTok{"change()"}\KeywordTok{\textgreater{}}\NormalTok{在组件类中更改 username}\KeywordTok{\textless{}/button\textgreater{}}
\KeywordTok{\textless{}div\textgreater{}}\NormalTok{username: \{\{ username \}\}}\KeywordTok{\textless{}/div\textgreater{}}
\end{Highlighting}
\end{Shaded}

\begin{Shaded}
\begin{Highlighting}[]
\ImportTok{export} \KeywordTok{class}\NormalTok{ AppComponent \{}
  \DataTypeTok{username}\OperatorTok{:}\NormalTok{ string }\OperatorTok{=} \StringTok{""}
  \FunctionTok{change}\NormalTok{() \{}
    \KeywordTok{this}\OperatorTok{.}\AttributeTok{username} \OperatorTok{=} \StringTok{"hello Angular"}
\NormalTok{  \}}
\NormalTok{\}}
\end{Highlighting}
\end{Shaded}

\hypertarget{46-ux5185ux5bb9ux6295ux5f71}{%
\paragraph{4.6 内容投影}\label{46-ux5185ux5bb9ux6295ux5f71}}

\begin{Shaded}
\begin{Highlighting}[]
\CommentTok{\textless{}!{-}{-} app.component.html {-}{-}\textgreater{}}
\KeywordTok{\textless{}bootstrap{-}panel\textgreater{}}
	\KeywordTok{\textless{}div}\OtherTok{ class=}\StringTok{"heading"}\KeywordTok{\textgreater{}}
\NormalTok{        Heading}
  \KeywordTok{\textless{}/div\textgreater{}}
  \KeywordTok{\textless{}div}\OtherTok{ class=}\StringTok{"body"}\KeywordTok{\textgreater{}}
\NormalTok{        Body}
  \KeywordTok{\textless{}/div\textgreater{}}
\KeywordTok{\textless{}/bootstrap{-}panel\textgreater{}}
\end{Highlighting}
\end{Shaded}

\begin{Shaded}
\begin{Highlighting}[]
\CommentTok{\textless{}!{-}{-} panel.component.html {-}{-}\textgreater{}}
\KeywordTok{\textless{}div}\OtherTok{ class=}\StringTok{"panel panel{-}default"}\KeywordTok{\textgreater{}}
  \KeywordTok{\textless{}div}\OtherTok{ class=}\StringTok{"panel{-}heading"}\KeywordTok{\textgreater{}}
    \KeywordTok{\textless{}ng{-}content}\OtherTok{ select=}\StringTok{".heading"}\KeywordTok{\textgreater{}\textless{}/ng{-}content\textgreater{}}
  \KeywordTok{\textless{}/div\textgreater{}}
  \KeywordTok{\textless{}div}\OtherTok{ class=}\StringTok{"panel{-}body"}\KeywordTok{\textgreater{}}
    \KeywordTok{\textless{}ng{-}content}\OtherTok{ select=}\StringTok{".body"}\KeywordTok{\textgreater{}\textless{}/ng{-}content\textgreater{}}
  \KeywordTok{\textless{}/div\textgreater{}}
\KeywordTok{\textless{}/div\textgreater{}}
\end{Highlighting}
\end{Shaded}

如果只有一个ng-content,不需要select属性。

ng-content在浏览器中会被 \textless{}div
class="heading"\textgreater{}\textless{}/div\textgreater{}
替代,如果不想要这个额外的div,可以使用ng-container替代这个div。

\begin{Shaded}
\begin{Highlighting}[]
\CommentTok{\textless{}!{-}{-} app.component.html {-}{-}\textgreater{}}
\KeywordTok{\textless{}bootstrap{-}panel\textgreater{}}
	\KeywordTok{\textless{}ng{-}container}\OtherTok{ class=}\StringTok{"heading"}\KeywordTok{\textgreater{}}
\NormalTok{        Heading}
    \KeywordTok{\textless{}/ng{-}container\textgreater{}}
    \KeywordTok{\textless{}ng{-}container}\OtherTok{ class=}\StringTok{"body"}\KeywordTok{\textgreater{}}
\NormalTok{        Body}
    \KeywordTok{\textless{}/ng{-}container\textgreater{}}
\KeywordTok{\textless{}/bootstrap{-}panel\textgreater{}}
\end{Highlighting}
\end{Shaded}

\hypertarget{47--ux6570ux636eux7ed1ux5b9aux5bb9ux9519ux5904ux7406}{%
\paragraph{4.7
数据绑定容错处理}\label{47--ux6570ux636eux7ed1ux5b9aux5bb9ux9519ux5904ux7406}}

\begin{Shaded}
\begin{Highlighting}[]
\CommentTok{// app.component.ts}
\ImportTok{export} \KeywordTok{class}\NormalTok{ AppComponent \{}
\NormalTok{    task }\OperatorTok{=}\NormalTok{ \{}
        \DataTypeTok{person}\OperatorTok{:}\NormalTok{ \{}
            \DataTypeTok{name}\OperatorTok{:} \StringTok{\textquotesingle{}张三\textquotesingle{}}
\NormalTok{        \}}
\NormalTok{    \}}
\NormalTok{\}}
\end{Highlighting}
\end{Shaded}

\begin{Shaded}
\begin{Highlighting}[]
\CommentTok{\textless{}!{-}{-} 方式一 {-}{-}\textgreater{}}
\KeywordTok{\textless{}span}\OtherTok{ *ngIf=}\StringTok{"task.person"}\KeywordTok{\textgreater{}}\NormalTok{\{\{ task.person.name \}\}}\KeywordTok{\textless{}/span\textgreater{}}
\CommentTok{\textless{}!{-}{-} 方式二 {-}{-}\textgreater{}}
\KeywordTok{\textless{}span\textgreater{}}\NormalTok{\{\{ task.person?.name \}\}}\KeywordTok{\textless{}/span\textgreater{}}
\end{Highlighting}
\end{Shaded}

\hypertarget{48-ux5168ux5c40ux6837ux5f0f}{%
\paragraph{4.8 全局样式}\label{48-ux5168ux5c40ux6837ux5f0f}}

\begin{Shaded}
\begin{Highlighting}[]
\CommentTok{/* 第一种方式 在 styles.css 文件中 */}
\ImportTok{@import} \StringTok{"\textasciitilde{}bootstrap/dist/css/bootstrap.css"}\OperatorTok{;}
\CommentTok{/* \textasciitilde{} 相对node\_modules文件夹 */}
\end{Highlighting}
\end{Shaded}

\begin{Shaded}
\begin{Highlighting}[]
\CommentTok{\textless{}!{-}{-} 第二种方式 在 index.html 文件中  {-}{-}\textgreater{}}
\KeywordTok{\textless{}link}\OtherTok{ href=}\StringTok{"https://cdn.jsdelivr.net/npm/bootstrap@3.3.7/dist/css/bootstrap.min.css"}\OtherTok{ rel=}\StringTok{"stylesheet"} \KeywordTok{/\textgreater{}}
\end{Highlighting}
\end{Shaded}

\begin{Shaded}
\begin{Highlighting}[]
\CommentTok{// 第三种方式 在 angular.json 文件中}
\StringTok{"styles"}\OperatorTok{:}\NormalTok{ [}
  \StringTok{"./node\_modules/bootstrap/dist/css/bootstrap.min.css"}\OperatorTok{,}
  \StringTok{"src/styles.css"}
\NormalTok{]}
\end{Highlighting}
\end{Shaded}

\hypertarget{5-ux6307ux4ee4-directive}{%
\subsubsection{5. 指令 Directive}\label{5-ux6307ux4ee4-directive}}

指令是 Angular 提供的操作 DOM 的途径。指令分为属性指令和结构指令。

属性指令:修改现有元素的外观或行为,使用 {[}{]} 包裹。

结构指令:增加、删除 DOM 节点以修改布局,使用*作为指令前缀

\hypertarget{51-ux5185ux7f6eux6307ux4ee4}{%
\paragraph{5.1 内置指令}\label{51-ux5185ux7f6eux6307ux4ee4}}

\hypertarget{511-ngif}{%
\subparagraph{\texorpdfstring{5.1.1 *ngIf
}{5.1.1 *ngIf }}\label{511-ngif}}

根据条件渲染 DOM 节点或移除 DOM 节点。

\begin{Shaded}
\begin{Highlighting}[]
\KeywordTok{\textless{}div}\OtherTok{ *ngIf=}\StringTok{"data.length == 0"}\KeywordTok{\textgreater{}}\NormalTok{没有更多数据}\KeywordTok{\textless{}/div\textgreater{}}
\end{Highlighting}
\end{Shaded}

\begin{Shaded}
\begin{Highlighting}[]
\KeywordTok{\textless{}div}\OtherTok{ *ngIf=}\StringTok{"data.length \textgreater{} 0; then dataList else noData"}\KeywordTok{\textgreater{}\textless{}/div\textgreater{}}
\KeywordTok{\textless{}ng{-}template}\OtherTok{ \#dataList}\KeywordTok{\textgreater{}}\NormalTok{课程列表}\KeywordTok{\textless{}/ng{-}template\textgreater{}}
\KeywordTok{\textless{}ng{-}template}\OtherTok{ \#noData}\KeywordTok{\textgreater{}}\NormalTok{没有更多数据}\KeywordTok{\textless{}/ng{-}template\textgreater{}}
\end{Highlighting}
\end{Shaded}

\hypertarget{512-hidden}{%
\subparagraph{5.1.2 {[}hidden{]}}\label{512-hidden}}

根据条件显示 DOM 节点或隐藏 DOM 节点 (display)。

\begin{Shaded}
\begin{Highlighting}[]
\KeywordTok{\textless{}div}\OtherTok{ [hidden]=}\StringTok{"data.length == 0"}\KeywordTok{\textgreater{}}\NormalTok{课程列表}\KeywordTok{\textless{}/div\textgreater{}}
\KeywordTok{\textless{}div}\OtherTok{ [hidden]=}\StringTok{"data.length \textgreater{} 0"}\KeywordTok{\textgreater{}}\NormalTok{没有更多数据}\KeywordTok{\textless{}/div\textgreater{}}
\end{Highlighting}
\end{Shaded}

\hypertarget{513-ngfor}{%
\subparagraph{5.1.3 *ngFor}\label{513-ngfor}}

遍历数据生成HTML结构

\begin{Shaded}
\begin{Highlighting}[]
\KeywordTok{interface}\NormalTok{ List \{}
  \DataTypeTok{id}\OperatorTok{:}\NormalTok{ number}
  \DataTypeTok{name}\OperatorTok{:}\NormalTok{ string}
  \DataTypeTok{age}\OperatorTok{:}\NormalTok{ number}
\NormalTok{\}}

\NormalTok{list}\OperatorTok{:}\NormalTok{ List[] }\OperatorTok{=}\NormalTok{ [}
\NormalTok{  \{ }\DataTypeTok{id}\OperatorTok{:} \DecValTok{1}\OperatorTok{,} \DataTypeTok{name}\OperatorTok{:} \StringTok{"张三"}\OperatorTok{,} \DataTypeTok{age}\OperatorTok{:} \DecValTok{20}\NormalTok{ \}}\OperatorTok{,}
\NormalTok{  \{ }\DataTypeTok{id}\OperatorTok{:} \DecValTok{2}\OperatorTok{,} \DataTypeTok{name}\OperatorTok{:} \StringTok{"李四"}\OperatorTok{,} \DataTypeTok{age}\OperatorTok{:} \DecValTok{30}\NormalTok{ \}}
\NormalTok{]}
\end{Highlighting}
\end{Shaded}

\begin{Shaded}
\begin{Highlighting}[]
\KeywordTok{\textless{}li}
\OtherTok{    *ngFor=}\StringTok{"}
\StringTok{      let item of list;}
\StringTok{      let i = index;}
\StringTok{      let isEven = even;}
\StringTok{      let isOdd = odd;}
\StringTok{      let isFirst = first;}
\StringTok{      let isLast = last;}
\StringTok{    "}
  \KeywordTok{\textgreater{}}
  \KeywordTok{\textless{}/li\textgreater{}}
\end{Highlighting}
\end{Shaded}

\begin{Shaded}
\begin{Highlighting}[]
\KeywordTok{\textless{}li}\OtherTok{ *ngFor=}\StringTok{"let item of list; trackBy: identify"}\KeywordTok{\textgreater{}\textless{}/li\textgreater{}}
\end{Highlighting}
\end{Shaded}

\begin{Shaded}
\begin{Highlighting}[]
\FunctionTok{identify}\NormalTok{(index}\OperatorTok{,}\NormalTok{ item)\{}
  \ControlFlowTok{return}\NormalTok{ item}\OperatorTok{.}\AttributeTok{id}\OperatorTok{;} 
\NormalTok{\}}
\end{Highlighting}
\end{Shaded}

\hypertarget{52-ux81eaux5b9aux4e49ux6307ux4ee4}{%
\paragraph{5.2 自定义指令}\label{52-ux81eaux5b9aux4e49ux6307ux4ee4}}

需求:为元素设置默认背景颜色,鼠标移入时的背景颜色以及移出时的背景颜色。

\begin{Shaded}
\begin{Highlighting}[]
\KeywordTok{\textless{}div}\OtherTok{ [appHover]=}\StringTok{"\{ bgColor: \textquotesingle{}skyblue\textquotesingle{} \}"}\KeywordTok{\textgreater{}}\NormalTok{Hello Angular}\KeywordTok{\textless{}/div\textgreater{}}
\end{Highlighting}
\end{Shaded}

\begin{Shaded}
\begin{Highlighting}[]
\ImportTok{import}\NormalTok{ \{ AfterViewInit}\OperatorTok{,}\NormalTok{ Directive}\OperatorTok{,}\NormalTok{ ElementRef}\OperatorTok{,}\NormalTok{ HostListener}\OperatorTok{,}\NormalTok{ Input \} }\ImportTok{from} \StringTok{"@angular/core"}

\CommentTok{// 接收参的数类型}
\KeywordTok{interface}\NormalTok{ Options \{}
\NormalTok{  bgColor}\OperatorTok{?:}\NormalTok{ string}
\NormalTok{\}}

\NormalTok{@}\FunctionTok{Directive}\NormalTok{(\{}
  \DataTypeTok{selector}\OperatorTok{:} \StringTok{"[appHover]"}
\NormalTok{\})}
\ImportTok{export} \KeywordTok{class}\NormalTok{ HoverDirective }\KeywordTok{implements}\NormalTok{ AfterViewInit \{}
  \CommentTok{// 接收参数}
\NormalTok{  @}\FunctionTok{Input}\NormalTok{(}\StringTok{"appHover"}\NormalTok{) }\DataTypeTok{appHover}\OperatorTok{:}\NormalTok{ Options }\OperatorTok{=}\NormalTok{ \{\}}
  \CommentTok{// 要操作的 DOM 节点}
  \DataTypeTok{element}\OperatorTok{:} \BuiltInTok{HTMLElement}
	\CommentTok{// 获取要操作的 DOM 节点}
  \FunctionTok{constructor}\NormalTok{(}\KeywordTok{private} \DataTypeTok{elementRef}\OperatorTok{:}\NormalTok{ ElementRef) \{}
    \KeywordTok{this}\OperatorTok{.}\AttributeTok{element} \OperatorTok{=} \KeywordTok{this}\OperatorTok{.}\AttributeTok{elementRef}\OperatorTok{.}\AttributeTok{nativeElement}
\NormalTok{  \}}
	\CommentTok{// 组件模板初始完成后设置元素的背景颜色}
  \FunctionTok{ngAfterViewInit}\NormalTok{() \{}
    \KeywordTok{this}\OperatorTok{.}\AttributeTok{element}\OperatorTok{.}\AttributeTok{style}\OperatorTok{.}\AttributeTok{backgroundColor} \OperatorTok{=} \KeywordTok{this}\OperatorTok{.}\AttributeTok{appHover}\OperatorTok{.}\AttributeTok{bgColor} \OperatorTok{||} \StringTok{"skyblue"}
\NormalTok{  \}}
	\CommentTok{// 为元素添加鼠标移入事件}
\NormalTok{  @}\FunctionTok{HostListener}\NormalTok{(}\StringTok{"mouseenter"}\NormalTok{) }\FunctionTok{enter}\NormalTok{() \{}
    \KeywordTok{this}\OperatorTok{.}\AttributeTok{element}\OperatorTok{.}\AttributeTok{style}\OperatorTok{.}\AttributeTok{backgroundColor} \OperatorTok{=} \StringTok{"pink"}
\NormalTok{  \}}
	\CommentTok{// 为元素添加鼠标移出事件}
\NormalTok{  @}\FunctionTok{HostListener}\NormalTok{(}\StringTok{"mouseleave"}\NormalTok{) }\FunctionTok{leave}\NormalTok{() \{}
    \KeywordTok{this}\OperatorTok{.}\AttributeTok{element}\OperatorTok{.}\AttributeTok{style}\OperatorTok{.}\AttributeTok{backgroundColor} \OperatorTok{=} \StringTok{"skyblue"}
\NormalTok{  \}}
\NormalTok{\}}
\end{Highlighting}
\end{Shaded}

\hypertarget{6-ux7ba1ux9053-pipe}{%
\subsubsection{6. 管道 Pipe}\label{6-ux7ba1ux9053-pipe}}

管道的作用是格式化组件模板数据。

\hypertarget{61-ux5185ux7f6eux7ba1ux9053}{%
\paragraph{6.1 内置管道}\label{61-ux5185ux7f6eux7ba1ux9053}}

\begin{enumerate}
\def\labelenumi{\arabic{enumi}.}
\item
  date 日期格式化
\item
  currency 货币格式化
\item
  uppercase 转大写
\item
  lowercase 转小写
\item
  json 格式化json 数据
\end{enumerate}

\begin{Shaded}
\begin{Highlighting}[]
\NormalTok{\{\{ date | date: "yyyy{-}MM{-}dd" \}\}}
\end{Highlighting}
\end{Shaded}

\hypertarget{62-ux81eaux5b9aux4e49ux7ba1ux9053}{%
\paragraph{6.2 自定义管道}\label{62-ux81eaux5b9aux4e49ux7ba1ux9053}}

需求:指定字符串不能超过规定的长度

\begin{Shaded}
\begin{Highlighting}[]
\CommentTok{// summary.pipe.ts}
\ImportTok{import}\NormalTok{ \{ }\BuiltInTok{Pipe}\OperatorTok{,}\NormalTok{ PipeTransform \} }\ImportTok{from} \StringTok{\textquotesingle{}@angular/core\textquotesingle{}}\OperatorTok{;}

\NormalTok{@}\FunctionTok{Pipe}\NormalTok{(\{}
   \DataTypeTok{name}\OperatorTok{:} \StringTok{\textquotesingle{}summary\textquotesingle{}} 
\NormalTok{\})}\OperatorTok{;}
\ImportTok{export} \KeywordTok{class}\NormalTok{ SummaryPipe }\KeywordTok{implements}\NormalTok{ PipeTransform \{}
    \FunctionTok{transform}\NormalTok{ (}\DataTypeTok{value}\OperatorTok{:}\NormalTok{ string}\OperatorTok{,}\NormalTok{ limit}\OperatorTok{?:}\NormalTok{ number) \{}
        \ControlFlowTok{if}\NormalTok{ (}\OperatorTok{!}\NormalTok{value) }\ControlFlowTok{return} \KeywordTok{null}\OperatorTok{;}
        \KeywordTok{let}\NormalTok{ actualLimit }\OperatorTok{=}\NormalTok{ (limit) }\OperatorTok{?}\NormalTok{ limit }\OperatorTok{:} \DecValTok{50}\OperatorTok{;}
        \ControlFlowTok{return}\NormalTok{ value}\OperatorTok{.}\FunctionTok{substr}\NormalTok{(}\DecValTok{0}\OperatorTok{,}\NormalTok{ actualLimit) }\OperatorTok{+} \StringTok{\textquotesingle{}...\textquotesingle{}}\OperatorTok{;}
\NormalTok{    \}}
\NormalTok{\}}
\end{Highlighting}
\end{Shaded}

\begin{Shaded}
\begin{Highlighting}[]
\CommentTok{// app.module.ts}
\ImportTok{import}\NormalTok{ \{ SummaryPipe \} from }\StringTok{\textquotesingle{}./summary.pipe\textquotesingle{}}
\NormalTok{@}\FunctionTok{NgModule}\NormalTok{(\{}
\NormalTok{    declarations}\OperatorTok{:}\NormalTok{ [}
\NormalTok{      SummaryPipe}
\NormalTok{    ] }
\NormalTok{\})}\OperatorTok{;}
\end{Highlighting}
\end{Shaded}

\hypertarget{7-ux7ec4ux4ef6ux901aux8baf}{%
\subsubsection{7. 组件通讯}\label{7-ux7ec4ux4ef6ux901aux8baf}}

\hypertarget{71-ux5411ux7ec4ux4ef6ux5185ux90e8ux4f20ux9012ux6570ux636e}{%
\paragraph{7.1
向组件内部传递数据}\label{71-ux5411ux7ec4ux4ef6ux5185ux90e8ux4f20ux9012ux6570ux636e}}

\begin{Shaded}
\begin{Highlighting}[]
\KeywordTok{\textless{}app{-}favorite}\OtherTok{ [isFavorite]=}\StringTok{"true"}\KeywordTok{\textgreater{}\textless{}/app{-}favorite\textgreater{}}
\end{Highlighting}
\end{Shaded}

\begin{Shaded}
\begin{Highlighting}[]
\CommentTok{// favorite.component.ts}
\ImportTok{import}\NormalTok{ \{ Input \} }\ImportTok{from} \StringTok{\textquotesingle{}@angular/core\textquotesingle{}}\OperatorTok{;}
\ImportTok{export} \KeywordTok{class}\NormalTok{ FavoriteComponent \{}
\NormalTok{    @}\FunctionTok{Input}\NormalTok{() }\DataTypeTok{isFavorite}\OperatorTok{:}\NormalTok{ boolean }\OperatorTok{=} \KeywordTok{false}\OperatorTok{;}
\NormalTok{\}}
\end{Highlighting}
\end{Shaded}

注意:在属性的外面加 {[}{]}
表示绑定动态值,在组件内接收后是布尔类型,不加 {[}{]}
表示绑定普通值,在组件内接收后是字符串类型。

\begin{Shaded}
\begin{Highlighting}[]
\KeywordTok{\textless{}app{-}favorite}\OtherTok{ [is{-}Favorite]=}\StringTok{"true"}\KeywordTok{\textgreater{}\textless{}/app{-}favorite\textgreater{}}
\end{Highlighting}
\end{Shaded}

\begin{Shaded}
\begin{Highlighting}[]
\ImportTok{import}\NormalTok{ \{ Input \} }\ImportTok{from} \StringTok{\textquotesingle{}@angular/core\textquotesingle{}}\OperatorTok{;}

\ImportTok{export} \KeywordTok{class}\NormalTok{ FavoriteComponent \{}
\NormalTok{  @}\FunctionTok{Input}\NormalTok{(}\StringTok{"is{-}Favorite"}\NormalTok{) }\DataTypeTok{isFavorite}\OperatorTok{:}\NormalTok{ boolean }\OperatorTok{=} \KeywordTok{false}
\NormalTok{\}}
\end{Highlighting}
\end{Shaded}

\hypertarget{72-ux7ec4ux4ef6ux5411ux5916ux90e8ux4f20ux9012ux6570ux636e}{%
\paragraph{7.2
组件向外部传递数据}\label{72-ux7ec4ux4ef6ux5411ux5916ux90e8ux4f20ux9012ux6570ux636e}}

需求:在子组件中通过点击按钮将数据传递给父组件

\begin{Shaded}
\begin{Highlighting}[]
\CommentTok{\textless{}!{-}{-} 子组件模板 {-}{-}\textgreater{}}
\KeywordTok{\textless{}button}\OtherTok{ (click)=}\StringTok{"onClick()"}\KeywordTok{\textgreater{}}\NormalTok{click}\KeywordTok{\textless{}/button\textgreater{}}
\end{Highlighting}
\end{Shaded}

\begin{Shaded}
\begin{Highlighting}[]
\CommentTok{// 子组件类}
\ImportTok{import}\NormalTok{ \{ }\BuiltInTok{EventEmitter}\OperatorTok{,}\NormalTok{ Output \} }\ImportTok{from} \StringTok{"@angular/core"}

\ImportTok{export} \KeywordTok{class}\NormalTok{ FavoriteComponent \{}
\NormalTok{  @}\FunctionTok{Output}\NormalTok{() change }\OperatorTok{=} \KeywordTok{new} \BuiltInTok{EventEmitter}\NormalTok{()}
  \FunctionTok{onClick}\NormalTok{() \{}
    \KeywordTok{this}\OperatorTok{.}\AttributeTok{change}\OperatorTok{.}\FunctionTok{emit}\NormalTok{(\{ }\DataTypeTok{name}\OperatorTok{:} \StringTok{"张三"}\NormalTok{ \})}
\NormalTok{  \}}
\NormalTok{\}}
\end{Highlighting}
\end{Shaded}

\begin{Shaded}
\begin{Highlighting}[]
\CommentTok{\textless{}!{-}{-} 父组件模板 {-}{-}\textgreater{}}
\KeywordTok{\textless{}app{-}favorite}\OtherTok{ (change)=}\StringTok{"onChange($event)"}\KeywordTok{\textgreater{}\textless{}/app{-}favorite\textgreater{}}
\end{Highlighting}
\end{Shaded}

\begin{Shaded}
\begin{Highlighting}[]
\CommentTok{// 父组件类}
\ImportTok{export} \KeywordTok{class}\NormalTok{ AppComponent \{}
  \FunctionTok{onChange}\NormalTok{(}\DataTypeTok{event}\OperatorTok{:}\NormalTok{ \{ }\DataTypeTok{name}\OperatorTok{:}\NormalTok{ string \}) \{}
    \BuiltInTok{console}\OperatorTok{.}\FunctionTok{log}\NormalTok{(}\BuiltInTok{event}\NormalTok{)}
\NormalTok{  \}}
\NormalTok{\}}
\end{Highlighting}
\end{Shaded}

\hypertarget{8-ux7ec4ux4ef6ux751fux547dux5468ux671f}{%
\subsubsection{8.
组件生命周期}\label{8-ux7ec4ux4ef6ux751fux547dux5468ux671f}}

\begin{figure}
\centering
\includegraphics{C:/Users/ZSH/Desktop/ng/ppt/images/4.png}
\caption{}
\end{figure}

\hypertarget{81-ux6302ux8f7dux9636ux6bb5}{%
\paragraph{8.1 挂载阶段}\label{81-ux6302ux8f7dux9636ux6bb5}}

挂载阶段的生命周期函数只在挂载阶段执行一次,数据更新时不再执行。

\begin{enumerate}
\def\labelenumi{\arabic{enumi}.}
\item
  constructor

  Angular 在实例化组件类时执行, 可以用来接收 Angular
  注入的服务实例对象。

\begin{Shaded}
\begin{Highlighting}[]
\ImportTok{export} \KeywordTok{class}\NormalTok{ ChildComponent \{}
  \FunctionTok{constructor}\NormalTok{ (}\KeywordTok{private} \DataTypeTok{test}\OperatorTok{:}\NormalTok{ TestService) \{}
    \BuiltInTok{console}\OperatorTok{.}\FunctionTok{log}\NormalTok{(}\KeywordTok{this}\OperatorTok{.}\AttributeTok{test}\NormalTok{) }\CommentTok{// "test"}
\NormalTok{  \}}
\NormalTok{\}}
\end{Highlighting}
\end{Shaded}
\item
  ngOnInit

  在首次接收到输入属性值后执行,在此处可以执行请求操作。

\begin{Shaded}
\begin{Highlighting}[]
\KeywordTok{\textless{}app{-}child}\OtherTok{ name=}\StringTok{"张三"}\KeywordTok{\textgreater{}\textless{}/app{-}child\textgreater{}}
\end{Highlighting}
\end{Shaded}

\begin{Shaded}
\begin{Highlighting}[]
\ImportTok{export} \KeywordTok{class}\NormalTok{ ChildComponent }\KeywordTok{implements}\NormalTok{ OnInit \{}
\NormalTok{  @}\FunctionTok{Input}\NormalTok{(}\StringTok{"name"}\NormalTok{) }\DataTypeTok{name}\OperatorTok{:}\NormalTok{ string }\OperatorTok{=} \StringTok{""}
  \FunctionTok{ngOnInit}\NormalTok{() \{}
    \BuiltInTok{console}\OperatorTok{.}\FunctionTok{log}\NormalTok{(}\KeywordTok{this}\OperatorTok{.}\AttributeTok{name}\NormalTok{) }\CommentTok{// "张三"}
\NormalTok{  \}}
\NormalTok{\}}
\end{Highlighting}
\end{Shaded}
\item
  ngAfterContentInit

  当内容投影初始渲染完成后调用。

\begin{Shaded}
\begin{Highlighting}[]
\KeywordTok{\textless{}app{-}child\textgreater{}}
	\KeywordTok{\textless{}div}\OtherTok{ \#box}\KeywordTok{\textgreater{}}\NormalTok{Hello Angular}\KeywordTok{\textless{}/div\textgreater{}}
\KeywordTok{\textless{}/app{-}child\textgreater{}}
\end{Highlighting}
\end{Shaded}

\begin{Shaded}
\begin{Highlighting}[]
\ImportTok{export} \KeywordTok{class}\NormalTok{ ChildComponent }\KeywordTok{implements}\NormalTok{ AfterContentInit \{}
\NormalTok{  @}\FunctionTok{ContentChild}\NormalTok{(}\StringTok{"box"}\NormalTok{) }\DataTypeTok{box}\OperatorTok{:}\NormalTok{ ElementRef}\OperatorTok{\textless{}}\BuiltInTok{HTMLDivElement}\OperatorTok{\textgreater{}} \OperatorTok{|} \KeywordTok{undefined}

  \FunctionTok{ngAfterContentInit}\NormalTok{() \{}
    \BuiltInTok{console}\OperatorTok{.}\FunctionTok{log}\NormalTok{(}\KeywordTok{this}\OperatorTok{.}\AttributeTok{box}\NormalTok{) }\CommentTok{// \textless{}div\textgreater{}Hello Angular\textless{}/div\textgreater{}}
\NormalTok{  \}}
\NormalTok{\}}
\end{Highlighting}
\end{Shaded}
\item
  ngAfterViewInit

  当组件视图渲染完成后调用。

\begin{Shaded}
\begin{Highlighting}[]
\CommentTok{\textless{}!{-}{-} app{-}child 组件模板 {-}{-}\textgreater{}}
\KeywordTok{\textless{}p}\OtherTok{ \#p}\KeywordTok{\textgreater{}}\NormalTok{app{-}child works}\KeywordTok{\textless{}/p\textgreater{}}
\end{Highlighting}
\end{Shaded}

\begin{Shaded}
\begin{Highlighting}[]
\ImportTok{export} \KeywordTok{class}\NormalTok{ ChildComponent }\KeywordTok{implements}\NormalTok{ AfterViewInit \{}
\NormalTok{  @}\FunctionTok{ViewChild}\NormalTok{(}\StringTok{"p"}\NormalTok{) }\DataTypeTok{p}\OperatorTok{:}\NormalTok{ ElementRef}\OperatorTok{\textless{}}\BuiltInTok{HTMLParagraphElement}\OperatorTok{\textgreater{}} \OperatorTok{|} \KeywordTok{undefined}
  \FunctionTok{ngAfterViewInit}\NormalTok{ () \{}
    \BuiltInTok{console}\OperatorTok{.}\FunctionTok{log}\NormalTok{(}\KeywordTok{this}\OperatorTok{.}\AttributeTok{p}\NormalTok{) }\CommentTok{// \textless{}p\textgreater{}app{-}child works\textless{}/p\textgreater{}}
\NormalTok{  \}}
\NormalTok{\}}
\end{Highlighting}
\end{Shaded}
\end{enumerate}

\hypertarget{82-ux66f4ux65b0ux9636ux6bb5}{%
\paragraph{8.2 更新阶段}\label{82-ux66f4ux65b0ux9636ux6bb5}}

\begin{enumerate}
\def\labelenumi{\arabic{enumi}.}
\item
  ngOnChanges

  \begin{enumerate}
  \def\labelenumii{\arabic{enumii}.}
  \item
    当输入属性值发生变化时执行,初始设置时也会执行一次,顺序优于
    ngOnInit
  \item
    不论多少输入属性同时变化,钩子函数只会执行一次,变化的值会同时存储在参数中
  \item
    参数类型为 SimpleChanges,子属性类型为 SimpleChange
  \item
    对于基本数据类型来说, 只要值发生变化就可以被检测到
  \item
    对于引用数据类型来说, 可以检测从一个对象变成另一个对象,
    但是检测不到同一个对象中属性值的变化,但是不影响组件模板更新数据。
  \end{enumerate}

  \textbf{基本数据类型值变化}

\begin{Shaded}
\begin{Highlighting}[]
\KeywordTok{\textless{}app{-}child}\OtherTok{ [name]=}\StringTok{"name"}\OtherTok{ [age]=}\StringTok{"age"}\KeywordTok{\textgreater{}\textless{}/app{-}child\textgreater{}}
\KeywordTok{\textless{}button}\OtherTok{ (click)=}\StringTok{"change()"}\KeywordTok{\textgreater{}}\NormalTok{change}\KeywordTok{\textless{}/button\textgreater{}}
\end{Highlighting}
\end{Shaded}

\begin{Shaded}
\begin{Highlighting}[]
\ImportTok{export} \KeywordTok{class}\NormalTok{ AppComponent \{}
  \DataTypeTok{name}\OperatorTok{:}\NormalTok{ string }\OperatorTok{=} \StringTok{"张三"}\OperatorTok{;}
	\DataTypeTok{age}\OperatorTok{:}\NormalTok{ number }\OperatorTok{=} \DecValTok{20}
  \FunctionTok{change}\NormalTok{() \{}
    \KeywordTok{this}\OperatorTok{.}\AttributeTok{name} \OperatorTok{=} \StringTok{"李四"}
    \KeywordTok{this}\OperatorTok{.}\AttributeTok{age} \OperatorTok{=} \DecValTok{30}
\NormalTok{  \}}
\NormalTok{\}}
\end{Highlighting}
\end{Shaded}

\begin{Shaded}
\begin{Highlighting}[]
\ImportTok{export} \KeywordTok{class}\NormalTok{ ChildComponent }\KeywordTok{implements}\NormalTok{ OnChanges \{}
\NormalTok{  @}\FunctionTok{Input}\NormalTok{(}\StringTok{"name"}\NormalTok{) }\DataTypeTok{name}\OperatorTok{:}\NormalTok{ string }\OperatorTok{=} \StringTok{""}
\NormalTok{	@}\FunctionTok{Input}\NormalTok{(}\StringTok{"age"}\NormalTok{) }\DataTypeTok{age}\OperatorTok{:}\NormalTok{ number }\OperatorTok{=} \DecValTok{0}

  \FunctionTok{ngOnChanges}\NormalTok{(}\DataTypeTok{changes}\OperatorTok{:}\NormalTok{ SimpleChanges) \{}
    \BuiltInTok{console}\OperatorTok{.}\FunctionTok{log}\NormalTok{(}\StringTok{"基本数据类型值变化可以被检测到"}\NormalTok{)}
\NormalTok{  \}}
\NormalTok{\}}
\end{Highlighting}
\end{Shaded}

  \textbf{引用数据类型变化}

\begin{Shaded}
\begin{Highlighting}[]
\KeywordTok{\textless{}app{-}child}\OtherTok{ [person]=}\StringTok{"person"}\KeywordTok{\textgreater{}\textless{}/app{-}child\textgreater{}}
\KeywordTok{\textless{}button}\OtherTok{ (click)=}\StringTok{"change()"}\KeywordTok{\textgreater{}}\NormalTok{change}\KeywordTok{\textless{}/button\textgreater{}}
\end{Highlighting}
\end{Shaded}

\begin{Shaded}
\begin{Highlighting}[]
\ImportTok{export} \KeywordTok{class}\NormalTok{ AppComponent \{}
\NormalTok{  person }\OperatorTok{=}\NormalTok{ \{ }\DataTypeTok{name}\OperatorTok{:} \StringTok{"张三"}\OperatorTok{,} \DataTypeTok{age}\OperatorTok{:} \DecValTok{20}\NormalTok{ \}}
  \FunctionTok{change}\NormalTok{() \{}
    \KeywordTok{this}\OperatorTok{.}\AttributeTok{person} \OperatorTok{=}\NormalTok{ \{ }\DataTypeTok{name}\OperatorTok{:} \StringTok{"李四"}\OperatorTok{,} \DataTypeTok{age}\OperatorTok{:} \DecValTok{30}\NormalTok{ \}}
\NormalTok{  \}}
\NormalTok{\}}
\end{Highlighting}
\end{Shaded}

\begin{Shaded}
\begin{Highlighting}[]
\ImportTok{export} \KeywordTok{class}\NormalTok{ ChildComponent }\KeywordTok{implements}\NormalTok{ OnChanges \{}
\NormalTok{  @}\FunctionTok{Input}\NormalTok{(}\StringTok{"person"}\NormalTok{) person }\OperatorTok{=}\NormalTok{ \{ }\DataTypeTok{name}\OperatorTok{:} \StringTok{""}\OperatorTok{,} \DataTypeTok{age}\OperatorTok{:} \DecValTok{0}\NormalTok{ \}}

  \FunctionTok{ngOnChanges}\NormalTok{(}\DataTypeTok{changes}\OperatorTok{:}\NormalTok{ SimpleChanges) \{}
    \BuiltInTok{console}\OperatorTok{.}\FunctionTok{log}\NormalTok{(}\StringTok{"对于引用数据类型, 只能检测到引用地址发生变化, 对象属性变化不能被检测到"}\NormalTok{)}
\NormalTok{  \}}
\NormalTok{\}}
\end{Highlighting}
\end{Shaded}
\item
  ngDoCheck:主要用于调试,只要输入属性发生变化,不论是基本数据类型还是引用数据类型还是引用数据类型中的属性变化,都会执行。
\item
  ngAfterContentChecked:内容投影更新完成后执行。
\item
  ngAfterViewChecked:组件视图更新完成后执行。
\end{enumerate}

\hypertarget{83-ux5378ux8f7dux9636ux6bb5}{%
\paragraph{8.3 卸载阶段}\label{83-ux5378ux8f7dux9636ux6bb5}}

\begin{enumerate}
\def\labelenumi{\arabic{enumi}.}
\item
  ngOnDestroy

  当组件被销毁之前调用, 用于清理操作。

\begin{Shaded}
\begin{Highlighting}[]
\ImportTok{export} \KeywordTok{class}\NormalTok{ HomeComponent }\KeywordTok{implements}\NormalTok{ OnDestroy \{}
  \FunctionTok{ngOnDestroy}\NormalTok{() \{}
    \BuiltInTok{console}\OperatorTok{.}\FunctionTok{log}\NormalTok{(}\StringTok{"组件被卸载"}\NormalTok{)}
\NormalTok{  \}}
\NormalTok{\}}
\end{Highlighting}
\end{Shaded}
\end{enumerate}

\hypertarget{9-ux4f9dux8d56ux6ce8ux5165}{%
\subsubsection{9. 依赖注入}\label{9-ux4f9dux8d56ux6ce8ux5165}}

\hypertarget{91-ux6982ux8ff0}{%
\paragraph{9.1 概述}\label{91-ux6982ux8ff0}}

依赖注入 ( Dependency Injection )
简称DI,是面向对象编程中的一种设计原则,用来减少代码之间的\textbf{耦合度}。

\begin{Shaded}
\begin{Highlighting}[]
\KeywordTok{class}\NormalTok{ MailService \{}
  \FunctionTok{constructor}\NormalTok{(APIKEY) \{\}}
\NormalTok{\}}

\KeywordTok{class}\NormalTok{ EmailSender \{}
  \DataTypeTok{mailService}\OperatorTok{:}\NormalTok{ MailService}
  \FunctionTok{constructor}\NormalTok{() \{}
    \KeywordTok{this}\OperatorTok{.}\AttributeTok{mailService} \OperatorTok{=} \KeywordTok{new} \FunctionTok{MailService}\NormalTok{(}\StringTok{"APIKEY1234567890"}\NormalTok{)}
\NormalTok{  \}}

  \FunctionTok{sendMail}\NormalTok{(mail) \{}
    \KeywordTok{this}\OperatorTok{.}\AttributeTok{mailService}\OperatorTok{.}\FunctionTok{sendMail}\NormalTok{(mail)}
\NormalTok{  \}}
\NormalTok{\}}

\KeywordTok{const}\NormalTok{ emailSender }\OperatorTok{=} \KeywordTok{new} \FunctionTok{EmailSender}\NormalTok{()}
\NormalTok{emailSender}\OperatorTok{.}\FunctionTok{sendMail}\NormalTok{(mail)}
\end{Highlighting}
\end{Shaded}

EmailSender 类运行时要使用 MailService 类,EmailSender 类依赖
MailService 类,MailService 类是 EmailSender 类的依赖项。

以上写法的耦合度太高,代码并不健壮。如果 MailService
类改变了参数的传递方式,在 EmailSender 类中的写法也要跟着改变。

\begin{Shaded}
\begin{Highlighting}[]
\KeywordTok{class}\NormalTok{ EmailSender \{}
  \DataTypeTok{mailService}\OperatorTok{:}\NormalTok{ MailService}
  \FunctionTok{constructor}\NormalTok{(}\DataTypeTok{mailService}\OperatorTok{:}\NormalTok{ MailService) \{}
    \KeywordTok{this}\OperatorTok{.}\AttributeTok{mailService} \OperatorTok{=}\NormalTok{ mailService}\OperatorTok{;}
\NormalTok{  \}}
\NormalTok{\}}
\KeywordTok{const}\NormalTok{ mailService }\OperatorTok{=} \KeywordTok{new} \FunctionTok{MailService}\NormalTok{(}\StringTok{"APIKEY1234567890"}\NormalTok{)}
\KeywordTok{const}\NormalTok{ emailSender }\OperatorTok{=} \KeywordTok{new} \FunctionTok{EmailSender}\NormalTok{(mailService)}
\end{Highlighting}
\end{Shaded}

在实例化 EmailSender 类时将它的依赖项通过 constructor
构造函数参数的形式注入到类的内部,这种写法就是依赖注入。

通过依赖注入降了代码之间的耦合度,增加了代码的可维护性。MailService
类中代码的更改再也不会影响 EmailSender 类。

\hypertarget{92-di-ux6846ux67b6}{%
\paragraph{9.2 DI 框架}\label{92-di-ux6846ux67b6}}

Angular 有自己的 DI
框架,它将实现依赖注入的过程隐藏了,对于开发者来说只需使用很简单的代码就可以使用复杂的依赖注入功能。

在 Angular 的 DI 框架中有四个核心概念:

\begin{enumerate}
\def\labelenumi{\arabic{enumi}.}
\item
  Dependency:组件要依赖的实例对象,服务实例对象
\item
  Token:获取服务实例对象的标识
\item
  Injector:注入器,负责创建维护服务类的实例对象并向组件中注入服务实例对象。
\item
  Provider:配置注入器的对象,指定创建服务实例对象的服务类和获取实例对象的标识。
\end{enumerate}

\hypertarget{921-ux6ce8ux5165ux5668-injectors}{%
\subparagraph{9.2.1 注入器
Injectors}\label{921-ux6ce8ux5165ux5668-injectors}}

注入器负责创建服务类实例对象,并将服务类实例对象注入到需要的组件中。

\begin{enumerate}
\def\labelenumi{\arabic{enumi}.}
\item
  创建注入器

\begin{Shaded}
\begin{Highlighting}[]
\ImportTok{import}\NormalTok{ \{ ReflectiveInjector \} }\ImportTok{from} \StringTok{"@angular/core"}
\CommentTok{// 服务类}
\KeywordTok{class}\NormalTok{ MailService \{\}}
\CommentTok{// 创建注入器并传入服务类}
\KeywordTok{const}\NormalTok{ injector }\OperatorTok{=}\NormalTok{ ReflectiveInjector}\OperatorTok{.}\FunctionTok{resolveAndCreate}\NormalTok{([MailService])}
\end{Highlighting}
\end{Shaded}
\item
  获取注入器中的服务类实例对象

\begin{Shaded}
\begin{Highlighting}[]
\KeywordTok{const}\NormalTok{ mailService }\OperatorTok{=}\NormalTok{ injector}\OperatorTok{.}\FunctionTok{get}\NormalTok{(MailService)}
\end{Highlighting}
\end{Shaded}
\item
  服务实例对象为单例模式,注入器在创建服务实例后会对其进行缓存

\begin{Shaded}
\begin{Highlighting}[]
\KeywordTok{const}\NormalTok{ mailService1 }\OperatorTok{=}\NormalTok{ injector}\OperatorTok{.}\FunctionTok{get}\NormalTok{(MailService)}
\KeywordTok{const}\NormalTok{ mailService2 }\OperatorTok{=}\NormalTok{ injector}\OperatorTok{.}\FunctionTok{get}\NormalTok{(MailService)}

\BuiltInTok{console}\OperatorTok{.}\FunctionTok{log}\NormalTok{(mailService1 }\OperatorTok{===}\NormalTok{ mailService2) }\CommentTok{// true}
\end{Highlighting}
\end{Shaded}
\item
  不同的注入器返回不同的服务实例对象

\begin{Shaded}
\begin{Highlighting}[]
\KeywordTok{const}\NormalTok{ injector }\OperatorTok{=}\NormalTok{ ReflectiveInjector}\OperatorTok{.}\FunctionTok{resolveAndCreate}\NormalTok{([MailService])}
\KeywordTok{const}\NormalTok{ childInjector }\OperatorTok{=}\NormalTok{ injector}\OperatorTok{.}\FunctionTok{resolveAndCreateChild}\NormalTok{([MailService])}

\KeywordTok{const}\NormalTok{ mailService1 }\OperatorTok{=}\NormalTok{ injector}\OperatorTok{.}\FunctionTok{get}\NormalTok{(MailService)}
\KeywordTok{const}\NormalTok{ mailService2 }\OperatorTok{=}\NormalTok{ childInjector}\OperatorTok{.}\FunctionTok{get}\NormalTok{(MailService)}

\BuiltInTok{console}\OperatorTok{.}\FunctionTok{log}\NormalTok{(mailService1 }\OperatorTok{===}\NormalTok{ mailService2)}
\end{Highlighting}
\end{Shaded}
\item
  服务实例的查找类似函数作用域链,当前级别可以找到就使用当前级别,当前级别找不到去父级中查找

\begin{Shaded}
\begin{Highlighting}[]
\KeywordTok{const}\NormalTok{ injector }\OperatorTok{=}\NormalTok{ ReflectiveInjector}\OperatorTok{.}\FunctionTok{resolveAndCreate}\NormalTok{([MailService])}
\KeywordTok{const}\NormalTok{ childInjector }\OperatorTok{=}\NormalTok{ injector}\OperatorTok{.}\FunctionTok{resolveAndCreateChild}\NormalTok{([])}

\KeywordTok{const}\NormalTok{ mailService1 }\OperatorTok{=}\NormalTok{ injector}\OperatorTok{.}\FunctionTok{get}\NormalTok{(MailService)}
\KeywordTok{const}\NormalTok{ mailService2 }\OperatorTok{=}\NormalTok{ childInjector}\OperatorTok{.}\FunctionTok{get}\NormalTok{(MailService)}

\BuiltInTok{console}\OperatorTok{.}\FunctionTok{log}\NormalTok{(mailService1 }\OperatorTok{===}\NormalTok{ mailService2)}
\end{Highlighting}
\end{Shaded}
\end{enumerate}

\hypertarget{922-ux63d0ux4f9bux8005-provider}{%
\subparagraph{9.2.2 提供者
Provider}\label{922-ux63d0ux4f9bux8005-provider}}

\begin{enumerate}
\def\labelenumi{\arabic{enumi}.}
\item
  配置注入器的对象,指定了创建实例对象的服务类和访问服务实例对象的标识。

\begin{Shaded}
\begin{Highlighting}[]
\KeywordTok{const}\NormalTok{ injector }\OperatorTok{=}\NormalTok{ ReflectiveInjector}\OperatorTok{.}\FunctionTok{resolveAndCreate}\NormalTok{([}
\NormalTok{  \{ }\DataTypeTok{provide}\OperatorTok{:}\NormalTok{ MailService}\OperatorTok{,} \DataTypeTok{useClass}\OperatorTok{:}\NormalTok{ MailService \}}
\NormalTok{])}
\end{Highlighting}
\end{Shaded}
\item
  访问依赖对象的标识也可以是字符串类型

\begin{Shaded}
\begin{Highlighting}[]
\KeywordTok{const}\NormalTok{ injector }\OperatorTok{=}\NormalTok{ ReflectiveInjector}\OperatorTok{.}\FunctionTok{resolveAndCreate}\NormalTok{([}
\NormalTok{  \{ }\DataTypeTok{provide}\OperatorTok{:} \StringTok{"mail"}\OperatorTok{,} \DataTypeTok{useClass}\OperatorTok{:}\NormalTok{ MailService \}}
\NormalTok{])}
\KeywordTok{const}\NormalTok{ mailService }\OperatorTok{=}\NormalTok{ injector}\OperatorTok{.}\FunctionTok{get}\NormalTok{(}\StringTok{"mail"}\NormalTok{)}
\end{Highlighting}
\end{Shaded}
\item
  useValue

\begin{Shaded}
\begin{Highlighting}[]
\KeywordTok{const}\NormalTok{ injector }\OperatorTok{=}\NormalTok{ ReflectiveInjector}\OperatorTok{.}\FunctionTok{resolveAndCreate}\NormalTok{([}
\NormalTok{  \{}
    \DataTypeTok{provide}\OperatorTok{:} \StringTok{"Config"}\OperatorTok{,}
    \DataTypeTok{useValue}\OperatorTok{:} \BuiltInTok{Object}\OperatorTok{.}\FunctionTok{freeze}\NormalTok{(\{}
      \DataTypeTok{APIKEY}\OperatorTok{:} \StringTok{"API1234567890"}\OperatorTok{,}
      \DataTypeTok{APISCRET}\OperatorTok{:} \StringTok{"500{-}400{-}300"}
\NormalTok{    \})}
\NormalTok{  \}}
\NormalTok{])}
\KeywordTok{const}\NormalTok{ Config }\OperatorTok{=}\NormalTok{ injector}\OperatorTok{.}\FunctionTok{get}\NormalTok{(}\StringTok{"Config"}\NormalTok{)}
\end{Highlighting}
\end{Shaded}
\end{enumerate}

将实例对象和外部的引用建立了松耦合关系,外部通过标识获取实例对象,只要标识保持不变,内部代码怎么变都不会影响到外部。

\hypertarget{10-ux670dux52a1-service}{%
\subsubsection{10. 服务 Service}\label{10-ux670dux52a1-service}}

\hypertarget{101-ux521bux5efaux670dux52a1}{%
\paragraph{10.1 创建服务}\label{101-ux521bux5efaux670dux52a1}}

\begin{Shaded}
\begin{Highlighting}[]
\ImportTok{import}\NormalTok{ \{ Injectable \} }\ImportTok{from} \StringTok{\textquotesingle{}@angular/core\textquotesingle{}}\OperatorTok{;}

\NormalTok{@}\FunctionTok{Injectable}\NormalTok{(\{}
  \DataTypeTok{providedIn}\OperatorTok{:} \StringTok{\textquotesingle{}root\textquotesingle{}}
\NormalTok{\})}
\ImportTok{export} \KeywordTok{class}\NormalTok{ TestService \{ \}}
\end{Highlighting}
\end{Shaded}

\begin{Shaded}
\begin{Highlighting}[]
\ImportTok{export} \KeywordTok{class}\NormalTok{ AppComponent \{}
 	\FunctionTok{constructor}\NormalTok{ (}\KeywordTok{private} \DataTypeTok{testService}\OperatorTok{:}\NormalTok{ TestService) \{\}}
\NormalTok{\}}
\end{Highlighting}
\end{Shaded}

\hypertarget{102-ux670dux52a1ux7684ux4f5cux7528ux57df}{%
\paragraph{10.2
服务的作用域}\label{102-ux670dux52a1ux7684ux4f5cux7528ux57df}}

使用服务可以轻松实现跨模块跨组件共享数据,这取决于服务的作用域。

\begin{enumerate}
\def\labelenumi{\arabic{enumi}.}
\item
  在根注入器中注册服务,所有模块使用同一个服务实例对象。

\begin{Shaded}
\begin{Highlighting}[]
\ImportTok{import}\NormalTok{ \{ Injectable \} }\ImportTok{from} \StringTok{\textquotesingle{}@angular/core\textquotesingle{}}\OperatorTok{;}

\NormalTok{@}\FunctionTok{Injectable}\NormalTok{(\{}
  \DataTypeTok{providedIn}\OperatorTok{:} \StringTok{\textquotesingle{}root\textquotesingle{}}
\NormalTok{\})}

\ImportTok{export} \KeywordTok{class}\NormalTok{ CarListService \{}
\NormalTok{\}}
\end{Highlighting}
\end{Shaded}
\item
  在模块级别注册服务,该模块中的所有组件使用同一个服务实例对象。

\begin{Shaded}
\begin{Highlighting}[]
\ImportTok{import}\NormalTok{ \{ Injectable \} }\ImportTok{from} \StringTok{\textquotesingle{}@angular/core\textquotesingle{}}\OperatorTok{;}
\ImportTok{import}\NormalTok{ \{ CarModule \} }\ImportTok{from} \StringTok{\textquotesingle{}./car.module\textquotesingle{}}\OperatorTok{;}

\NormalTok{@}\FunctionTok{Injectable}\NormalTok{(\{}
  \DataTypeTok{providedIn}\OperatorTok{:}\NormalTok{ CarModule}\OperatorTok{,}
\NormalTok{\})}

\ImportTok{export} \KeywordTok{class}\NormalTok{ CarListService \{}
\NormalTok{\}}
\end{Highlighting}
\end{Shaded}

\begin{Shaded}
\begin{Highlighting}[]
\ImportTok{import}\NormalTok{ \{ CarListService \} }\ImportTok{from} \StringTok{\textquotesingle{}./car{-}list.service\textquotesingle{}}\OperatorTok{;}

\NormalTok{@}\FunctionTok{NgModule}\NormalTok{(\{}
  \DataTypeTok{providers}\OperatorTok{:}\NormalTok{ [CarListService]}\OperatorTok{,}
\NormalTok{\})}
\ImportTok{export} \KeywordTok{class}\NormalTok{ CarModule \{}
\NormalTok{\}}
\end{Highlighting}
\end{Shaded}
\item
  在组件级别注册服务,该组件及其子组件使用同一个服务实例对象。

\begin{Shaded}
\begin{Highlighting}[]
\ImportTok{import}\NormalTok{ \{ Component \} }\ImportTok{from} \StringTok{\textquotesingle{}@angular/core\textquotesingle{}}\OperatorTok{;}
\ImportTok{import}\NormalTok{ \{ CarListService \} }\ImportTok{from} \StringTok{\textquotesingle{}../car{-}list.service.ts\textquotesingle{}}

\NormalTok{@}\FunctionTok{Component}\NormalTok{(\{}
  \DataTypeTok{selector}\OperatorTok{:}    \StringTok{\textquotesingle{}app{-}car{-}list\textquotesingle{}}\OperatorTok{,}
  \DataTypeTok{templateUrl}\OperatorTok{:} \StringTok{\textquotesingle{}./car{-}list.component.html\textquotesingle{}}\OperatorTok{,}
  \DataTypeTok{providers}\OperatorTok{:}\NormalTok{  [ CarListService ]}
\NormalTok{\})}
\end{Highlighting}
\end{Shaded}
\end{enumerate}

\hypertarget{11-ux8868ux5355}{%
\subsubsection{11. 表单}\label{11-ux8868ux5355}}

在 Angular 中,表单有两种类型,分别为模板驱动和模型驱动。

\hypertarget{111-ux6a21ux677fux9a71ux52a8}{%
\paragraph{11.1 模板驱动}\label{111-ux6a21ux677fux9a71ux52a8}}

\hypertarget{1111-ux6982ux8ff0}{%
\subparagraph{11.1.1 概述}\label{1111-ux6982ux8ff0}}

表单的控制逻辑写在组件模板中,适合简单的表单类型。

\hypertarget{1112-ux5febux901fux4e0aux624b}{%
\subparagraph{11.1.2 快速上手}\label{1112-ux5febux901fux4e0aux624b}}

\begin{enumerate}
\def\labelenumi{\arabic{enumi}.}
\item
  引入依赖模块 FormsModule

\begin{Shaded}
\begin{Highlighting}[]
\ImportTok{import}\NormalTok{ \{ FormsModule \} }\ImportTok{from} \StringTok{"@angular/forms"}

\NormalTok{@}\FunctionTok{NgModule}\NormalTok{(\{}
  \DataTypeTok{imports}\OperatorTok{:}\NormalTok{ [FormsModule]}\OperatorTok{,}
\NormalTok{\})}
\ImportTok{export} \KeywordTok{class}\NormalTok{ AppModule \{\}}
\end{Highlighting}
\end{Shaded}
\item
  将 DOM 表单转换为 ngForm

\begin{Shaded}
\begin{Highlighting}[]
\KeywordTok{\textless{}form}\OtherTok{ \#f=}\StringTok{"ngForm"}\OtherTok{ (submit)=}\StringTok{"onSubmit(f)"}\KeywordTok{\textgreater{}\textless{}/form\textgreater{}}
\end{Highlighting}
\end{Shaded}
\item
  声明表单字段为 ngModel

\begin{Shaded}
\begin{Highlighting}[]
\KeywordTok{\textless{}form}\OtherTok{ \#f=}\StringTok{"ngForm"}\OtherTok{ (submit)=}\StringTok{"onSubmit(f)"}\KeywordTok{\textgreater{}}
  \KeywordTok{\textless{}input}\OtherTok{ type=}\StringTok{"text"}\OtherTok{ name=}\StringTok{"username"}\OtherTok{ ngModel} \KeywordTok{/\textgreater{}}
  \KeywordTok{\textless{}button\textgreater{}}\NormalTok{提交\textless{}/button\textgreater{}}
\KeywordTok{\textless{}/form\textgreater{}}
\end{Highlighting}
\end{Shaded}
\item
  获取表单字段值

\begin{Shaded}
\begin{Highlighting}[]
\ImportTok{import}\NormalTok{ \{ NgForm \} }\ImportTok{from} \StringTok{"@angular/forms"}

\ImportTok{export} \KeywordTok{class}\NormalTok{ AppComponent \{}
  \FunctionTok{onSubmit}\NormalTok{(}\DataTypeTok{form}\OperatorTok{:}\NormalTok{ NgForm) \{}
    \BuiltInTok{console}\OperatorTok{.}\FunctionTok{log}\NormalTok{(form}\OperatorTok{.}\AttributeTok{value}\NormalTok{)}
\NormalTok{  \}}
\NormalTok{\}}
\end{Highlighting}
\end{Shaded}
\item
  表单分组

\begin{Shaded}
\begin{Highlighting}[]
\KeywordTok{\textless{}form}\OtherTok{ \#f=}\StringTok{"ngForm"}\OtherTok{ (submit)=}\StringTok{"onSubmit(f)"}\KeywordTok{\textgreater{}}
  \KeywordTok{\textless{}div}\OtherTok{ ngModelGroup=}\StringTok{"user"}\KeywordTok{\textgreater{}}
    \KeywordTok{\textless{}input}\OtherTok{ type=}\StringTok{"text"}\OtherTok{ name=}\StringTok{"username"}\OtherTok{ ngModel} \KeywordTok{/\textgreater{}}
  \KeywordTok{\textless{}/div\textgreater{}}
  \KeywordTok{\textless{}div}\OtherTok{ ngModelGroup=}\StringTok{"contact"}\KeywordTok{\textgreater{}}
    \KeywordTok{\textless{}input}\OtherTok{ type=}\StringTok{"text"}\OtherTok{ name=}\StringTok{"phone"}\OtherTok{ ngModel} \KeywordTok{/\textgreater{}}
  \KeywordTok{\textless{}/div\textgreater{}}
  \KeywordTok{\textless{}button\textgreater{}}\NormalTok{提交\textless{}/button\textgreater{}}
\KeywordTok{\textless{}/form\textgreater{}}
\end{Highlighting}
\end{Shaded}
\end{enumerate}

\hypertarget{1113-ux8868ux5355ux9a8cux8bc1}{%
\subparagraph{11.1.3 表单验证}\label{1113-ux8868ux5355ux9a8cux8bc1}}

\begin{itemize}
\item
  required 必填字段
\item
  minlength 字段最小长度
\item
  maxlength 字段最大长度
\item
  pattern 验证正则 例如:pattern="\textbackslash d" 匹配一个数值
\end{itemize}

\begin{Shaded}
\begin{Highlighting}[]
\KeywordTok{\textless{}form}\OtherTok{ \#f=}\StringTok{"ngForm"}\OtherTok{ (submit)=}\StringTok{"onSubmit(f)"}\KeywordTok{\textgreater{}}
  \KeywordTok{\textless{}input}\OtherTok{ type=}\StringTok{"text"}\OtherTok{ name=}\StringTok{"username"}\OtherTok{ ngModel required pattern=}\StringTok{"\textbackslash{}d"} \KeywordTok{/\textgreater{}}
  \KeywordTok{\textless{}button\textgreater{}}\NormalTok{提交\textless{}/button\textgreater{}}
\KeywordTok{\textless{}/form\textgreater{}}
\end{Highlighting}
\end{Shaded}

\begin{Shaded}
\begin{Highlighting}[]
\ImportTok{export} \KeywordTok{class}\NormalTok{ AppComponent \{}
  \FunctionTok{onSubmit}\NormalTok{(}\DataTypeTok{form}\OperatorTok{:}\NormalTok{ NgForm) \{}
    \CommentTok{// 查看表单整体是否验证通过}
    \BuiltInTok{console}\OperatorTok{.}\FunctionTok{log}\NormalTok{(form}\OperatorTok{.}\AttributeTok{valid}\NormalTok{)}
\NormalTok{  \}}
\NormalTok{\}}
\end{Highlighting}
\end{Shaded}

\begin{Shaded}
\begin{Highlighting}[]
\CommentTok{\textless{}!{-}{-} 表单整体未通过验证时禁用提交表单 {-}{-}\textgreater{}}
\KeywordTok{\textless{}button}\OtherTok{ type=}\StringTok{"submit"}\OtherTok{ [disabled]=}\StringTok{"f.invalid"}\KeywordTok{\textgreater{}}\NormalTok{提交\textless{}/button\textgreater{}}
\end{Highlighting}
\end{Shaded}

在组件模板中显示表单项未通过时的错误信息。

\begin{Shaded}
\begin{Highlighting}[]
\KeywordTok{\textless{}form}\OtherTok{ \#f=}\StringTok{"ngForm"}\OtherTok{ (submit)=}\StringTok{"onSubmit(f)"}\KeywordTok{\textgreater{}}
  \KeywordTok{\textless{}input}\OtherTok{ \#username=}\StringTok{"ngModel"} \KeywordTok{/\textgreater{}}
  \KeywordTok{\textless{}div}\OtherTok{ *ngIf=}\StringTok{"username.touched }\ErrorTok{\&\&}\StringTok{ !username.valid }\ErrorTok{\&\&}\StringTok{ username.errors"}\KeywordTok{\textgreater{}}
    \KeywordTok{\textless{}div}\OtherTok{ *ngIf=}\StringTok{"username.errors.required"}\KeywordTok{\textgreater{}}\NormalTok{请填写用户名}\KeywordTok{\textless{}/div\textgreater{}}
    \KeywordTok{\textless{}div}\OtherTok{ *ngIf=}\StringTok{"username.errors.pattern"}\KeywordTok{\textgreater{}}\NormalTok{不符合正则规则}\KeywordTok{\textless{}/div\textgreater{}}
  \KeywordTok{\textless{}/div\textgreater{}}
\KeywordTok{\textless{}/form\textgreater{}}
\end{Highlighting}
\end{Shaded}

指定表单项未通过验证时的样式。

\begin{Shaded}
\begin{Highlighting}[]
\NormalTok{input}\FunctionTok{.ng{-}touched.ng{-}invalid}\NormalTok{ \{}
  \KeywordTok{border}\NormalTok{: }\DecValTok{2}\DataTypeTok{px} \DecValTok{solid} \ConstantTok{red}\OperatorTok{;}
\NormalTok{\}}
\end{Highlighting}
\end{Shaded}

\hypertarget{112-ux6a21ux578bux9a71ux52a8}{%
\paragraph{11.2 模型驱动}\label{112-ux6a21ux578bux9a71ux52a8}}

\hypertarget{1121-ux6982ux8ff0}{%
\subparagraph{11.2.1 概述}\label{1121-ux6982ux8ff0}}

表单的控制逻辑写在组件类中,对验证逻辑拥有更多的控制权,适合复杂的表单的类型。

在模型驱动表单中,表单字段需要是 FormControl
类的实例,实例对象可以验证表单字段中的值,值是否被修改过等等

\begin{figure}
\centering
\includegraphics{C:/Users/ZSH/Desktop/ng/ppt/images/6.jpg}
\caption{}
\end{figure}

一组表单字段构成整个表单,整个表单需要是 FormGroup
类的实例,它可以对表单进行整体验证。

\begin{figure}
\centering
\includegraphics{C:/Users/ZSH/Desktop/ng/ppt/images/7.jpg}
\caption{}
\end{figure}

\begin{enumerate}
\def\labelenumi{\arabic{enumi}.}
\item
  FormControl:表单组中的一个表单项
\item
  FormGroup:表单组,表单至少是一个 FormGroup
\item
  FormArray:用于复杂表单,可以动态添加表单项或表单组,在表单验证时,FormArray
  中有一项没通过,整体没通过。
\end{enumerate}

\hypertarget{1122-ux5febux901fux4e0aux624b}{%
\subparagraph{11.2.2 快速上手}\label{1122-ux5febux901fux4e0aux624b}}

\begin{enumerate}
\def\labelenumi{\arabic{enumi}.}
\item
  引入 ReactiveFormsModule

\begin{Shaded}
\begin{Highlighting}[]
\ImportTok{import}\NormalTok{ \{ ReactiveFormsModule \} }\ImportTok{from} \StringTok{"@angular/forms"}

\NormalTok{@}\FunctionTok{NgModule}\NormalTok{(\{}
  \DataTypeTok{imports}\OperatorTok{:}\NormalTok{ [ReactiveFormsModule]}
\NormalTok{\})}
\ImportTok{export} \KeywordTok{class}\NormalTok{ AppModule \{\}}
\end{Highlighting}
\end{Shaded}
\item
  在组件类中创建 FormGroup 表单控制对象

\begin{Shaded}
\begin{Highlighting}[]
\ImportTok{import}\NormalTok{ \{ FormControl}\OperatorTok{,}\NormalTok{ FormGroup \} }\ImportTok{from} \StringTok{"@angular/forms"}

\ImportTok{export} \KeywordTok{class}\NormalTok{ AppComponent \{}
  \DataTypeTok{contactForm}\OperatorTok{:}\NormalTok{ FormGroup }\OperatorTok{=} \KeywordTok{new} \FunctionTok{FormGroup}\NormalTok{(\{}
    \DataTypeTok{name}\OperatorTok{:} \KeywordTok{new} \FunctionTok{FormControl}\NormalTok{()}\OperatorTok{,}
    \DataTypeTok{phone}\OperatorTok{:} \KeywordTok{new} \FunctionTok{FormControl}\NormalTok{()}
\NormalTok{  \})}
\NormalTok{\}}
\end{Highlighting}
\end{Shaded}
\item
  关联组件模板中的表单

\begin{Shaded}
\begin{Highlighting}[]
\KeywordTok{\textless{}form}\OtherTok{ [formGroup]=}\StringTok{"contactForm"}\OtherTok{ (submit)=}\StringTok{"onSubmit()"}\KeywordTok{\textgreater{}}
  \KeywordTok{\textless{}input}\OtherTok{ type=}\StringTok{"text"}\OtherTok{ formControlName=}\StringTok{"name"} \KeywordTok{/\textgreater{}}
  \KeywordTok{\textless{}input}\OtherTok{ type=}\StringTok{"text"}\OtherTok{ formControlName=}\StringTok{"phone"} \KeywordTok{/\textgreater{}}
  \KeywordTok{\textless{}button\textgreater{}}\NormalTok{提交\textless{}/button\textgreater{}}
\KeywordTok{\textless{}/form\textgreater{}}
\end{Highlighting}
\end{Shaded}
\item
  获取表单值

\begin{Shaded}
\begin{Highlighting}[]
\ImportTok{export} \KeywordTok{class}\NormalTok{ AppComponent \{}
  \FunctionTok{onSubmit}\NormalTok{() \{}
    \BuiltInTok{console}\OperatorTok{.}\FunctionTok{log}\NormalTok{(}\KeywordTok{this}\OperatorTok{.}\AttributeTok{contactForm}\OperatorTok{.}\AttributeTok{value}\NormalTok{)}
\NormalTok{  \}}
\NormalTok{\}}
\end{Highlighting}
\end{Shaded}
\item
  设置表单默认值

\begin{Shaded}
\begin{Highlighting}[]
\NormalTok{contactForm}\OperatorTok{:}\NormalTok{ FormGroup }\OperatorTok{=} \KeywordTok{new} \FunctionTok{FormGroup}\NormalTok{(\{}
  \DataTypeTok{name}\OperatorTok{:} \KeywordTok{new} \FunctionTok{FormControl}\NormalTok{(}\StringTok{"默认值"}\NormalTok{)}\OperatorTok{,}
  \DataTypeTok{phone}\OperatorTok{:} \KeywordTok{new} \FunctionTok{FormControl}\NormalTok{(}\DecValTok{15888888888}\NormalTok{)}
\NormalTok{\})}
\end{Highlighting}
\end{Shaded}
\item
  表单分组

\begin{Shaded}
\begin{Highlighting}[]
\NormalTok{contactForm}\OperatorTok{:}\NormalTok{ FormGroup }\OperatorTok{=} \KeywordTok{new} \FunctionTok{FormGroup}\NormalTok{(\{}
  \DataTypeTok{fullName}\OperatorTok{:} \KeywordTok{new} \FunctionTok{FormGroup}\NormalTok{(\{}
    \DataTypeTok{firstName}\OperatorTok{:} \KeywordTok{new} \FunctionTok{FormControl}\NormalTok{()}\OperatorTok{,}
    \DataTypeTok{lastName}\OperatorTok{:} \KeywordTok{new} \FunctionTok{FormControl}\NormalTok{()}
\NormalTok{  \})}\OperatorTok{,}
  \DataTypeTok{phone}\OperatorTok{:} \KeywordTok{new} \FunctionTok{FormControl}\NormalTok{()}
\NormalTok{\})}
\end{Highlighting}
\end{Shaded}

\begin{Shaded}
\begin{Highlighting}[]
\KeywordTok{\textless{}form}\OtherTok{ [formGroup]=}\StringTok{"contactForm"}\OtherTok{ (submit)=}\StringTok{"onSubmit()"}\KeywordTok{\textgreater{}}
  \KeywordTok{\textless{}div}\OtherTok{ formGroupName=}\StringTok{"fullName"}\KeywordTok{\textgreater{}}
    \KeywordTok{\textless{}input}\OtherTok{ type=}\StringTok{"text"}\OtherTok{ formControlName=}\StringTok{"firstName"} \KeywordTok{/\textgreater{}}
    \KeywordTok{\textless{}input}\OtherTok{ type=}\StringTok{"text"}\OtherTok{ formControlName=}\StringTok{"lastName"} \KeywordTok{/\textgreater{}}
  \KeywordTok{\textless{}/div\textgreater{}}
  \KeywordTok{\textless{}input}\OtherTok{ type=}\StringTok{"text"}\OtherTok{ formControlName=}\StringTok{"phone"} \KeywordTok{/\textgreater{}}
  \KeywordTok{\textless{}button\textgreater{}}\NormalTok{提交\textless{}/button\textgreater{}}
\KeywordTok{\textless{}/form\textgreater{}}
\end{Highlighting}
\end{Shaded}

\begin{Shaded}
\begin{Highlighting}[]
\FunctionTok{onSubmit}\NormalTok{() \{}
  \BuiltInTok{console}\OperatorTok{.}\FunctionTok{log}\NormalTok{(}\KeywordTok{this}\OperatorTok{.}\AttributeTok{contactForm}\OperatorTok{.}\AttributeTok{value}\OperatorTok{.}\AttributeTok{name}\OperatorTok{.}\AttributeTok{username}\NormalTok{)}
  \BuiltInTok{console}\OperatorTok{.}\FunctionTok{log}\NormalTok{(}\KeywordTok{this}\OperatorTok{.}\AttributeTok{contactForm}\OperatorTok{.}\FunctionTok{get}\NormalTok{([}\StringTok{"name"}\OperatorTok{,} \StringTok{"username"}\NormalTok{])}\OperatorTok{?.}\AttributeTok{value}\NormalTok{)}
\NormalTok{\}}
\end{Highlighting}
\end{Shaded}
\end{enumerate}

\hypertarget{1123-formarray}{%
\subparagraph{11.2.3 FormArray}\label{1123-formarray}}

需求:在页面中默认显示一组联系方式,通过点击按钮可以添加更多联系方式组。

\begin{Shaded}
\begin{Highlighting}[]
\ImportTok{import}\NormalTok{ \{ Component}\OperatorTok{,}\NormalTok{ OnInit \} }\ImportTok{from} \StringTok{"@angular/core"}
\ImportTok{import}\NormalTok{ \{ FormArray}\OperatorTok{,}\NormalTok{ FormControl}\OperatorTok{,}\NormalTok{ FormGroup \} }\ImportTok{from} \StringTok{"@angular/forms"}
\NormalTok{@}\FunctionTok{Component}\NormalTok{(\{}
  \DataTypeTok{selector}\OperatorTok{:} \StringTok{"app{-}root"}\OperatorTok{,}
  \DataTypeTok{templateUrl}\OperatorTok{:} \StringTok{"./app.component.html"}\OperatorTok{,}
  \DataTypeTok{styles}\OperatorTok{:}\NormalTok{ []}
\NormalTok{\})}
\ImportTok{export} \KeywordTok{class}\NormalTok{ AppComponent }\KeywordTok{implements}\NormalTok{ OnInit \{}
  \CommentTok{// 表单}
  \DataTypeTok{contactForm}\OperatorTok{:}\NormalTok{ FormGroup }\OperatorTok{=} \KeywordTok{new} \FunctionTok{FormGroup}\NormalTok{(\{}
    \DataTypeTok{contacts}\OperatorTok{:} \KeywordTok{new} \FunctionTok{FormArray}\NormalTok{([])}
\NormalTok{  \})}

\NormalTok{  get }\FunctionTok{contacts}\NormalTok{() \{}
    \ControlFlowTok{return} \KeywordTok{this}\OperatorTok{.}\AttributeTok{contactForm}\OperatorTok{.}\FunctionTok{get}\NormalTok{(}\StringTok{"contacts"}\NormalTok{) }\ImportTok{as}\NormalTok{ FormArray}
\NormalTok{  \}}

  \CommentTok{// 添加联系方式}
  \FunctionTok{addContact}\NormalTok{() \{}
    \CommentTok{// 联系方式}
    \KeywordTok{const} \DataTypeTok{myContact}\OperatorTok{:}\NormalTok{ FormGroup }\OperatorTok{=} \KeywordTok{new} \FunctionTok{FormGroup}\NormalTok{(\{}
      \DataTypeTok{name}\OperatorTok{:} \KeywordTok{new} \FunctionTok{FormControl}\NormalTok{()}\OperatorTok{,}
      \DataTypeTok{address}\OperatorTok{:} \KeywordTok{new} \FunctionTok{FormControl}\NormalTok{()}\OperatorTok{,}
      \DataTypeTok{phone}\OperatorTok{:} \KeywordTok{new} \FunctionTok{FormControl}\NormalTok{()}
\NormalTok{    \})}
    \CommentTok{// 向联系方式数组中添加联系方式}
    \KeywordTok{this}\OperatorTok{.}\AttributeTok{contacts}\OperatorTok{.}\FunctionTok{push}\NormalTok{(myContact)}
\NormalTok{  \}}

  \CommentTok{// 删除联系方式}
  \FunctionTok{removeContact}\NormalTok{(}\DataTypeTok{i}\OperatorTok{:}\NormalTok{ number) \{}
    \KeywordTok{this}\OperatorTok{.}\AttributeTok{contacts}\OperatorTok{.}\FunctionTok{removeAt}\NormalTok{(i)}
\NormalTok{  \}}

  \FunctionTok{ngOnInit}\NormalTok{() \{}
    \CommentTok{// 添加默认的联系方式}
    \KeywordTok{this}\OperatorTok{.}\FunctionTok{addContact}\NormalTok{()}
\NormalTok{  \}}

  \FunctionTok{onSubmit}\NormalTok{() \{}
    \BuiltInTok{console}\OperatorTok{.}\FunctionTok{log}\NormalTok{(}\KeywordTok{this}\OperatorTok{.}\AttributeTok{contactForm}\OperatorTok{.}\AttributeTok{value}\NormalTok{)}
\NormalTok{  \}}
\NormalTok{\}}
\end{Highlighting}
\end{Shaded}

\begin{Shaded}
\begin{Highlighting}[]
\KeywordTok{\textless{}form}\OtherTok{ [formGroup]=}\StringTok{"contactForm"}\OtherTok{ (submit)=}\StringTok{"onSubmit()"}\KeywordTok{\textgreater{}}
  \KeywordTok{\textless{}div}\OtherTok{ formArrayName=}\StringTok{"contacts"}\KeywordTok{\textgreater{}}
    \KeywordTok{\textless{}div}
\OtherTok{      *ngFor=}\StringTok{"let contact of contacts.controls; let i = index"}
\OtherTok{      [formGroupName]=}\StringTok{"i"}
    \KeywordTok{\textgreater{}}
      \KeywordTok{\textless{}input}\OtherTok{ type=}\StringTok{"text"}\OtherTok{ formControlName=}\StringTok{"name"} \KeywordTok{/\textgreater{}}
      \KeywordTok{\textless{}input}\OtherTok{ type=}\StringTok{"text"}\OtherTok{ formControlName=}\StringTok{"address"} \KeywordTok{/\textgreater{}}
      \KeywordTok{\textless{}input}\OtherTok{ type=}\StringTok{"text"}\OtherTok{ formControlName=}\StringTok{"phone"} \KeywordTok{/\textgreater{}}
      \KeywordTok{\textless{}button}\OtherTok{ (click)=}\StringTok{"removeContact(i)"}\KeywordTok{\textgreater{}}\NormalTok{删除联系方式\textless{}/button\textgreater{}}
    \KeywordTok{\textless{}/div\textgreater{}}
  \KeywordTok{\textless{}/div\textgreater{}}
  \KeywordTok{\textless{}button}\OtherTok{ (click)=}\StringTok{"addContact()"}\KeywordTok{\textgreater{}}\NormalTok{添加联系方式\textless{}/button\textgreater{}}
  \KeywordTok{\textless{}button\textgreater{}}\NormalTok{提交\textless{}/button\textgreater{}}
\KeywordTok{\textless{}/form\textgreater{}}
\end{Highlighting}
\end{Shaded}

\hypertarget{1124-ux5185ux7f6eux8868ux5355ux9a8cux8bc1ux5668}{%
\subparagraph{11.2.4
内置表单验证器}\label{1124-ux5185ux7f6eux8868ux5355ux9a8cux8bc1ux5668}}

\begin{enumerate}
\def\labelenumi{\arabic{enumi}.}
\item
  使用内置验证器提供的验证规则验证表单字段

\begin{Shaded}
\begin{Highlighting}[]
\ImportTok{import}\NormalTok{ \{ FormControl}\OperatorTok{,}\NormalTok{ FormGroup}\OperatorTok{,}\NormalTok{ Validators \} }\ImportTok{from} \StringTok{"@angular/forms"}

\NormalTok{contactForm}\OperatorTok{:}\NormalTok{ FormGroup }\OperatorTok{=} \KeywordTok{new} \FunctionTok{FormGroup}\NormalTok{(\{}
  \DataTypeTok{name}\OperatorTok{:} \KeywordTok{new} \FunctionTok{FormControl}\NormalTok{(}\StringTok{"默认值"}\OperatorTok{,}\NormalTok{ [}
\NormalTok{    Validators}\OperatorTok{.}\AttributeTok{required}\OperatorTok{,}
\NormalTok{    Validators}\OperatorTok{.}\FunctionTok{minLength}\NormalTok{(}\DecValTok{2}\NormalTok{)}
\NormalTok{  ])}
\NormalTok{\})}
\end{Highlighting}
\end{Shaded}
\item
  获取整体表单是否验证通过

\begin{Shaded}
\begin{Highlighting}[]
\FunctionTok{onSubmit}\NormalTok{() \{}
  \BuiltInTok{console}\OperatorTok{.}\FunctionTok{log}\NormalTok{(}\KeywordTok{this}\OperatorTok{.}\AttributeTok{contactForm}\OperatorTok{.}\AttributeTok{valid}\NormalTok{)}
\NormalTok{\}}
\end{Highlighting}
\end{Shaded}

\begin{Shaded}
\begin{Highlighting}[]
\CommentTok{\textless{}!{-}{-} 表单整体未验证通过时禁用表单按钮 {-}{-}\textgreater{}}
\KeywordTok{\textless{}button}\OtherTok{ [disabled]=}\StringTok{"contactForm.invalid"}\KeywordTok{\textgreater{}}\NormalTok{提交\textless{}/button\textgreater{}}
\end{Highlighting}
\end{Shaded}
\item
  在组件模板中显示为验证通过时的错误信息

\begin{Shaded}
\begin{Highlighting}[]
\NormalTok{get }\FunctionTok{name}\NormalTok{() \{}
  \ControlFlowTok{return} \KeywordTok{this}\OperatorTok{.}\AttributeTok{contactForm}\OperatorTok{.}\FunctionTok{get}\NormalTok{(}\StringTok{"name"}\NormalTok{)}\OperatorTok{!}
\NormalTok{\}}
\end{Highlighting}
\end{Shaded}

\begin{Shaded}
\begin{Highlighting}[]
\KeywordTok{\textless{}form}\OtherTok{ [formGroup]=}\StringTok{"contactForm"}\OtherTok{ (submit)=}\StringTok{"onSubmit()"}\KeywordTok{\textgreater{}}
  \KeywordTok{\textless{}input}\OtherTok{ type=}\StringTok{"text"}\OtherTok{ formControlName=}\StringTok{"name"} \KeywordTok{/\textgreater{}}
  \KeywordTok{\textless{}div}\OtherTok{ *ngIf=}\StringTok{"name.touched }\ErrorTok{\&\&}\StringTok{ name.invalid }\ErrorTok{\&\&}\StringTok{ name.errors"}\KeywordTok{\textgreater{}}
    \KeywordTok{\textless{}div}\OtherTok{ *ngIf=}\StringTok{"name.errors.required"}\KeywordTok{\textgreater{}}\NormalTok{请填写姓名}\KeywordTok{\textless{}/div\textgreater{}}
    \KeywordTok{\textless{}div}\OtherTok{ *ngIf=}\StringTok{"name.errors.maxlength"}\KeywordTok{\textgreater{}}
\NormalTok{      姓名长度不能大于}
\NormalTok{      \{\{ name.errors.maxlength.requiredLength \}\} 实际填写长度为}
\NormalTok{      \{\{ name.errors.maxlength.actualLength \}\}}
    \KeywordTok{\textless{}/div\textgreater{}}
  \KeywordTok{\textless{}/div\textgreater{}}
\KeywordTok{\textless{}/form\textgreater{}}
\end{Highlighting}
\end{Shaded}
\end{enumerate}

\hypertarget{1125-ux81eaux5b9aux4e49ux540cux6b65ux8868ux5355ux9a8cux8bc1ux5668}{%
\subparagraph{11.2.5
自定义同步表单验证器}\label{1125-ux81eaux5b9aux4e49ux540cux6b65ux8868ux5355ux9a8cux8bc1ux5668}}

\begin{enumerate}
\def\labelenumi{\arabic{enumi}.}
\item
  自定义验证器的类型是 TypeScript 类
\item
  类中包含具体的验证方法,验证方法必须为静态方法
\item
  验证方法有一个参数 control,类型为 AbstractControl。其实就是
  FormControl 类的实例对象的类型
\item
  如果验证成功,返回 null
\item
  如果验证失败,返回对象,对象中的属性即为验证标识,值为
  true,标识该项验证失败
\item
  验证方法的返回值为 ValidationErrors \textbar{} null
\end{enumerate}

\begin{Shaded}
\begin{Highlighting}[]
\ImportTok{import}\NormalTok{ \{ AbstractControl}\OperatorTok{,}\NormalTok{ ValidationErrors \} }\ImportTok{from} \StringTok{"@angular/forms"}

\ImportTok{export} \KeywordTok{class}\NormalTok{ NameValidators \{}
  \CommentTok{// 字段值中不能包含空格}
  \KeywordTok{static} \FunctionTok{cannotContainSpace}\NormalTok{(}\DataTypeTok{control}\OperatorTok{:}\NormalTok{ AbstractControl)}\OperatorTok{:}\NormalTok{ ValidationErrors }\OperatorTok{|} \KeywordTok{null}\NormalTok{ \{}
    \CommentTok{// 验证未通过}
    \ControlFlowTok{if}\NormalTok{ (}\SpecialStringTok{/}\SpecialCharTok{\textbackslash{}s}\SpecialStringTok{/}\OperatorTok{.}\FunctionTok{test}\NormalTok{(control}\OperatorTok{.}\AttributeTok{value}\NormalTok{)) }\ControlFlowTok{return}\NormalTok{ \{ }\DataTypeTok{cannotContainSpace}\OperatorTok{:} \KeywordTok{true}\NormalTok{ \}}
    \CommentTok{// 验证通过}
    \ControlFlowTok{return} \KeywordTok{null}
\NormalTok{  \}}
\NormalTok{\}}
\end{Highlighting}
\end{Shaded}

\begin{Shaded}
\begin{Highlighting}[]
\ImportTok{import}\NormalTok{ \{ NameValidators \} }\ImportTok{from} \StringTok{"./Name.validators"}

\NormalTok{contactForm}\OperatorTok{:}\NormalTok{ FormGroup }\OperatorTok{=} \KeywordTok{new} \FunctionTok{FormGroup}\NormalTok{(\{}
  \DataTypeTok{name}\OperatorTok{:} \KeywordTok{new} \FunctionTok{FormControl}\NormalTok{(}\StringTok{""}\OperatorTok{,}\NormalTok{ [}
\NormalTok{    Validators}\OperatorTok{.}\AttributeTok{required}\OperatorTok{,}
\NormalTok{    NameValidators}\OperatorTok{.}\AttributeTok{cannotContainSpace}
\NormalTok{  ])}
\NormalTok{\})}
\end{Highlighting}
\end{Shaded}

\begin{Shaded}
\begin{Highlighting}[]
\KeywordTok{\textless{}div}\OtherTok{ *ngIf=}\StringTok{"name.touched }\ErrorTok{\&\&}\StringTok{ name.invalid }\ErrorTok{\&\&}\StringTok{ name.errors"}\KeywordTok{\textgreater{}}
	\KeywordTok{\textless{}div}\OtherTok{ *ngIf=}\StringTok{"name.errors.cannotContainSpace"}\KeywordTok{\textgreater{}}\NormalTok{姓名中不能包含空格}\KeywordTok{\textless{}/div\textgreater{}}
\KeywordTok{\textless{}/div\textgreater{}}
\end{Highlighting}
\end{Shaded}

\hypertarget{1126-ux81eaux5b9aux4e49ux5f02ux6b65ux8868ux5355ux9a8cux8bc1ux5668}{%
\subparagraph{11.2.6
自定义异步表单验证器}\label{1126-ux81eaux5b9aux4e49ux5f02ux6b65ux8868ux5355ux9a8cux8bc1ux5668}}

\begin{Shaded}
\begin{Highlighting}[]
\ImportTok{import}\NormalTok{ \{ AbstractControl}\OperatorTok{,}\NormalTok{ ValidationErrors \} }\ImportTok{from} \StringTok{"@angular/forms"}
\ImportTok{import}\NormalTok{ \{ Observable \} }\ImportTok{from} \StringTok{"rxjs"}

\ImportTok{export} \KeywordTok{class}\NormalTok{ NameValidators \{}
  \KeywordTok{static} \FunctionTok{shouldBeUnique}\NormalTok{(}\DataTypeTok{control}\OperatorTok{:}\NormalTok{ AbstractControl)}\OperatorTok{:} \BuiltInTok{Promise}\OperatorTok{\textless{}}\NormalTok{ValidationErrors }\OperatorTok{|} \KeywordTok{null}\OperatorTok{\textgreater{}}\NormalTok{ \{}
    \ControlFlowTok{return} \KeywordTok{new} \BuiltInTok{Promise}\NormalTok{(resolve }\KeywordTok{=\textgreater{}}\NormalTok{ \{}
      \ControlFlowTok{if}\NormalTok{ (control}\OperatorTok{.}\AttributeTok{value} \OperatorTok{==} \StringTok{"admin"}\NormalTok{) \{}
         \FunctionTok{resolve}\NormalTok{(\{ }\DataTypeTok{shouldBeUnique}\OperatorTok{:} \KeywordTok{true}\NormalTok{ \})}
\NormalTok{       \} }\ControlFlowTok{else}\NormalTok{ \{}
         \FunctionTok{resolve}\NormalTok{(}\KeywordTok{null}\NormalTok{)}
\NormalTok{       \}}
\NormalTok{    \})}
\NormalTok{  \}}
\NormalTok{\}}
\end{Highlighting}
\end{Shaded}

\begin{Shaded}
\begin{Highlighting}[]
\NormalTok{contactForm}\OperatorTok{:}\NormalTok{ FormGroup }\OperatorTok{=} \KeywordTok{new} \FunctionTok{FormGroup}\NormalTok{(\{}
    \DataTypeTok{name}\OperatorTok{:} \KeywordTok{new} \FunctionTok{FormControl}\NormalTok{(}
      \StringTok{""}\OperatorTok{,}
\NormalTok{      [}
\NormalTok{        Validators}\OperatorTok{.}\AttributeTok{required}
\NormalTok{      ]}\OperatorTok{,}
\NormalTok{      NameValidators}\OperatorTok{.}\AttributeTok{shouldBeUnique}
\NormalTok{    )}
\NormalTok{  \})}
\end{Highlighting}
\end{Shaded}

\begin{Shaded}
\begin{Highlighting}[]
\KeywordTok{\textless{}div}\OtherTok{ *ngIf=}\StringTok{"name.touched }\ErrorTok{\&\&}\StringTok{ name.invalid }\ErrorTok{\&\&}\StringTok{ name.errors"}\KeywordTok{\textgreater{}}
  \KeywordTok{\textless{}div}\OtherTok{ *ngIf=}\StringTok{"name.errors.shouldBeUnique"}\KeywordTok{\textgreater{}}\NormalTok{用户名重复}\KeywordTok{\textless{}/div\textgreater{}}
\KeywordTok{\textless{}/div\textgreater{}}
\KeywordTok{\textless{}div}\OtherTok{ *ngIf=}\StringTok{"name.pending"}\KeywordTok{\textgreater{}}\NormalTok{正在检测姓名是否重复}\KeywordTok{\textless{}/div\textgreater{}}
\end{Highlighting}
\end{Shaded}

\hypertarget{1127-formbuilder}{%
\subparagraph{11.2.7 FormBuilder}\label{1127-formbuilder}}

创建表单的快捷方式。

\begin{enumerate}
\def\labelenumi{\arabic{enumi}.}
\item
  \texttt{this.fb.control}:表单项
\item
  \texttt{this.fb.group}:表单组,表单至少是一个 FormGroup
\item
  \texttt{this.fb.array}:用于复杂表单,可以动态添加表单项或表单组,在表单验证时,FormArray
  中有一项没通过,整体没通过。
\end{enumerate}

\begin{Shaded}
\begin{Highlighting}[]
\ImportTok{import}\NormalTok{ \{ FormBuilder}\OperatorTok{,}\NormalTok{ FormGroup}\OperatorTok{,}\NormalTok{ Validators \} }\ImportTok{from} \StringTok{"@angular/forms"}

\ImportTok{export} \KeywordTok{class}\NormalTok{ AppComponent \{}
  \DataTypeTok{contactForm}\OperatorTok{:}\NormalTok{ FormGroup}
  \FunctionTok{constructor}\NormalTok{(}\KeywordTok{private} \DataTypeTok{fb}\OperatorTok{:}\NormalTok{ FormBuilder) \{}
    \KeywordTok{this}\OperatorTok{.}\AttributeTok{contactForm} \OperatorTok{=} \KeywordTok{this}\OperatorTok{.}\AttributeTok{fb}\OperatorTok{.}\FunctionTok{group}\NormalTok{(\{}
      \DataTypeTok{fullName}\OperatorTok{:} \KeywordTok{this}\OperatorTok{.}\AttributeTok{fb}\OperatorTok{.}\FunctionTok{group}\NormalTok{(\{}
        \DataTypeTok{firstName}\OperatorTok{:}\NormalTok{ [}\StringTok{"😝"}\OperatorTok{,}\NormalTok{ [Validators}\OperatorTok{.}\AttributeTok{required}\NormalTok{]]}\OperatorTok{,}
        \DataTypeTok{lastName}\OperatorTok{:}\NormalTok{ [}\StringTok{""}\NormalTok{]}
\NormalTok{      \})}\OperatorTok{,}
      \DataTypeTok{phone}\OperatorTok{:}\NormalTok{ []}
\NormalTok{    \})}
\NormalTok{  \}}
\NormalTok{\}}
\end{Highlighting}
\end{Shaded}

\hypertarget{1128-ux7ec3ux4e60}{%
\subparagraph{11.2.8 练习}\label{1128-ux7ec3ux4e60}}

\begin{enumerate}
\def\labelenumi{\arabic{enumi}.}
\item
  获取一组复选框中选中的值

\begin{Shaded}
\begin{Highlighting}[]
\KeywordTok{\textless{}form}\OtherTok{ [formGroup]=}\StringTok{"form"}\OtherTok{ (submit)=}\StringTok{"onSubmit()"}\KeywordTok{\textgreater{}}
  \KeywordTok{\textless{}label}\OtherTok{ *ngFor=}\StringTok{"let item of Data"}\KeywordTok{\textgreater{}}
    \KeywordTok{\textless{}input}\OtherTok{ type=}\StringTok{"checkbox"}\OtherTok{ [value]=}\StringTok{"item.value"}\OtherTok{ (change)=}\StringTok{"onChange($event)"} \KeywordTok{/\textgreater{}}
\NormalTok{    \{\{ item.name \}\}}
  \KeywordTok{\textless{}/label\textgreater{}}
  \KeywordTok{\textless{}button\textgreater{}}\NormalTok{提交\textless{}/button\textgreater{}}
\KeywordTok{\textless{}/form\textgreater{}}
\end{Highlighting}
\end{Shaded}

\begin{Shaded}
\begin{Highlighting}[]
\ImportTok{import}\NormalTok{ \{ Component \} }\ImportTok{from} \StringTok{"@angular/core"}
\ImportTok{import}\NormalTok{ \{ FormArray}\OperatorTok{,}\NormalTok{ FormBuilder}\OperatorTok{,}\NormalTok{ FormGroup \} }\ImportTok{from} \StringTok{"@angular/forms"}
\KeywordTok{interface}\NormalTok{ Data \{}
  \DataTypeTok{name}\OperatorTok{:}\NormalTok{ string}
  \DataTypeTok{value}\OperatorTok{:}\NormalTok{ string}
\NormalTok{\}}
\NormalTok{@}\FunctionTok{Component}\NormalTok{(\{}
  \DataTypeTok{selector}\OperatorTok{:} \StringTok{"app{-}checkbox"}\OperatorTok{,}
  \DataTypeTok{templateUrl}\OperatorTok{:} \StringTok{"./checkbox.component.html"}\OperatorTok{,}
  \DataTypeTok{styles}\OperatorTok{:}\NormalTok{ []}
\NormalTok{\})}
\ImportTok{export} \KeywordTok{class}\NormalTok{ CheckboxComponent \{}
  \DataTypeTok{Data}\OperatorTok{:} \BuiltInTok{Array}\OperatorTok{\textless{}}\NormalTok{Data}\OperatorTok{\textgreater{}} \OperatorTok{=}\NormalTok{ [}
\NormalTok{    \{ }\DataTypeTok{name}\OperatorTok{:} \StringTok{"Pear"}\OperatorTok{,} \DataTypeTok{value}\OperatorTok{:} \StringTok{"pear"}\NormalTok{ \}}\OperatorTok{,}
\NormalTok{    \{ }\DataTypeTok{name}\OperatorTok{:} \StringTok{"Plum"}\OperatorTok{,} \DataTypeTok{value}\OperatorTok{:} \StringTok{"plum"}\NormalTok{ \}}\OperatorTok{,}
\NormalTok{    \{ }\DataTypeTok{name}\OperatorTok{:} \StringTok{"Kiwi"}\OperatorTok{,} \DataTypeTok{value}\OperatorTok{:} \StringTok{"kiwi"}\NormalTok{ \}}\OperatorTok{,}
\NormalTok{    \{ }\DataTypeTok{name}\OperatorTok{:} \StringTok{"Apple"}\OperatorTok{,} \DataTypeTok{value}\OperatorTok{:} \StringTok{"apple"}\NormalTok{ \}}\OperatorTok{,}
\NormalTok{    \{ }\DataTypeTok{name}\OperatorTok{:} \StringTok{"Lime"}\OperatorTok{,} \DataTypeTok{value}\OperatorTok{:} \StringTok{"lime"}\NormalTok{ \}}
\NormalTok{  ]}
  \DataTypeTok{form}\OperatorTok{:}\NormalTok{ FormGroup}

  \FunctionTok{constructor}\NormalTok{(}\KeywordTok{private} \DataTypeTok{fb}\OperatorTok{:}\NormalTok{ FormBuilder) \{}
    \KeywordTok{this}\OperatorTok{.}\AttributeTok{form} \OperatorTok{=} \KeywordTok{this}\OperatorTok{.}\AttributeTok{fb}\OperatorTok{.}\FunctionTok{group}\NormalTok{(\{}
      \DataTypeTok{checkArray}\OperatorTok{:} \KeywordTok{this}\OperatorTok{.}\AttributeTok{fb}\OperatorTok{.}\FunctionTok{array}\NormalTok{([])}
\NormalTok{    \})}
\NormalTok{  \}}

  \FunctionTok{onChange}\NormalTok{(}\DataTypeTok{event}\OperatorTok{:} \BuiltInTok{Event}\NormalTok{) \{}
    \KeywordTok{const}\NormalTok{ target }\OperatorTok{=} \BuiltInTok{event}\OperatorTok{.}\AttributeTok{target} \ImportTok{as} \BuiltInTok{HTMLInputElement}
    \KeywordTok{const}\NormalTok{ checked }\OperatorTok{=}\NormalTok{ target}\OperatorTok{.}\AttributeTok{checked}
    \KeywordTok{const}\NormalTok{ value }\OperatorTok{=}\NormalTok{ target}\OperatorTok{.}\AttributeTok{value}
    \KeywordTok{const}\NormalTok{ checkArray }\OperatorTok{=} \KeywordTok{this}\OperatorTok{.}\AttributeTok{form}\OperatorTok{.}\FunctionTok{get}\NormalTok{(}\StringTok{"checkArray"}\NormalTok{) }\ImportTok{as}\NormalTok{ FormArray}

    \ControlFlowTok{if}\NormalTok{ (checked) \{}
\NormalTok{      checkArray}\OperatorTok{.}\FunctionTok{push}\NormalTok{(}\KeywordTok{this}\OperatorTok{.}\AttributeTok{fb}\OperatorTok{.}\FunctionTok{control}\NormalTok{(value))}
\NormalTok{    \} }\ControlFlowTok{else}\NormalTok{ \{}
      \KeywordTok{const}\NormalTok{ index }\OperatorTok{=}\NormalTok{ checkArray}\OperatorTok{.}\AttributeTok{controls}\OperatorTok{.}\FunctionTok{findIndex}\NormalTok{(}
\NormalTok{        control }\KeywordTok{=\textgreater{}}\NormalTok{ control}\OperatorTok{.}\AttributeTok{value} \OperatorTok{===}\NormalTok{ value}
\NormalTok{      )}
\NormalTok{      checkArray}\OperatorTok{.}\FunctionTok{removeAt}\NormalTok{(index)}
\NormalTok{    \}}
\NormalTok{  \}}

  \FunctionTok{onSubmit}\NormalTok{() \{}
    \BuiltInTok{console}\OperatorTok{.}\FunctionTok{log}\NormalTok{(}\KeywordTok{this}\OperatorTok{.}\AttributeTok{form}\OperatorTok{.}\AttributeTok{value}\NormalTok{)}
\NormalTok{  \}}
\NormalTok{\}}
\end{Highlighting}
\end{Shaded}
\item
  获取单选框中选中的值

\begin{Shaded}
\begin{Highlighting}[]
\ImportTok{export} \KeywordTok{class}\NormalTok{ AppComponent \{}
  \DataTypeTok{form}\OperatorTok{:}\NormalTok{ FormGroup}

  \FunctionTok{constructor}\NormalTok{(}\KeywordTok{public} \DataTypeTok{fb}\OperatorTok{:}\NormalTok{ FormBuilder) \{}
    \KeywordTok{this}\OperatorTok{.}\AttributeTok{form} \OperatorTok{=} \KeywordTok{this}\OperatorTok{.}\AttributeTok{fb}\OperatorTok{.}\FunctionTok{group}\NormalTok{(\{ }\DataTypeTok{gender}\OperatorTok{:} \StringTok{""}\NormalTok{ \})}
\NormalTok{  \}}

  \FunctionTok{onSubmit}\NormalTok{() \{}
    \BuiltInTok{console}\OperatorTok{.}\FunctionTok{log}\NormalTok{(}\KeywordTok{this}\OperatorTok{.}\AttributeTok{form}\OperatorTok{.}\AttributeTok{value}\NormalTok{)}
\NormalTok{  \}}
\NormalTok{\}}
\end{Highlighting}
\end{Shaded}

\begin{Shaded}
\begin{Highlighting}[]
\KeywordTok{\textless{}form}\OtherTok{ [formGroup]=}\StringTok{"form"}\OtherTok{ (submit)=}\StringTok{"onSubmit()"}\KeywordTok{\textgreater{}}
  \KeywordTok{\textless{}input}\OtherTok{ type=}\StringTok{"radio"}\OtherTok{ value=}\StringTok{"male"}\OtherTok{ formControlName=}\StringTok{"gender"} \KeywordTok{/\textgreater{}}\NormalTok{ Male}
  \KeywordTok{\textless{}input}\OtherTok{ type=}\StringTok{"radio"}\OtherTok{ value=}\StringTok{"female"}\OtherTok{ formControlName=}\StringTok{"gender"} \KeywordTok{/\textgreater{}}\NormalTok{ Female}
  \KeywordTok{\textless{}button}\OtherTok{ type=}\StringTok{"submit"}\KeywordTok{\textgreater{}}\NormalTok{Submit}\KeywordTok{\textless{}/button\textgreater{}}
\KeywordTok{\textless{}/form\textgreater{}}
\end{Highlighting}
\end{Shaded}
\end{enumerate}

\hypertarget{1129-ux5176ux4ed6}{%
\subparagraph{11.2.9 其他}\label{1129-ux5176ux4ed6}}

\begin{enumerate}
\def\labelenumi{\arabic{enumi}.}
\item
  patchValue:设置表单控件的值(可以设置全部,也可以设置其中某一个,其他不受影响)
\item
  setValue:设置表单控件的值 (设置全部,不能排除任何一个)
\item
  valueChanges:当表单控件的值发生变化时被触发的事件
\item
  reset:表单内容置空
\end{enumerate}

\hypertarget{12-ux8defux7531}{%
\subsubsection{12. 路由}\label{12-ux8defux7531}}

\hypertarget{121-ux6982ux8ff0}{%
\paragraph{12.1 概述}\label{121-ux6982ux8ff0}}

在 Angular 中,路由是以模块为单位的,每个模块都可以有自己的路由。

\hypertarget{122-ux5febux901fux4e0aux624b}{%
\paragraph{12.2 快速上手}\label{122-ux5febux901fux4e0aux624b}}

\begin{enumerate}
\def\labelenumi{\arabic{enumi}.}
\item
  创建页面组件、Layout 组件以及 Navigation 组件,供路由使用

  \begin{enumerate}
  \def\labelenumii{\arabic{enumii}.}
  \item
    创建\textbf{首页}页面组件\texttt{ng\ g\ c\ pages/home}
  \item
    创建\textbf{关于我们}页面组件\texttt{ng\ g\ c\ pages/about}
  \item
    创建\textbf{布局}组件\texttt{ng\ g\ c\ pages/layout}
  \item
    创建\textbf{导航}组件\texttt{ng\ g\ c\ pages/navigation}
  \end{enumerate}
\item
  创建路由规则

\begin{Shaded}
\begin{Highlighting}[]
\CommentTok{// app.module.ts}
\ImportTok{import}\NormalTok{ \{ Routes \} }\ImportTok{from} \StringTok{"@angular/router"}

\KeywordTok{const}\NormalTok{ routes}\OperatorTok{:}\NormalTok{ Routes }\OperatorTok{=}\NormalTok{ [}
\NormalTok{  \{}
    \DataTypeTok{path}\OperatorTok{:} \StringTok{"home"}\OperatorTok{,}
    \DataTypeTok{component}\OperatorTok{:}\NormalTok{ HomeComponent}
\NormalTok{  \}}\OperatorTok{,}
\NormalTok{  \{}
    \DataTypeTok{path}\OperatorTok{:} \StringTok{"about"}\OperatorTok{,}
    \DataTypeTok{component}\OperatorTok{:}\NormalTok{ AboutComponent}
\NormalTok{  \}}
\NormalTok{]}
\end{Highlighting}
\end{Shaded}
\item
  引入路由模块并启动

\begin{Shaded}
\begin{Highlighting}[]
\CommentTok{// app.module.ts}
\ImportTok{import}\NormalTok{ \{ RouterModule}\OperatorTok{,}\NormalTok{ Routes \} }\ImportTok{from} \StringTok{"@angular/router"}

\NormalTok{@}\FunctionTok{NgModule}\NormalTok{(\{}
  \DataTypeTok{imports}\OperatorTok{:}\NormalTok{ [RouterModule}\OperatorTok{.}\FunctionTok{forRoot}\NormalTok{(routes}\OperatorTok{,}\NormalTok{ \{ }\DataTypeTok{useHash}\OperatorTok{:} \KeywordTok{true}\NormalTok{ \})]}\OperatorTok{,}
\NormalTok{\})}
\ImportTok{export} \KeywordTok{class}\NormalTok{ AppModule \{\}}
\end{Highlighting}
\end{Shaded}
\item
  添加路由插座

\begin{Shaded}
\begin{Highlighting}[]
\CommentTok{\textless{}!{-}{-} 路由插座即占位组件 匹配到的路由组件将会显示在这个地方 {-}{-}\textgreater{}}
\KeywordTok{\textless{}router{-}outlet\textgreater{}\textless{}/router{-}outlet\textgreater{}}
\end{Highlighting}
\end{Shaded}
\item
  在导航组件中定义链接

\begin{Shaded}
\begin{Highlighting}[]
\KeywordTok{\textless{}a}\OtherTok{ routerLink=}\StringTok{"/home"}\KeywordTok{\textgreater{}}\NormalTok{首页}\KeywordTok{\textless{}/a\textgreater{}}
\KeywordTok{\textless{}a}\OtherTok{ routerLink=}\StringTok{"/about"}\KeywordTok{\textgreater{}}\NormalTok{关于我们}\KeywordTok{\textless{}/a\textgreater{}}
\end{Highlighting}
\end{Shaded}
\end{enumerate}

\hypertarget{123-ux5339ux914dux89c4ux5219}{%
\paragraph{12.3 匹配规则}\label{123-ux5339ux914dux89c4ux5219}}

\hypertarget{1231-ux91cdux5b9aux5411}{%
\subparagraph{12.3.1 重定向}\label{1231-ux91cdux5b9aux5411}}

\begin{Shaded}
\begin{Highlighting}[]
\KeywordTok{const}\NormalTok{ routes}\OperatorTok{:}\NormalTok{ Routes }\OperatorTok{=}\NormalTok{ [}
\NormalTok{  \{}
    \DataTypeTok{path}\OperatorTok{:} \StringTok{"home"}\OperatorTok{,}
    \DataTypeTok{component}\OperatorTok{:}\NormalTok{ HomeComponent}
\NormalTok{  \}}\OperatorTok{,}
\NormalTok{  \{}
    \DataTypeTok{path}\OperatorTok{:} \StringTok{"about"}\OperatorTok{,}
    \DataTypeTok{component}\OperatorTok{:}\NormalTok{ AboutComponent}
\NormalTok{  \}}\OperatorTok{,}
\NormalTok{  \{}
    \DataTypeTok{path}\OperatorTok{:} \StringTok{""}\OperatorTok{,}
    \CommentTok{// 重定向}
    \DataTypeTok{redirectTo}\OperatorTok{:} \StringTok{"home"}\OperatorTok{,}
    \CommentTok{// 完全匹配}
    \DataTypeTok{pathMatch}\OperatorTok{:} \StringTok{"full"}
\NormalTok{  \}}
\NormalTok{]}
\end{Highlighting}
\end{Shaded}

\hypertarget{1232-404-ux9875ux9762}{%
\subparagraph{12.3.2 404 页面}\label{1232-404-ux9875ux9762}}

\begin{Shaded}
\begin{Highlighting}[]
\KeywordTok{const}\NormalTok{ routes}\OperatorTok{:}\NormalTok{ Routes }\OperatorTok{=}\NormalTok{ [}
\NormalTok{  \{}
    \DataTypeTok{path}\OperatorTok{:} \StringTok{"home"}\OperatorTok{,}
    \DataTypeTok{component}\OperatorTok{:}\NormalTok{ HomeComponent}
\NormalTok{  \}}\OperatorTok{,}
\NormalTok{  \{}
    \DataTypeTok{path}\OperatorTok{:} \StringTok{"about"}\OperatorTok{,}
    \DataTypeTok{component}\OperatorTok{:}\NormalTok{ AboutComponent}
\NormalTok{  \}}\OperatorTok{,}
\NormalTok{  \{}
    \DataTypeTok{path}\OperatorTok{:} \StringTok{"**"}\OperatorTok{,}
    \DataTypeTok{component}\OperatorTok{:}\NormalTok{ NotFoundComponent}
\NormalTok{  \}}
\NormalTok{]}
\end{Highlighting}
\end{Shaded}

\hypertarget{124-ux8defux7531ux4f20ux53c2}{%
\paragraph{12.4 路由传参}\label{124-ux8defux7531ux4f20ux53c2}}

\hypertarget{1241-ux67e5ux8be2ux53c2ux6570}{%
\subparagraph{12.4.1 查询参数}\label{1241-ux67e5ux8be2ux53c2ux6570}}

\begin{Shaded}
\begin{Highlighting}[]
\KeywordTok{\textless{}a}\OtherTok{ routerLink=}\StringTok{"/about"}\OtherTok{ [queryParams]=}\StringTok{"\{ name: \textquotesingle{}kitty\textquotesingle{} \}"}\KeywordTok{\textgreater{}}\NormalTok{关于我们}\KeywordTok{\textless{}/a\textgreater{}}
\end{Highlighting}
\end{Shaded}

\begin{Shaded}
\begin{Highlighting}[]
\ImportTok{import}\NormalTok{ \{ ActivatedRoute \} }\ImportTok{from} \StringTok{"@angular/router"}

\ImportTok{export} \KeywordTok{class}\NormalTok{ AboutComponent }\KeywordTok{implements}\NormalTok{ OnInit \{}
  \FunctionTok{constructor}\NormalTok{(}\KeywordTok{private} \DataTypeTok{route}\OperatorTok{:}\NormalTok{ ActivatedRoute) \{\}}

  \FunctionTok{ngOnInit}\NormalTok{()}\OperatorTok{:} \KeywordTok{void}\NormalTok{ \{}
    \KeywordTok{this}\OperatorTok{.}\AttributeTok{route}\OperatorTok{.}\AttributeTok{queryParamMap}\OperatorTok{.}\FunctionTok{subscribe}\NormalTok{(query }\KeywordTok{=\textgreater{}}\NormalTok{ \{}
\NormalTok{      query}\OperatorTok{.}\FunctionTok{get}\NormalTok{(}\StringTok{"name"}\NormalTok{)}
\NormalTok{    \})}
\NormalTok{  \}}
\NormalTok{\}}
\end{Highlighting}
\end{Shaded}

\hypertarget{1242-ux52a8ux6001ux53c2ux6570}{%
\subparagraph{12.4.2 动态参数}\label{1242-ux52a8ux6001ux53c2ux6570}}

\begin{Shaded}
\begin{Highlighting}[]
\KeywordTok{const}\NormalTok{ routes}\OperatorTok{:}\NormalTok{ Routes }\OperatorTok{=}\NormalTok{ [}
\NormalTok{  \{}
    \DataTypeTok{path}\OperatorTok{:} \StringTok{"home"}\OperatorTok{,}
    \DataTypeTok{component}\OperatorTok{:}\NormalTok{ HomeComponent}
\NormalTok{  \}}\OperatorTok{,}
\NormalTok{  \{}
    \DataTypeTok{path}\OperatorTok{:} \StringTok{"about/:name"}\OperatorTok{,}
    \DataTypeTok{component}\OperatorTok{:}\NormalTok{ AboutComponent}
\NormalTok{  \}}
\NormalTok{]}
\end{Highlighting}
\end{Shaded}

\begin{Shaded}
\begin{Highlighting}[]
\KeywordTok{\textless{}a}\OtherTok{ [routerLink]=}\StringTok{"[\textquotesingle{}/about\textquotesingle{}, \textquotesingle{}zhangsan\textquotesingle{}]"}\KeywordTok{\textgreater{}}\NormalTok{关于我们}\KeywordTok{\textless{}/a\textgreater{}}
\end{Highlighting}
\end{Shaded}

\begin{Shaded}
\begin{Highlighting}[]
\ImportTok{import}\NormalTok{ \{ ActivatedRoute \} }\ImportTok{from} \StringTok{"@angular/router"}

\ImportTok{export} \KeywordTok{class}\NormalTok{ AboutComponent }\KeywordTok{implements}\NormalTok{ OnInit \{}
  \FunctionTok{constructor}\NormalTok{(}\KeywordTok{private} \DataTypeTok{route}\OperatorTok{:}\NormalTok{ ActivatedRoute) \{\}}

  \FunctionTok{ngOnInit}\NormalTok{()}\OperatorTok{:} \KeywordTok{void}\NormalTok{ \{}
    \KeywordTok{this}\OperatorTok{.}\AttributeTok{route}\OperatorTok{.}\AttributeTok{paramMap}\OperatorTok{.}\FunctionTok{subscribe}\NormalTok{(params }\KeywordTok{=\textgreater{}}\NormalTok{ \{}
\NormalTok{      params}\OperatorTok{.}\FunctionTok{get}\NormalTok{(}\StringTok{"name"}\NormalTok{)}
\NormalTok{    \})}
\NormalTok{  \}}
\NormalTok{\}}
\end{Highlighting}
\end{Shaded}

\hypertarget{125-ux8defux7531ux5d4cux5957}{%
\paragraph{12.5 路由嵌套}\label{125-ux8defux7531ux5d4cux5957}}

路由嵌套指的是如何定义子级路由。

\begin{Shaded}
\begin{Highlighting}[]
\KeywordTok{const}\NormalTok{ routes}\OperatorTok{:}\NormalTok{ Routes }\OperatorTok{=}\NormalTok{ [}
\NormalTok{  \{}
    \DataTypeTok{path}\OperatorTok{:} \StringTok{"about"}\OperatorTok{,}
    \DataTypeTok{component}\OperatorTok{:}\NormalTok{ AboutComponent}\OperatorTok{,}
    \DataTypeTok{children}\OperatorTok{:}\NormalTok{ [}
\NormalTok{      \{}
        \DataTypeTok{path}\OperatorTok{:} \StringTok{"introduce"}\OperatorTok{,}
        \DataTypeTok{component}\OperatorTok{:}\NormalTok{ IntroduceComponent}
\NormalTok{      \}}\OperatorTok{,}
\NormalTok{      \{}
        \DataTypeTok{path}\OperatorTok{:} \StringTok{"history"}\OperatorTok{,}
        \DataTypeTok{component}\OperatorTok{:}\NormalTok{ HistoryComponent}
\NormalTok{      \}}
\NormalTok{    ]}
\NormalTok{  \}}
\NormalTok{]}
\end{Highlighting}
\end{Shaded}

\begin{Shaded}
\begin{Highlighting}[]
\CommentTok{\textless{}!{-}{-} about.component.html {-}{-}\textgreater{}}
\KeywordTok{\textless{}app{-}layout\textgreater{}}
  \KeywordTok{\textless{}p\textgreater{}}\NormalTok{about works!}\KeywordTok{\textless{}/p\textgreater{}}
  \KeywordTok{\textless{}a}\OtherTok{ routerLink=}\StringTok{"/about/introduce"}\KeywordTok{\textgreater{}}\NormalTok{公司简介}\KeywordTok{\textless{}/a\textgreater{}}
  \KeywordTok{\textless{}a}\OtherTok{ routerLink=}\StringTok{"/about/history"}\KeywordTok{\textgreater{}}\NormalTok{发展历史}\KeywordTok{\textless{}/a\textgreater{}}
  \KeywordTok{\textless{}div\textgreater{}}
    \KeywordTok{\textless{}router{-}outlet\textgreater{}\textless{}/router{-}outlet\textgreater{}}
  \KeywordTok{\textless{}/div\textgreater{}}
\KeywordTok{\textless{}/app{-}layout\textgreater{}}
\end{Highlighting}
\end{Shaded}

\hypertarget{126-ux547dux540dux63d2ux5ea7}{%
\paragraph{12.6 命名插座}\label{126-ux547dux540dux63d2ux5ea7}}

将子级路由组件显示到不同的路由插座中。

\begin{Shaded}
\begin{Highlighting}[]
\NormalTok{\{}
  \DataTypeTok{path}\OperatorTok{:} \StringTok{"about"}\OperatorTok{,}
  \DataTypeTok{component}\OperatorTok{:}\NormalTok{ AboutComponent}\OperatorTok{,}
  \DataTypeTok{children}\OperatorTok{:}\NormalTok{ [}
\NormalTok{    \{}
      \DataTypeTok{path}\OperatorTok{:} \StringTok{"introduce"}\OperatorTok{,}
      \DataTypeTok{component}\OperatorTok{:}\NormalTok{ IntroduceComponent}\OperatorTok{,}
      \DataTypeTok{outlet}\OperatorTok{:} \StringTok{"left"}
\NormalTok{    \}}\OperatorTok{,}
\NormalTok{    \{}
      \DataTypeTok{path}\OperatorTok{:} \StringTok{"history"}\OperatorTok{,}
      \DataTypeTok{component}\OperatorTok{:}\NormalTok{ HistoryComponent}\OperatorTok{,}
      \DataTypeTok{outlet}\OperatorTok{:} \StringTok{"right"}
\NormalTok{    \}}
\NormalTok{  ]}
\NormalTok{\}}
\end{Highlighting}
\end{Shaded}

\begin{Shaded}
\begin{Highlighting}[]
\CommentTok{\textless{}!{-}{-} about.component.html {-}{-}\textgreater{}}
\KeywordTok{\textless{}app{-}layout\textgreater{}}
  \KeywordTok{\textless{}p\textgreater{}}\NormalTok{about works!}\KeywordTok{\textless{}/p\textgreater{}}
  \KeywordTok{\textless{}router{-}outlet}\OtherTok{ name=}\StringTok{"left"}\KeywordTok{\textgreater{}\textless{}/router{-}outlet\textgreater{}}
  \KeywordTok{\textless{}router{-}outlet}\OtherTok{ name=}\StringTok{"right"}\KeywordTok{\textgreater{}\textless{}/router{-}outlet\textgreater{}}
\KeywordTok{\textless{}/app{-}layout\textgreater{}}
\end{Highlighting}
\end{Shaded}

\begin{Shaded}
\begin{Highlighting}[]
\KeywordTok{\textless{}a}
\OtherTok{    [routerLink]=}\StringTok{"[}
\StringTok{      \textquotesingle{}/about\textquotesingle{},}
\StringTok{      \{}
\StringTok{        outlets: \{}
\StringTok{          left: [\textquotesingle{}introduce\textquotesingle{}],}
\StringTok{          right: [\textquotesingle{}history\textquotesingle{}]}
\StringTok{        \}}
\StringTok{      \}}
\StringTok{    ]"}
    \KeywordTok{\textgreater{}}\NormalTok{关于我们}
\KeywordTok{\textless{}/a\textgreater{}}
\end{Highlighting}
\end{Shaded}

\hypertarget{127-ux5bfcux822aux8defux7531}{%
\paragraph{12.7 导航路由}\label{127-ux5bfcux822aux8defux7531}}

\begin{Shaded}
\begin{Highlighting}[]
\CommentTok{\textless{}!{-}{-} app.component.html {-}{-}\textgreater{}}
 \KeywordTok{\textless{}button}\OtherTok{ (click)=}\StringTok{"jump()"}\KeywordTok{\textgreater{}}\NormalTok{跳转到发展历史}\KeywordTok{\textless{}/button\textgreater{}}
\end{Highlighting}
\end{Shaded}

\begin{Shaded}
\begin{Highlighting}[]
\CommentTok{// app.component.ts}
\ImportTok{import}\NormalTok{ \{ Router \} }\ImportTok{from} \StringTok{"@angular/router"}

\ImportTok{export} \KeywordTok{class}\NormalTok{ HomeComponent \{}
  \FunctionTok{constructor}\NormalTok{(}\KeywordTok{private} \DataTypeTok{router}\OperatorTok{:}\NormalTok{ Router) \{\}}
  \FunctionTok{jump}\NormalTok{() \{}
    \KeywordTok{this}\OperatorTok{.}\AttributeTok{router}\OperatorTok{.}\FunctionTok{navigate}\NormalTok{([}\StringTok{"/about/history"}\NormalTok{]}\OperatorTok{,}\NormalTok{ \{}
      \DataTypeTok{queryParams}\OperatorTok{:}\NormalTok{ \{}
        \DataTypeTok{name}\OperatorTok{:} \StringTok{"Kitty"}
\NormalTok{      \}}
\NormalTok{    \})}
\NormalTok{  \}}
\NormalTok{\}}
\end{Highlighting}
\end{Shaded}

\hypertarget{128-ux8defux7531ux6a21ux5757}{%
\paragraph{12.8 路由模块}\label{128-ux8defux7531ux6a21ux5757}}

将根模块中的路由配置抽象成一个单独的路由模块,称之为根路由模块,然后在根模块中引入根路由模块。

\begin{Shaded}
\begin{Highlighting}[]
\ImportTok{import}\NormalTok{ \{ NgModule \} }\ImportTok{from} \StringTok{"@angular/core"}

\ImportTok{import}\NormalTok{ \{ HomeComponent \} }\ImportTok{from} \StringTok{"./pages/home/home.component"}
\ImportTok{import}\NormalTok{ \{ NotFoundComponent \} }\ImportTok{from} \StringTok{"./pages/not{-}found/not{-}found.component"}

\KeywordTok{const}\NormalTok{ routes}\OperatorTok{:}\NormalTok{ Routes }\OperatorTok{=}\NormalTok{ [}
\NormalTok{  \{}
    \DataTypeTok{path}\OperatorTok{:} \StringTok{""}\OperatorTok{,}
    \DataTypeTok{component}\OperatorTok{:}\NormalTok{ HomeComponent}
\NormalTok{  \}}\OperatorTok{,}
\NormalTok{  \{}
    \DataTypeTok{path}\OperatorTok{:} \StringTok{"**"}\OperatorTok{,}
    \DataTypeTok{component}\OperatorTok{:}\NormalTok{ NotFoundComponent}
\NormalTok{  \}}
\NormalTok{]}

\NormalTok{@}\FunctionTok{NgModule}\NormalTok{(\{}
  \DataTypeTok{declarations}\OperatorTok{:}\NormalTok{ []}\OperatorTok{,}
  \DataTypeTok{imports}\OperatorTok{:}\NormalTok{ [RouterModule}\OperatorTok{.}\FunctionTok{forRoot}\NormalTok{(routes}\OperatorTok{,}\NormalTok{ \{ }\DataTypeTok{useHash}\OperatorTok{:} \KeywordTok{true}\NormalTok{ \})]}\OperatorTok{,}
  \CommentTok{// 导出 Angular 路由功能模块,因为在根模块的根组件中使用了 RouterModule 模块中提供的路由插座组件}
  \DataTypeTok{exports}\OperatorTok{:}\NormalTok{ [RouterModule]}
\NormalTok{\})}
\ImportTok{export} \KeywordTok{class}\NormalTok{ AppRoutingModule \{\}}
\end{Highlighting}
\end{Shaded}

\begin{Shaded}
\begin{Highlighting}[]
\ImportTok{import}\NormalTok{ \{ BrowserModule \} }\ImportTok{from} \StringTok{"@angular/platform{-}browser"}
\ImportTok{import}\NormalTok{ \{ NgModule \} }\ImportTok{from} \StringTok{"@angular/core"}
\ImportTok{import}\NormalTok{ \{ AppComponent \} }\ImportTok{from} \StringTok{"./app.component"}
\ImportTok{import}\NormalTok{ \{ AppRoutingModule \} }\ImportTok{from} \StringTok{"./app{-}routing.module"}
\ImportTok{import}\NormalTok{ \{ HomeComponent \} }\ImportTok{from} \StringTok{"./pages/home/home.component"}
\ImportTok{import}\NormalTok{ \{ NotFoundComponent \} }\ImportTok{from} \StringTok{"./pages/not{-}found/not{-}found.component"}

\NormalTok{@}\FunctionTok{NgModule}\NormalTok{(\{}
  \DataTypeTok{declarations}\OperatorTok{:}\NormalTok{ [AppComponent,HomeComponent}\OperatorTok{,}\NormalTok{ NotFoundComponent]}\OperatorTok{,}
  \DataTypeTok{imports}\OperatorTok{:}\NormalTok{ [BrowserModule}\OperatorTok{,}\NormalTok{ AppRoutingModule]}\OperatorTok{,}
  \DataTypeTok{providers}\OperatorTok{:}\NormalTok{ []}\OperatorTok{,}
  \DataTypeTok{bootstrap}\OperatorTok{:}\NormalTok{ [AppComponent]}
\NormalTok{\})}
\ImportTok{export} \KeywordTok{class}\NormalTok{ AppModule \{\}}
\end{Highlighting}
\end{Shaded}

\hypertarget{129-ux8defux7531ux61d2ux52a0ux8f7d}{%
\paragraph{12.9 路由懒加载}\label{129-ux8defux7531ux61d2ux52a0ux8f7d}}

路由懒加载是以模块为单位的。

\begin{enumerate}
\def\labelenumi{\arabic{enumi}.}
\item
  创建用户模块 \texttt{ng\ g\ m\ user\ -\/-routing=true}
  一并创建该模块的路由模块
\item
  创建登录页面组件 \texttt{ng\ g\ c\ user/pages/login}
\item
  创建注册页面组件 \texttt{ng\ g\ c\ user/pages/register}
\item
  配置用户模块的路由规则

\begin{Shaded}
\begin{Highlighting}[]
\ImportTok{import}\NormalTok{ \{ NgModule \} }\ImportTok{from} \StringTok{"@angular/core"}
\ImportTok{import}\NormalTok{ \{ Routes}\OperatorTok{,}\NormalTok{ RouterModule \} }\ImportTok{from} \StringTok{"@angular/router"}
\ImportTok{import}\NormalTok{ \{ LoginComponent \} }\ImportTok{from} \StringTok{"./pages/login/login.component"}
\ImportTok{import}\NormalTok{ \{ RegisterComponent \} }\ImportTok{from} \StringTok{"./pages/register/register.component"}

\KeywordTok{const}\NormalTok{ routes}\OperatorTok{:}\NormalTok{ Routes }\OperatorTok{=}\NormalTok{ [}
\NormalTok{  \{}
    \DataTypeTok{path}\OperatorTok{:} \StringTok{"login"}\OperatorTok{,}
    \DataTypeTok{component}\OperatorTok{:}\NormalTok{ LoginComponent}
\NormalTok{  \}}\OperatorTok{,}
\NormalTok{  \{}
    \DataTypeTok{path}\OperatorTok{:} \StringTok{"register"}\OperatorTok{,}
    \DataTypeTok{component}\OperatorTok{:}\NormalTok{ RegisterComponent}
\NormalTok{  \}}
\NormalTok{]}

\NormalTok{@}\FunctionTok{NgModule}\NormalTok{(\{}
  \DataTypeTok{imports}\OperatorTok{:}\NormalTok{ [RouterModule}\OperatorTok{.}\FunctionTok{forChild}\NormalTok{(routes)]}\OperatorTok{,}
  \DataTypeTok{exports}\OperatorTok{:}\NormalTok{ [RouterModule]}
\NormalTok{\})}
\ImportTok{export} \KeywordTok{class}\NormalTok{ UserRoutingModule \{\}}
\end{Highlighting}
\end{Shaded}
\item
  将用户路由模块关联到主路由模块

\begin{Shaded}
\begin{Highlighting}[]
\CommentTok{// app{-}routing.module.ts}
\KeywordTok{const}\NormalTok{ routes}\OperatorTok{:}\NormalTok{ Routes }\OperatorTok{=}\NormalTok{ [}
\NormalTok{  \{}
    \DataTypeTok{path}\OperatorTok{:} \StringTok{"user"}\OperatorTok{,}
    \DataTypeTok{loadChildren}\OperatorTok{:}\NormalTok{ () }\KeywordTok{=\textgreater{}} \ImportTok{import}\NormalTok{(}\StringTok{"./user/user.module"}\NormalTok{)}\OperatorTok{.}\FunctionTok{then}\NormalTok{(m }\KeywordTok{=\textgreater{}}\NormalTok{ m}\OperatorTok{.}\AttributeTok{UserModule}\NormalTok{)}
\NormalTok{  \}}
\NormalTok{]}
\end{Highlighting}
\end{Shaded}
\item
  在导航组件中添加访问链接

\begin{Shaded}
\begin{Highlighting}[]
\KeywordTok{\textless{}a}\OtherTok{ routerLink=}\StringTok{"/user/login"}\KeywordTok{\textgreater{}}\NormalTok{登录\textless{}/a\textgreater{}}
\KeywordTok{\textless{}a}\OtherTok{ routerLink=}\StringTok{"/user/register"}\KeywordTok{\textgreater{}}\NormalTok{注册}\KeywordTok{\textless{}/a\textgreater{}}
\end{Highlighting}
\end{Shaded}
\end{enumerate}

\hypertarget{1210-ux8defux7531ux5b88ux536b}{%
\paragraph{12.10 路由守卫}\label{1210-ux8defux7531ux5b88ux536b}}

路由守卫会告诉路由是否允许导航到请求的路由。

路由守方法可以返回 boolean 或 Observable
\textless{}boolean\textgreater{} 或 Promise
\textless{}boolean\textgreater{},它们在将来的某个时间点解析为布尔值。

\hypertarget{12101-canactivate}{%
\subparagraph{12.10.1 CanActivate}\label{12101-canactivate}}

检查用户是否可以访问某一个路由。

CanActivate 为接口,路由守卫类要实现该接口,该接口规定类中需要有
canActivate 方法,方法决定是否允许访问目标路由。

路由可以应用多个守卫,所有守卫方法都允许,路由才被允许访问,有一个守卫方法不允许,则路由不允许被访问。

创建路由守卫:\texttt{ng\ g\ guard\ guards/auth}

\begin{Shaded}
\begin{Highlighting}[]
\ImportTok{import}\NormalTok{ \{ Injectable \} }\ImportTok{from} \StringTok{"@angular/core"}
\ImportTok{import}\NormalTok{ \{ CanActivate}\OperatorTok{,}\NormalTok{ ActivatedRouteSnapshot}\OperatorTok{,}\NormalTok{ RouterStateSnapshot}\OperatorTok{,}\NormalTok{ UrlTree}\OperatorTok{,}\NormalTok{ Router \} }\ImportTok{from} \StringTok{"@angular/router"}
\ImportTok{import}\NormalTok{ \{ Observable \} }\ImportTok{from} \StringTok{"rxjs"}

\NormalTok{@}\FunctionTok{Injectable}\NormalTok{(\{}
  \DataTypeTok{providedIn}\OperatorTok{:} \StringTok{"root"}
\NormalTok{\})}
\ImportTok{export} \KeywordTok{class}\NormalTok{ AuthGuard }\KeywordTok{implements}\NormalTok{ CanActivate \{}
  \FunctionTok{constructor}\NormalTok{(}\KeywordTok{private} \DataTypeTok{router}\OperatorTok{:}\NormalTok{ Router) \{\}}
  \FunctionTok{canActivate}\NormalTok{()}\OperatorTok{:}\NormalTok{ boolean }\OperatorTok{|}\NormalTok{ UrlTree \{}
    \CommentTok{// 用于实现跳转}
    \ControlFlowTok{return} \KeywordTok{this}\OperatorTok{.}\AttributeTok{router}\OperatorTok{.}\FunctionTok{createUrlTree}\NormalTok{([}\StringTok{"/user/login"}\NormalTok{])}
    \CommentTok{// 禁止访问目标路由}
    \ControlFlowTok{return} \KeywordTok{false}
    \CommentTok{// 允许访问目标路由}
    \ControlFlowTok{return} \KeywordTok{true}
\NormalTok{  \}}
\NormalTok{\}}
\end{Highlighting}
\end{Shaded}

\begin{Shaded}
\begin{Highlighting}[]
\NormalTok{\{}
  \DataTypeTok{path}\OperatorTok{:} \StringTok{"about"}\OperatorTok{,}
  \DataTypeTok{component}\OperatorTok{:}\NormalTok{ AboutComponent}\OperatorTok{,}
  \DataTypeTok{canActivate}\OperatorTok{:}\NormalTok{ [AuthGuard]}
\NormalTok{\}}
\end{Highlighting}
\end{Shaded}

\hypertarget{12102-canactivatechild}{%
\subparagraph{12.10.2 CanActivateChild}\label{12102-canactivatechild}}

检查用户是否方可访问某个子路由。

创建路由守卫:\texttt{ng\ g\ guard\ guards/admin} 注意:选择
CanActivateChild,需要将箭头移动到这个选项并且敲击空格确认选择。

\begin{Shaded}
\begin{Highlighting}[]
\ImportTok{import}\NormalTok{ \{ Injectable \} }\ImportTok{from} \StringTok{"@angular/core"}
\ImportTok{import}\NormalTok{ \{ CanActivateChild}\OperatorTok{,}\NormalTok{ ActivatedRouteSnapshot}\OperatorTok{,}\NormalTok{ RouterStateSnapshot}\OperatorTok{,}\NormalTok{ UrlTree \} }\ImportTok{from} \StringTok{"@angular/router"}
\ImportTok{import}\NormalTok{ \{ Observable \} }\ImportTok{from} \StringTok{"rxjs"}

\NormalTok{@}\FunctionTok{Injectable}\NormalTok{(\{}
  \DataTypeTok{providedIn}\OperatorTok{:} \StringTok{"root"}
\NormalTok{\})}
\ImportTok{export} \KeywordTok{class}\NormalTok{ AdminGuard }\KeywordTok{implements}\NormalTok{ CanActivateChild \{}
  \FunctionTok{canActivateChild}\NormalTok{()}\OperatorTok{:}\NormalTok{ boolean }\OperatorTok{|}\NormalTok{ UrlTree \{}
    \ControlFlowTok{return} \KeywordTok{true}
\NormalTok{  \}}
\NormalTok{\}}
\end{Highlighting}
\end{Shaded}

\begin{Shaded}
\begin{Highlighting}[]
\NormalTok{\{}
  \DataTypeTok{path}\OperatorTok{:} \StringTok{"about"}\OperatorTok{,}
  \DataTypeTok{component}\OperatorTok{:}\NormalTok{ AboutComponent}\OperatorTok{,}
  \DataTypeTok{canActivateChild}\OperatorTok{:}\NormalTok{ [AdminGuard]}\OperatorTok{,}
  \DataTypeTok{children}\OperatorTok{:}\NormalTok{ [}
\NormalTok{    \{}
      \DataTypeTok{path}\OperatorTok{:} \StringTok{"introduce"}\OperatorTok{,}
      \DataTypeTok{component}\OperatorTok{:}\NormalTok{ IntroduceComponent}
\NormalTok{    \}}
\NormalTok{  ]}
\NormalTok{\}}
\end{Highlighting}
\end{Shaded}

\hypertarget{12103-candeactivate}{%
\subparagraph{12.10.3 CanDeactivate}\label{12103-candeactivate}}

检查用户是否可以退出路由。比如用户在表单中输入的内容没有保存,用户又要离开路由,此时可以调用该守卫提示用户。

\begin{Shaded}
\begin{Highlighting}[]
\ImportTok{import}\NormalTok{ \{ Injectable \} }\ImportTok{from} \StringTok{"@angular/core"}
\ImportTok{import}\NormalTok{ \{}
\NormalTok{  CanDeactivate}\OperatorTok{,}
\NormalTok{  ActivatedRouteSnapshot}\OperatorTok{,}
\NormalTok{  RouterStateSnapshot}\OperatorTok{,}
\NormalTok{  UrlTree}
\NormalTok{\} }\ImportTok{from} \StringTok{"@angular/router"}
\ImportTok{import}\NormalTok{ \{ Observable \} }\ImportTok{from} \StringTok{"rxjs"}

\ImportTok{export} \KeywordTok{interface}\NormalTok{ CanComponentLeave \{}
  \DataTypeTok{canLeave}\OperatorTok{:}\NormalTok{ () }\KeywordTok{=\textgreater{}}\NormalTok{ boolean}
\NormalTok{\}}

\NormalTok{@}\FunctionTok{Injectable}\NormalTok{(\{}
  \DataTypeTok{providedIn}\OperatorTok{:} \StringTok{"root"}
\NormalTok{\})}
\ImportTok{export} \KeywordTok{class}\NormalTok{ UnsaveGuard }\KeywordTok{implements}\NormalTok{ CanDeactivate}\OperatorTok{\textless{}}\NormalTok{CanComponentLeave}\OperatorTok{\textgreater{}}\NormalTok{ \{}
  \FunctionTok{canDeactivate}\NormalTok{(}\DataTypeTok{component}\OperatorTok{:}\NormalTok{ CanComponentLeave)}\OperatorTok{:}\NormalTok{ boolean \{}
    \ControlFlowTok{if}\NormalTok{ (component}\OperatorTok{.}\FunctionTok{canLeave}\NormalTok{()) \{}
      \ControlFlowTok{return} \KeywordTok{true}
\NormalTok{    \}}
    \ControlFlowTok{return} \KeywordTok{false}
\NormalTok{  \}}
\NormalTok{\}}
\end{Highlighting}
\end{Shaded}

\begin{Shaded}
\begin{Highlighting}[]
\NormalTok{\{}
  \DataTypeTok{path}\OperatorTok{:} \StringTok{""}\OperatorTok{,}
  \DataTypeTok{component}\OperatorTok{:}\NormalTok{ HomeComponent}\OperatorTok{,}
  \DataTypeTok{canDeactivate}\OperatorTok{:}\NormalTok{ [UnsaveGuard]}
\NormalTok{\}}
\end{Highlighting}
\end{Shaded}

\begin{Shaded}
\begin{Highlighting}[]
\ImportTok{import}\NormalTok{ \{ CanComponentLeave \} }\ImportTok{from} \StringTok{"src/app/guards/unsave.guard"}

\ImportTok{export} \KeywordTok{class}\NormalTok{ HomeComponent }\KeywordTok{implements}\NormalTok{ CanComponentLeave \{}
  \DataTypeTok{myForm}\OperatorTok{:}\NormalTok{ FormGroup }\OperatorTok{=} \KeywordTok{new} \FunctionTok{FormGroup}\NormalTok{(\{}
    \DataTypeTok{username}\OperatorTok{:} \KeywordTok{new} \FunctionTok{FormControl}\NormalTok{()}
\NormalTok{  \})}
  \FunctionTok{canLeave}\NormalTok{()}\OperatorTok{:}\NormalTok{ boolean \{}
    \ControlFlowTok{if}\NormalTok{ (}\KeywordTok{this}\OperatorTok{.}\AttributeTok{myForm}\OperatorTok{.}\AttributeTok{dirty}\NormalTok{) \{}
      \ControlFlowTok{if}\NormalTok{ (}\BuiltInTok{window}\OperatorTok{.}\FunctionTok{confirm}\NormalTok{(}\StringTok{"有数据未保存, 确定要离开吗"}\NormalTok{)) \{}
        \ControlFlowTok{return} \KeywordTok{true}
\NormalTok{      \} }\ControlFlowTok{else}\NormalTok{ \{}
        \ControlFlowTok{return} \KeywordTok{false}
\NormalTok{      \}}
\NormalTok{    \}}
    \ControlFlowTok{return} \KeywordTok{true}
\NormalTok{  \}}
\end{Highlighting}
\end{Shaded}

\hypertarget{12104-resolve}{%
\subparagraph{12.10.4 Resolve}\label{12104-resolve}}

允许在进入路由之前先获取数据,待数据获取完成之后再进入路由。

\texttt{ng\ g\ resolver\ \textless{}name\textgreater{}}

\begin{Shaded}
\begin{Highlighting}[]
\ImportTok{import}\NormalTok{ \{ Injectable \} }\ImportTok{from} \StringTok{"@angular/core"}
\ImportTok{import}\NormalTok{ \{ Resolve \} }\ImportTok{from} \StringTok{"@angular/router"}

\NormalTok{type returnType }\OperatorTok{=} \BuiltInTok{Promise}\OperatorTok{\textless{}}\NormalTok{\{ }\DataTypeTok{name}\OperatorTok{:}\NormalTok{ string \}}\OperatorTok{\textgreater{}}

\NormalTok{@}\FunctionTok{Injectable}\NormalTok{(\{}
  \DataTypeTok{providedIn}\OperatorTok{:} \StringTok{"root"}
\NormalTok{\})}
\ImportTok{export} \KeywordTok{class}\NormalTok{ ResolveGuard }\KeywordTok{implements}\NormalTok{ Resolve}\OperatorTok{\textless{}}\NormalTok{returnType}\OperatorTok{\textgreater{}}\NormalTok{ \{}
  \FunctionTok{resolve}\NormalTok{()}\OperatorTok{:}\NormalTok{ returnType \{}
    \ControlFlowTok{return} \KeywordTok{new} \BuiltInTok{Promise}\NormalTok{(}\KeywordTok{function}\NormalTok{ (resolve) \{}
      \PreprocessorTok{setTimeout}\NormalTok{(() }\KeywordTok{=\textgreater{}}\NormalTok{ \{}
        \FunctionTok{resolve}\NormalTok{(\{ }\DataTypeTok{name}\OperatorTok{:} \StringTok{"张三"}\NormalTok{ \})}
\NormalTok{      \}}\OperatorTok{,} \DecValTok{2000}\NormalTok{)}
\NormalTok{    \})}
\NormalTok{  \}}
\NormalTok{\}}
\end{Highlighting}
\end{Shaded}

\begin{Shaded}
\begin{Highlighting}[]
\NormalTok{\{}
   \DataTypeTok{path}\OperatorTok{:} \StringTok{""}\OperatorTok{,}
   \DataTypeTok{component}\OperatorTok{:}\NormalTok{ HomeComponent}\OperatorTok{,}
   \DataTypeTok{resolve}\OperatorTok{:}\NormalTok{ \{}
     \DataTypeTok{user}\OperatorTok{:}\NormalTok{ ResolveGuard}
\NormalTok{   \}}
\NormalTok{\}}
\end{Highlighting}
\end{Shaded}

\begin{Shaded}
\begin{Highlighting}[]
\ImportTok{export} \KeywordTok{class}\NormalTok{ HomeComponent \{}
  \FunctionTok{constructor}\NormalTok{(}\KeywordTok{private} \DataTypeTok{route}\OperatorTok{:}\NormalTok{ ActivatedRoute) \{\}}
  \FunctionTok{ngOnInit}\NormalTok{()}\OperatorTok{:} \KeywordTok{void}\NormalTok{ \{}
    \BuiltInTok{console}\OperatorTok{.}\FunctionTok{log}\NormalTok{(}\KeywordTok{this}\OperatorTok{.}\AttributeTok{route}\OperatorTok{.}\AttributeTok{snapshot}\OperatorTok{.}\AttributeTok{data}\OperatorTok{.}\AttributeTok{user}\NormalTok{)}
\NormalTok{  \}}
\NormalTok{\}}
\end{Highlighting}
\end{Shaded}

\hypertarget{13-rxjs}{%
\subsubsection{\texorpdfstring{13.
\href{https://rxjs.dev/}{RxJS}}{13. RxJS}}\label{13-rxjs}}

\hypertarget{131-ux6982ux8ff0}{%
\paragraph{13.1 概述}\label{131-ux6982ux8ff0}}

\begin{figure}
\centering
\includegraphics{C:/Users/ZSH/Desktop/ng/ppt/images/65.png}
\caption{}
\end{figure}

\hypertarget{1311-ux4ec0ux4e48ux662f-rxjs-}{%
\subparagraph{13.1.1 什么是 RxJS
?}\label{1311-ux4ec0ux4e48ux662f-rxjs-}}

RxJS 是一个用于处理异步编程的 JavaScript
库,目标是使编写异步和基于回调的代码更容易。

\hypertarget{1312-ux4e3aux4ec0ux4e48ux8981ux5b66ux4e60-rxjs-}{%
\subparagraph{13.1.2 为什么要学习 RxJS
?}\label{1312-ux4e3aux4ec0ux4e48ux8981ux5b66ux4e60-rxjs-}}

就像 Angular 深度集成 TypeScript 一样,Angular 也深度集成了 RxJS。

服务、表单、事件、全局状态管理、异步请求 ...

\hypertarget{1313-ux5febux901fux5165ux95e8}{%
\subparagraph{13.1.3 快速入门}\label{1313-ux5febux901fux5165ux95e8}}

\begin{enumerate}
\def\labelenumi{\arabic{enumi}.}
\item
  可观察对象 ( Observable ) :类比 Promise
  对象,内部可以用于执行异步代码,通过调用内部提供的方法将异步代码执行的结果传递到可观察对象外部。
\item
  观察者 ( Observer ):类比 then
  方法中的回调函数,用于接收可观察对象中传递出来数据。
\item
  订阅 ( subscribe ):类比 then
  方法,通过订阅将可观察对象和观察者连接起来,当可观察对象发出数据时,订阅者可以接收到数据。

  \begin{figure}
  \centering
  \includegraphics{C:/Users/ZSH/Desktop/ng/ppt/images/63.png}
  \caption{}
  \end{figure}

\begin{Shaded}
\begin{Highlighting}[]
\ImportTok{import}\NormalTok{ \{ Observable \} from }\StringTok{"rxjs"}

\NormalTok{const observable }\OperatorTok{=} \KeywordTok{new} \FunctionTok{Observable}\NormalTok{(}\KeywordTok{function}\NormalTok{ (observer) \{}
  \FunctionTok{setTimeout}\NormalTok{(}\KeywordTok{function}\NormalTok{ () \{}
\NormalTok{    observer}\OperatorTok{.}\FunctionTok{next}\NormalTok{(\{}
\NormalTok{      name}\OperatorTok{:} \StringTok{"张三"}
\NormalTok{    \})}
\NormalTok{  \}}\OperatorTok{,} \DecValTok{2000}\NormalTok{)}
\NormalTok{\})}

\NormalTok{const observer }\OperatorTok{=}\NormalTok{ \{}
\NormalTok{  next}\OperatorTok{:} \KeywordTok{function}\NormalTok{ (value) \{}
    \BuiltInTok{console}\OperatorTok{.}\FunctionTok{log}\NormalTok{(value)}
\NormalTok{  \}}
\NormalTok{\}}

\NormalTok{observable}\OperatorTok{.}\FunctionTok{subscribe}\NormalTok{(observer)}
\end{Highlighting}
\end{Shaded}
\end{enumerate}

\hypertarget{132-ux53efux89c2ux5bdfux5bf9ux8c61}{%
\paragraph{13.2 可观察对象}\label{132-ux53efux89c2ux5bdfux5bf9ux8c61}}

\hypertarget{1321-observable}{%
\subparagraph{13.2.1 Observable}\label{1321-observable}}

\begin{enumerate}
\def\labelenumi{\arabic{enumi}.}
\item
  在 Observable 对象内部可以多次调用 next 方法向外发送数据。

\begin{Shaded}
\begin{Highlighting}[]
\KeywordTok{const}\NormalTok{ observable }\OperatorTok{=} \KeywordTok{new} \FunctionTok{Observable}\NormalTok{(}\KeywordTok{function}\NormalTok{ (observer) \{}
  \KeywordTok{let}\NormalTok{ index }\OperatorTok{=} \DecValTok{0}
  \PreprocessorTok{setInterval}\NormalTok{(}\KeywordTok{function}\NormalTok{ () \{}
\NormalTok{    observer}\OperatorTok{.}\FunctionTok{next}\NormalTok{(index}\OperatorTok{++}\NormalTok{)}
\NormalTok{  \}}\OperatorTok{,} \DecValTok{1000}\NormalTok{)}
\NormalTok{\})}

\KeywordTok{const}\NormalTok{ observer }\OperatorTok{=}\NormalTok{ \{}
  \DataTypeTok{next}\OperatorTok{:} \KeywordTok{function}\NormalTok{ (value) \{}
    \BuiltInTok{console}\OperatorTok{.}\FunctionTok{log}\NormalTok{(value)}
\NormalTok{  \}}
\NormalTok{\}}

\NormalTok{observable}\OperatorTok{.}\FunctionTok{subscribe}\NormalTok{(observer)}
\end{Highlighting}
\end{Shaded}
\item
  当所有数据发送完成以后,可以调用 complete 方法终止数据发送。

\begin{Shaded}
\begin{Highlighting}[]
\KeywordTok{const}\NormalTok{ observable }\OperatorTok{=} \KeywordTok{new} \FunctionTok{Observable}\NormalTok{(}\KeywordTok{function}\NormalTok{ (observer) \{}
  \KeywordTok{let}\NormalTok{ index }\OperatorTok{=} \DecValTok{0}
  \KeywordTok{let}\NormalTok{ timer }\OperatorTok{=} \PreprocessorTok{setInterval}\NormalTok{(}\KeywordTok{function}\NormalTok{ () \{}
\NormalTok{    observer}\OperatorTok{.}\FunctionTok{next}\NormalTok{(index}\OperatorTok{++}\NormalTok{)}
    \ControlFlowTok{if}\NormalTok{ (index }\OperatorTok{===} \DecValTok{3}\NormalTok{) \{}
\NormalTok{      observer}\OperatorTok{.}\FunctionTok{complete}\NormalTok{()}
      \PreprocessorTok{clearInterval}\NormalTok{(timer)}
\NormalTok{    \}}
\NormalTok{  \}}\OperatorTok{,} \DecValTok{1000}\NormalTok{)}
\NormalTok{\})}

\KeywordTok{const}\NormalTok{ observer }\OperatorTok{=}\NormalTok{ \{}
  \DataTypeTok{next}\OperatorTok{:} \KeywordTok{function}\NormalTok{ (value) \{}
    \BuiltInTok{console}\OperatorTok{.}\FunctionTok{log}\NormalTok{(value)}
\NormalTok{  \}}\OperatorTok{,}
  \DataTypeTok{complete}\OperatorTok{:} \KeywordTok{function}\NormalTok{ () \{}
    \BuiltInTok{console}\OperatorTok{.}\FunctionTok{log}\NormalTok{(}\StringTok{"数据发送完成"}\NormalTok{)}
\NormalTok{  \}}
\NormalTok{\}}

\NormalTok{observable}\OperatorTok{.}\FunctionTok{subscribe}\NormalTok{(observer)}
\end{Highlighting}
\end{Shaded}
\item
  当 Observable 内部逻辑发生错误时,可以调用 error
  方法将失败信息发送给订阅者,Observable 终止。

\begin{Shaded}
\begin{Highlighting}[]
\ImportTok{import}\NormalTok{ \{ Observable \} }\ImportTok{from} \StringTok{"rxjs"}

\KeywordTok{const}\NormalTok{ observable }\OperatorTok{=} \KeywordTok{new} \FunctionTok{Observable}\NormalTok{(}\KeywordTok{function}\NormalTok{ (observer) \{}
  \KeywordTok{let}\NormalTok{ index }\OperatorTok{=} \DecValTok{0}
  \KeywordTok{let}\NormalTok{ timer }\OperatorTok{=} \PreprocessorTok{setInterval}\NormalTok{(}\KeywordTok{function}\NormalTok{ () \{}
\NormalTok{    observer}\OperatorTok{.}\FunctionTok{next}\NormalTok{(index}\OperatorTok{++}\NormalTok{)}
    \ControlFlowTok{if}\NormalTok{ (index }\OperatorTok{===} \DecValTok{3}\NormalTok{) \{}
\NormalTok{      observer}\OperatorTok{.}\FunctionTok{error}\NormalTok{(}\StringTok{"发生错误"}\NormalTok{)}
      \PreprocessorTok{clearInterval}\NormalTok{(timer)}
\NormalTok{    \}}
\NormalTok{  \}}\OperatorTok{,} \DecValTok{1000}\NormalTok{)}
\NormalTok{\})}

\KeywordTok{const}\NormalTok{ observer }\OperatorTok{=}\NormalTok{ \{}
  \DataTypeTok{next}\OperatorTok{:} \KeywordTok{function}\NormalTok{ (value) \{}
    \BuiltInTok{console}\OperatorTok{.}\FunctionTok{log}\NormalTok{(value)}
\NormalTok{  \}}\OperatorTok{,}
  \DataTypeTok{error}\OperatorTok{:} \KeywordTok{function}\NormalTok{ (error) \{}
    \BuiltInTok{console}\OperatorTok{.}\FunctionTok{log}\NormalTok{(error)}
\NormalTok{  \}}
\NormalTok{\}}

\NormalTok{observable}\OperatorTok{.}\FunctionTok{subscribe}\NormalTok{(observer)}
\end{Highlighting}
\end{Shaded}
\item
  可观察对象是惰性的,只有被订阅后才会执行

\begin{Shaded}
\begin{Highlighting}[]
\KeywordTok{const}\NormalTok{ observable }\OperatorTok{=} \KeywordTok{new} \FunctionTok{Observable}\NormalTok{(}\KeywordTok{function}\NormalTok{ () \{}
  \BuiltInTok{console}\OperatorTok{.}\FunctionTok{log}\NormalTok{(}\StringTok{"Hello RxJS"}\NormalTok{)}
\NormalTok{\})}
\CommentTok{// observable.subscribe()}
\end{Highlighting}
\end{Shaded}
\item
  可观察对象可以有 n 多订阅者,每次被订阅时都会得到执行

  \begin{figure}
  \centering
  \includegraphics{C:/Users/ZSH/Desktop/ng/ppt/images/64.png}
  \caption{}
  \end{figure}

\begin{Shaded}
\begin{Highlighting}[]
\KeywordTok{const}\NormalTok{ observable }\OperatorTok{=} \KeywordTok{new} \FunctionTok{Observable}\NormalTok{(}\KeywordTok{function}\NormalTok{ () \{}
  \BuiltInTok{console}\OperatorTok{.}\FunctionTok{log}\NormalTok{(}\StringTok{"Hello RxJS"}\NormalTok{)}
\NormalTok{\})}

\NormalTok{observable}\OperatorTok{.}\FunctionTok{subscribe}\NormalTok{()}
\NormalTok{observable}\OperatorTok{.}\FunctionTok{subscribe}\NormalTok{()}
\NormalTok{observable}\OperatorTok{.}\FunctionTok{subscribe}\NormalTok{()}
\NormalTok{observable}\OperatorTok{.}\FunctionTok{subscribe}\NormalTok{()}
\NormalTok{observable}\OperatorTok{.}\FunctionTok{subscribe}\NormalTok{()}
\end{Highlighting}
\end{Shaded}
\item
  取消订阅

\begin{Shaded}
\begin{Highlighting}[]
\ImportTok{import}\NormalTok{ \{ interval \} }\ImportTok{from} \StringTok{"rxjs"}

\KeywordTok{const}\NormalTok{ obs }\OperatorTok{=} \FunctionTok{interval}\NormalTok{(}\DecValTok{1000}\NormalTok{)}
\KeywordTok{const}\NormalTok{ subscription }\OperatorTok{=}\NormalTok{ obs}\OperatorTok{.}\FunctionTok{subscribe}\NormalTok{(}\BuiltInTok{console}\OperatorTok{.}\FunctionTok{log}\NormalTok{)}

\PreprocessorTok{setTimeout}\NormalTok{(}\KeywordTok{function}\NormalTok{ () \{}
\NormalTok{  subscription}\OperatorTok{.}\FunctionTok{unsubscribe}\NormalTok{()}
\NormalTok{\}}\OperatorTok{,} \DecValTok{2000}\NormalTok{)}
\end{Highlighting}
\end{Shaded}
\end{enumerate}

\hypertarget{1322-subject}{%
\subparagraph{13.2.2 Subject}\label{1322-subject}}

\begin{enumerate}
\def\labelenumi{\arabic{enumi}.}
\item
  用于创建空的可观察对象,在订阅后不会立即执行,next
  方法可以在可观察对象外部调用

\begin{Shaded}
\begin{Highlighting}[]
\ImportTok{import}\NormalTok{ \{ Subject \} }\ImportTok{from} \StringTok{"rxjs"}

\KeywordTok{const}\NormalTok{ demoSubject }\OperatorTok{=} \KeywordTok{new} \FunctionTok{Subject}\NormalTok{()}

\NormalTok{demoSubject}\OperatorTok{.}\FunctionTok{subscribe}\NormalTok{(\{}\DataTypeTok{next}\OperatorTok{:} \KeywordTok{function}\NormalTok{ (value) \{}\BuiltInTok{console}\OperatorTok{.}\FunctionTok{log}\NormalTok{(value)\}\})}
\NormalTok{demoSubject}\OperatorTok{.}\FunctionTok{subscribe}\NormalTok{(\{}\DataTypeTok{next}\OperatorTok{:} \KeywordTok{function}\NormalTok{ (value) \{}\BuiltInTok{console}\OperatorTok{.}\FunctionTok{log}\NormalTok{(value)\}\})}

\PreprocessorTok{setTimeout}\NormalTok{(}\KeywordTok{function}\NormalTok{ () \{}
\NormalTok{  demoSubject}\OperatorTok{.}\FunctionTok{next}\NormalTok{(}\StringTok{"hahaha"}\NormalTok{)}
\NormalTok{\}}\OperatorTok{,} \DecValTok{3000}\NormalTok{)}
\end{Highlighting}
\end{Shaded}
\end{enumerate}

\hypertarget{1323-behaviorsubject}{%
\subparagraph{13.2.3 BehaviorSubject}\label{1323-behaviorsubject}}

拥有 Subject 全部功能,但是在创建 Obervable
对象时可以传入默认值,观察者订阅后可以直接拿到默认值。

\begin{Shaded}
\begin{Highlighting}[]
\ImportTok{import}\NormalTok{ \{ BehaviorSubject \} }\ImportTok{from} \StringTok{"rxjs"}

\KeywordTok{const}\NormalTok{ demoBehavior }\OperatorTok{=} \KeywordTok{new} \FunctionTok{BehaviorSubject}\NormalTok{(}\StringTok{"默认值"}\NormalTok{)}
\NormalTok{demoBehavior}\OperatorTok{.}\FunctionTok{subscribe}\NormalTok{(\{}\DataTypeTok{next}\OperatorTok{:} \KeywordTok{function}\NormalTok{ (value) \{}\BuiltInTok{console}\OperatorTok{.}\FunctionTok{log}\NormalTok{(value)\}\})}
\NormalTok{demoBehavior}\OperatorTok{.}\FunctionTok{next}\NormalTok{(}\StringTok{"Hello"}\NormalTok{)}
\end{Highlighting}
\end{Shaded}

\hypertarget{1323-replaysubject}{%
\subparagraph{13.2.3 ReplaySubject}\label{1323-replaysubject}}

功能类似 Subject,但有新订阅者时两者处理方式不同,Subject
不会广播历史结果,而 ReplaySubject 会广播所有历史结果。

\begin{Shaded}
\begin{Highlighting}[]
\ImportTok{import}\NormalTok{ \{ ReplaySubject \} }\ImportTok{from} \StringTok{"rxjs"}

\KeywordTok{const}\NormalTok{ rSubject }\OperatorTok{=} \KeywordTok{new} \FunctionTok{ReplaySubject}\NormalTok{()}

\NormalTok{rSubject}\OperatorTok{.}\FunctionTok{subscribe}\NormalTok{(value }\KeywordTok{=\textgreater{}}\NormalTok{ \{}
  \BuiltInTok{console}\OperatorTok{.}\FunctionTok{log}\NormalTok{(value)}
\NormalTok{\})}

\NormalTok{rSubject}\OperatorTok{.}\FunctionTok{next}\NormalTok{(}\StringTok{"Hello 1"}\NormalTok{)}
\NormalTok{rSubject}\OperatorTok{.}\FunctionTok{next}\NormalTok{(}\StringTok{"Hello 2"}\NormalTok{)}

\PreprocessorTok{setTimeout}\NormalTok{(}\KeywordTok{function}\NormalTok{ () \{}
\NormalTok{  rSubject}\OperatorTok{.}\FunctionTok{subscribe}\NormalTok{(\{}\DataTypeTok{next}\OperatorTok{:} \KeywordTok{function}\NormalTok{ (value) \{}\BuiltInTok{console}\OperatorTok{.}\FunctionTok{log}\NormalTok{(value)\}\})}
\NormalTok{\}}\OperatorTok{,} \DecValTok{3000}\NormalTok{)}
\end{Highlighting}
\end{Shaded}

\hypertarget{133-ux8f85ux52a9ux65b9ux6cd5}{%
\paragraph{13.3 辅助方法}\label{133-ux8f85ux52a9ux65b9ux6cd5}}

\hypertarget{1331-range}{%
\subparagraph{13.3.1 range}\label{1331-range}}

range(start, length),调用方法后返回 observable
对象,被订阅后会发出指定范围的数值。

\begin{figure}
\centering
\includegraphics{C:/Users/ZSH/Desktop/ng/ppt/images/9.png}
\caption{}
\end{figure}

\begin{Shaded}
\begin{Highlighting}[]
\ImportTok{import}\NormalTok{ \{ range \} }\ImportTok{from} \StringTok{"rxjs"}

\FunctionTok{range}\NormalTok{(}\DecValTok{0}\OperatorTok{,} \DecValTok{5}\NormalTok{)}\OperatorTok{.}\FunctionTok{subscribe}\NormalTok{(n }\KeywordTok{=\textgreater{}} \BuiltInTok{console}\OperatorTok{.}\FunctionTok{log}\NormalTok{(n))}

\CommentTok{// 0}
\CommentTok{// 1}
\CommentTok{// 2}
\CommentTok{// 3}
\CommentTok{// 4}
\end{Highlighting}
\end{Shaded}

方法内部并不是一次发出 length 个数值,而是发送了 length
次,每次发送一个数值,就是说内部调用了 length 次 next 方法。

\hypertarget{1332-of}{%
\subparagraph{13.3.2 of}\label{1332-of}}

将参数列表作为数据流返回。

\begin{figure}
\centering
\includegraphics{C:/Users/ZSH/Desktop/ng/ppt/images/5.png}
\caption{}
\end{figure}

\begin{Shaded}
\begin{Highlighting}[]
\KeywordTok{of}\NormalTok{(}\StringTok{"a"}\OperatorTok{,} \StringTok{"b"}\OperatorTok{,}\NormalTok{ []}\OperatorTok{,}\NormalTok{ \{\}}\OperatorTok{,} \KeywordTok{true}\OperatorTok{,} \DecValTok{20}\NormalTok{)}\OperatorTok{.}\FunctionTok{subscribe}\NormalTok{(v }\KeywordTok{=\textgreater{}} \BuiltInTok{console}\OperatorTok{.}\FunctionTok{log}\NormalTok{(v))}
\end{Highlighting}
\end{Shaded}

\hypertarget{1333-from}{%
\subparagraph{13.3.3 from}\label{1333-from}}

将 Array,Promise, Iterator 转换为 observable 对象。

\begin{figure}
\centering
\includegraphics{C:/Users/ZSH/Desktop/ng/ppt/images/10.png}
\caption{}
\end{figure}

\begin{Shaded}
\begin{Highlighting}[]
\ImportTok{from}\NormalTok{([}\StringTok{"a"}\OperatorTok{,} \StringTok{"b"}\OperatorTok{,} \StringTok{"c"}\NormalTok{])}\OperatorTok{.}\FunctionTok{subscribe}\NormalTok{(v }\KeywordTok{=\textgreater{}} \BuiltInTok{console}\OperatorTok{.}\FunctionTok{log}\NormalTok{(v))}
\CommentTok{// a}
\CommentTok{// b}
\CommentTok{// c}
\end{Highlighting}
\end{Shaded}

\begin{Shaded}
\begin{Highlighting}[]
\ImportTok{import}\NormalTok{ \{ }\ImportTok{from}\NormalTok{ \} }\ImportTok{from} \StringTok{"rxjs"}

\KeywordTok{function} \FunctionTok{p}\NormalTok{() \{}
  \ControlFlowTok{return} \KeywordTok{new} \BuiltInTok{Promise}\NormalTok{(}\KeywordTok{function}\NormalTok{ (resolve) \{}
    \FunctionTok{resolve}\NormalTok{([}\DecValTok{100}\OperatorTok{,} \DecValTok{200}\NormalTok{])}
\NormalTok{  \})}
\NormalTok{\}}

\ImportTok{from}\NormalTok{(}\FunctionTok{p}\NormalTok{())}\OperatorTok{.}\FunctionTok{subscribe}\NormalTok{(v }\KeywordTok{=\textgreater{}} \BuiltInTok{console}\OperatorTok{.}\FunctionTok{log}\NormalTok{(v))}
\CommentTok{// [100, 200]}
\end{Highlighting}
\end{Shaded}

\hypertarget{1334-intervaltimer}{%
\subparagraph{13.3.4 interval、timer}\label{1334-intervaltimer}}

\textbf{Interval:}每隔一段时间发出一个数值,数值递增

\begin{figure}
\centering
\includegraphics{C:/Users/ZSH/Desktop/ng/ppt/images/11.png}
\caption{}
\end{figure}

\begin{Shaded}
\begin{Highlighting}[]
\ImportTok{import}\NormalTok{ \{ interval \} }\ImportTok{from} \StringTok{"rxjs"}

\FunctionTok{interval}\NormalTok{(}\DecValTok{1000}\NormalTok{)}\OperatorTok{.}\FunctionTok{subscribe}\NormalTok{(n }\KeywordTok{=\textgreater{}} \BuiltInTok{console}\OperatorTok{.}\FunctionTok{log}\NormalTok{(n))}
\end{Highlighting}
\end{Shaded}

\textbf{timer:}间隔时间过去以后发出数值,行为终止,或间隔时间发出数值后,继续按第二个参数的时间间隔继续发出值

\begin{figure}
\centering
\includegraphics{C:/Users/ZSH/Desktop/ng/ppt/images/12.png}
\caption{}
\end{figure}

\begin{Shaded}
\begin{Highlighting}[]
\ImportTok{import}\NormalTok{ \{ timer \} }\ImportTok{from} \StringTok{"rxjs"}

\FunctionTok{timer}\NormalTok{(}\DecValTok{2000}\NormalTok{)}\OperatorTok{.}\FunctionTok{subscribe}\NormalTok{(n }\KeywordTok{=\textgreater{}} \BuiltInTok{console}\OperatorTok{.}\FunctionTok{log}\NormalTok{(n))}
\FunctionTok{timer}\NormalTok{(}\DecValTok{0}\OperatorTok{,} \DecValTok{1000}\NormalTok{)}\OperatorTok{.}\FunctionTok{subscribe}\NormalTok{(n }\KeywordTok{=\textgreater{}} \BuiltInTok{console}\OperatorTok{.}\FunctionTok{log}\NormalTok{(n))}
\end{Highlighting}
\end{Shaded}

\hypertarget{1335-concat}{%
\subparagraph{13.3.5 concat}\label{1335-concat}}

合并数据流,先让第一个数据流发出值,结束后再让第二个数据流发出值,进行整体合并。

\begin{figure}
\centering
\includegraphics{C:/Users/ZSH/Desktop/ng/ppt/images/8.png}
\caption{}
\end{figure}

\begin{Shaded}
\begin{Highlighting}[]
\ImportTok{import}\NormalTok{ \{ concat}\OperatorTok{,}\NormalTok{ range \} }\ImportTok{from} \StringTok{"rxjs"}

\FunctionTok{concat}\NormalTok{(}\FunctionTok{range}\NormalTok{(}\DecValTok{1}\OperatorTok{,} \DecValTok{5}\NormalTok{)}\OperatorTok{,} \FunctionTok{range}\NormalTok{(}\DecValTok{6}\OperatorTok{,} \DecValTok{5}\NormalTok{))}\OperatorTok{.}\FunctionTok{subscribe}\NormalTok{(}\BuiltInTok{console}\OperatorTok{.}\FunctionTok{log}\NormalTok{)}
\end{Highlighting}
\end{Shaded}

\hypertarget{1336-merge}{%
\subparagraph{13.3.6 merge}\label{1336-merge}}

合并数据流,多个参数一起发出数据流,按照时间线进行交叉合并。

\begin{figure}
\centering
\includegraphics{C:/Users/ZSH/Desktop/ng/ppt/images/33.png}
\caption{}
\end{figure}

\begin{Shaded}
\begin{Highlighting}[]
\ImportTok{import}\NormalTok{ \{ merge}\OperatorTok{,}\NormalTok{ fromEvent}\OperatorTok{,}\NormalTok{ interval \} }\ImportTok{from} \StringTok{"rxjs"}

\KeywordTok{const}\NormalTok{ clicks }\OperatorTok{=} \FunctionTok{fromEvent}\NormalTok{(}\BuiltInTok{document}\OperatorTok{,} \StringTok{"click"}\NormalTok{)}
\KeywordTok{const}\NormalTok{ timer }\OperatorTok{=} \FunctionTok{interval}\NormalTok{(}\DecValTok{1000}\NormalTok{)}

\FunctionTok{merge}\NormalTok{(clicks}\OperatorTok{,}\NormalTok{ timer)}\OperatorTok{.}\FunctionTok{subscribe}\NormalTok{(}\BuiltInTok{console}\OperatorTok{.}\FunctionTok{log}\NormalTok{)}
\end{Highlighting}
\end{Shaded}

\hypertarget{1337-combinelatest}{%
\subparagraph{13.3.7 combineLatest}\label{1337-combinelatest}}

将两个 Obserable
中最新发出的数据流进行组合成新的数据流,以数组的形式发出。和当前最新的进行组合。

\begin{figure}
\centering
\includegraphics{C:/Users/ZSH/Desktop/ng/ppt/images/40.png}
\caption{}
\end{figure}

\begin{Shaded}
\begin{Highlighting}[]
\ImportTok{import}\NormalTok{ \{ combineLatest}\OperatorTok{,}\NormalTok{ timer \} }\ImportTok{from} \StringTok{"rxjs"}

\KeywordTok{const}\NormalTok{ firstTimer }\OperatorTok{=} \FunctionTok{timer}\NormalTok{(}\DecValTok{0}\OperatorTok{,} \DecValTok{1000}\NormalTok{) }\CommentTok{// emit 0, 1, 2... after every second, starting from now}
\KeywordTok{const}\NormalTok{ secondTimer }\OperatorTok{=} \FunctionTok{timer}\NormalTok{(}\DecValTok{500}\OperatorTok{,} \DecValTok{1000}\NormalTok{) }\CommentTok{// emit 0, 1, 2... after every second, starting 0,5s from now}
\FunctionTok{combineLatest}\NormalTok{(firstTimer}\OperatorTok{,}\NormalTok{ secondTimer)}\OperatorTok{.}\FunctionTok{subscribe}\NormalTok{(}\BuiltInTok{console}\OperatorTok{.}\FunctionTok{log}\NormalTok{)}

\CommentTok{// [0, 0] after 0.5s}
\CommentTok{// [1, 0] after 1s}
\CommentTok{// [1, 1] after 1.5s}
\CommentTok{// [2, 1] after 2s}
\end{Highlighting}
\end{Shaded}

\hypertarget{1338-zip}{%
\subparagraph{13.3.8 zip}\label{1338-zip}}

将多个 Observable 中的数据流进行组合。和将来最新的进行组合。

\begin{figure}
\centering
\includegraphics{C:/Users/ZSH/Desktop/ng/ppt/images/39.png}
\caption{}
\end{figure}

\begin{Shaded}
\begin{Highlighting}[]
\ImportTok{import}\NormalTok{ \{ zip}\OperatorTok{,} \KeywordTok{of}\NormalTok{ \} }\ImportTok{from} \StringTok{"rxjs"}
\ImportTok{import}\NormalTok{ \{ map \} }\ImportTok{from} \StringTok{"rxjs/operators"}

\KeywordTok{let}\NormalTok{ age }\OperatorTok{=} \KeywordTok{of}\NormalTok{(}\DecValTok{27}\OperatorTok{,} \DecValTok{25}\OperatorTok{,} \DecValTok{29}\NormalTok{)}
\KeywordTok{let}\NormalTok{ name }\OperatorTok{=} \KeywordTok{of}\NormalTok{(}\StringTok{"Foo"}\OperatorTok{,} \StringTok{"Bar"}\OperatorTok{,} \StringTok{"Beer"}\NormalTok{)}
\KeywordTok{let}\NormalTok{ isDev }\OperatorTok{=} \KeywordTok{of}\NormalTok{(}\KeywordTok{true}\OperatorTok{,} \KeywordTok{true}\OperatorTok{,} \KeywordTok{false}\NormalTok{)}

\FunctionTok{zip}\NormalTok{(name}\OperatorTok{,}\NormalTok{ age}\OperatorTok{,}\NormalTok{ isDev)}
  \OperatorTok{.}\FunctionTok{pipe}\NormalTok{(}\FunctionTok{map}\NormalTok{(([name}\OperatorTok{,}\NormalTok{ age}\OperatorTok{,}\NormalTok{ isDev]) }\KeywordTok{=\textgreater{}}\NormalTok{ (\{ name}\OperatorTok{,}\NormalTok{ age}\OperatorTok{,}\NormalTok{ isDev \})))}
  \OperatorTok{.}\FunctionTok{subscribe}\NormalTok{(}\BuiltInTok{console}\OperatorTok{.}\FunctionTok{log}\NormalTok{)}

\CommentTok{// \{ name: \textquotesingle{}Foo\textquotesingle{}, age: 27, isDev: true \}}
\CommentTok{// \{ name: \textquotesingle{}Bar\textquotesingle{}, age: 25, isDev: true \}}
\CommentTok{// \{ name: \textquotesingle{}Beer\textquotesingle{}, age: 29, isDev: false \}}
\end{Highlighting}
\end{Shaded}

\hypertarget{1339-forkjoin}{%
\subparagraph{13.3.9 forkJoin}\label{1339-forkjoin}}

forkJoin 是 Rx 版本的 Promise.all(),即表示等到所有的 Observable
都完成后,才一次性返回值。

\begin{figure}
\centering
\includegraphics{C:/Users/ZSH/Desktop/ng/ppt/images/41.png}
\caption{}
\end{figure}

\begin{Shaded}
\begin{Highlighting}[]
\ImportTok{import}\NormalTok{ axios }\ImportTok{from} \StringTok{"axios"}
\ImportTok{import}\NormalTok{ \{ forkJoin}\OperatorTok{,} \ImportTok{from}\NormalTok{ \} }\ImportTok{from} \StringTok{"rxjs"}

\NormalTok{axios}\OperatorTok{.}\AttributeTok{interceptors}\OperatorTok{.}\AttributeTok{response}\OperatorTok{.}\FunctionTok{use}\NormalTok{(response }\KeywordTok{=\textgreater{}}\NormalTok{ response}\OperatorTok{.}\AttributeTok{data}\NormalTok{)}

\FunctionTok{forkJoin}\NormalTok{(\{}
  \DataTypeTok{goods}\OperatorTok{:} \ImportTok{from}\NormalTok{(axios}\OperatorTok{.}\FunctionTok{get}\NormalTok{(}\StringTok{"http://localhost:3005/goods"}\NormalTok{))}\OperatorTok{,}
  \DataTypeTok{category}\OperatorTok{:} \ImportTok{from}\NormalTok{(axios}\OperatorTok{.}\FunctionTok{get}\NormalTok{(}\StringTok{"http://localhost:3005/category"}\NormalTok{))}
\NormalTok{\})}\OperatorTok{.}\FunctionTok{subscribe}\NormalTok{(}\BuiltInTok{console}\OperatorTok{.}\FunctionTok{log}\NormalTok{)}
\end{Highlighting}
\end{Shaded}

\hypertarget{13310-throwerror}{%
\subparagraph{13.3.10 throwError}\label{13310-throwerror}}

返回可观察对象并向订阅者抛出错误。

\begin{figure}
\centering
\includegraphics{C:/Users/ZSH/Desktop/ng/ppt/images/42.png}
\caption{}
\end{figure}

\begin{Shaded}
\begin{Highlighting}[]
\ImportTok{import}\NormalTok{ \{ throwError \} }\ImportTok{from} \StringTok{"rxjs"}

\FunctionTok{throwError}\NormalTok{(}\StringTok{"发生了未知错误"}\NormalTok{)}\OperatorTok{.}\FunctionTok{subscribe}\NormalTok{(\{ }\DataTypeTok{error}\OperatorTok{:} \BuiltInTok{console}\OperatorTok{.}\FunctionTok{log}\NormalTok{ \})}
\end{Highlighting}
\end{Shaded}

\hypertarget{13311-retry}{%
\subparagraph{13.3.11 retry}\label{13311-retry}}

如果 Observable 对象抛出错误,则该辅助方法会重新订阅 Observable
以获取数据流,参数为重新订阅次数。

\begin{figure}
\centering
\includegraphics{C:/Users/ZSH/Desktop/ng/ppt/images/43.png}
\caption{}
\end{figure}

\begin{Shaded}
\begin{Highlighting}[]
\ImportTok{import}\NormalTok{ \{ interval}\OperatorTok{,} \KeywordTok{of}\OperatorTok{,}\NormalTok{ throwError \} }\ImportTok{from} \StringTok{"rxjs"}
\ImportTok{import}\NormalTok{ \{ mergeMap}\OperatorTok{,}\NormalTok{ retry \} }\ImportTok{from} \StringTok{"rxjs/operators"}

\FunctionTok{interval}\NormalTok{(}\DecValTok{1000}\NormalTok{)}
  \OperatorTok{.}\FunctionTok{pipe}\NormalTok{(}
    \FunctionTok{mergeMap}\NormalTok{(val }\KeywordTok{=\textgreater{}}\NormalTok{ \{}
      \ControlFlowTok{if}\NormalTok{ (val }\OperatorTok{\textgreater{}} \DecValTok{2}\NormalTok{) \{}
        \ControlFlowTok{return} \FunctionTok{throwError}\NormalTok{(}\StringTok{"Error!"}\NormalTok{)}
\NormalTok{      \}}
      \ControlFlowTok{return} \KeywordTok{of}\NormalTok{(val)}
\NormalTok{    \})}\OperatorTok{,}
    \FunctionTok{retry}\NormalTok{(}\DecValTok{2}\NormalTok{)}
\NormalTok{  )}
  \OperatorTok{.}\FunctionTok{subscribe}\NormalTok{(\{}
    \DataTypeTok{next}\OperatorTok{:} \BuiltInTok{console}\OperatorTok{.}\FunctionTok{log}\OperatorTok{,}
    \DataTypeTok{error}\OperatorTok{:} \BuiltInTok{console}\OperatorTok{.}\FunctionTok{log}
\NormalTok{  \})}
\end{Highlighting}
\end{Shaded}

\hypertarget{13312-race}{%
\subparagraph{13.3.12 race}\label{13312-race}}

接收并同时执行多个可观察对象,只将最快发出的数据流传递给订阅者。

\begin{Shaded}
\begin{Highlighting}[]
\ImportTok{import}\NormalTok{ \{ race}\OperatorTok{,}\NormalTok{ timer \} }\ImportTok{from} \StringTok{"rxjs"}
\ImportTok{import}\NormalTok{ \{ mapTo \} }\ImportTok{from} \StringTok{"rxjs/operators"}

\KeywordTok{const}\NormalTok{ obs1 }\OperatorTok{=} \FunctionTok{timer}\NormalTok{(}\DecValTok{1000}\NormalTok{)}\OperatorTok{.}\FunctionTok{pipe}\NormalTok{(}\FunctionTok{mapTo}\NormalTok{(}\StringTok{"fast one"}\NormalTok{))}
\KeywordTok{const}\NormalTok{ obs2 }\OperatorTok{=} \FunctionTok{timer}\NormalTok{(}\DecValTok{3000}\NormalTok{)}\OperatorTok{.}\FunctionTok{pipe}\NormalTok{(}\FunctionTok{mapTo}\NormalTok{(}\StringTok{"medium one"}\NormalTok{))}
\KeywordTok{const}\NormalTok{ obs3 }\OperatorTok{=} \FunctionTok{timer}\NormalTok{(}\DecValTok{5000}\NormalTok{)}\OperatorTok{.}\FunctionTok{pipe}\NormalTok{(}\FunctionTok{mapTo}\NormalTok{(}\StringTok{"slow one"}\NormalTok{))}

\FunctionTok{race}\NormalTok{(obs3}\OperatorTok{,}\NormalTok{ obs1}\OperatorTok{,}\NormalTok{ obs2)}\OperatorTok{.}\FunctionTok{subscribe}\NormalTok{(}\BuiltInTok{console}\OperatorTok{.}\FunctionTok{log}\NormalTok{)}
\end{Highlighting}
\end{Shaded}

\hypertarget{13313-fromevent}{%
\subparagraph{13.3.13 fromEvent}\label{13313-fromevent}}

将事件转换为 Observable。

\begin{Shaded}
\begin{Highlighting}[]
\ImportTok{import}\NormalTok{ \{ fromEvent \} }\ImportTok{from} \StringTok{"rxjs"}

\KeywordTok{const}\NormalTok{ btn }\OperatorTok{=} \BuiltInTok{document}\OperatorTok{.}\FunctionTok{getElementById}\NormalTok{(}\StringTok{"btn"}\NormalTok{)}
\CommentTok{// 可以将 Observer 简写成一个函数,表示 next}
\FunctionTok{fromEvent}\NormalTok{(btn}\OperatorTok{,} \StringTok{"click"}\NormalTok{)}\OperatorTok{.}\FunctionTok{subscribe}\NormalTok{(e }\KeywordTok{=\textgreater{}} \BuiltInTok{console}\OperatorTok{.}\FunctionTok{log}\NormalTok{(e))}
\end{Highlighting}
\end{Shaded}

\hypertarget{134-ux64cdux4f5cux7b26}{%
\paragraph{13.4 操作符}\label{134-ux64cdux4f5cux7b26}}

\begin{enumerate}
\def\labelenumi{\arabic{enumi}.}
\item
  数据流:从可观察对象内部输出的数据就是数据流,可观察对象内部可以向外部源源不断的输出数据。
\item
  操作符:用于操作数据流,可以对象数据流进行转换,过滤等等操作。
\end{enumerate}

\hypertarget{1341-mapmapto}{%
\subparagraph{13.4.1 map、mapTo}\label{1341-mapmapto}}

\textbf{map:}对数据流进行转换,基于原有值进行转换。

\begin{figure}
\centering
\includegraphics{C:/Users/ZSH/Desktop/ng/ppt/images/13.png}
\caption{}
\end{figure}

\begin{Shaded}
\begin{Highlighting}[]
\ImportTok{import}\NormalTok{ \{ interval \} }\ImportTok{from} \StringTok{"rxjs"}
\ImportTok{import}\NormalTok{ \{ map \} }\ImportTok{from} \StringTok{"rxjs/operators"}

\FunctionTok{interval}\NormalTok{(}\DecValTok{1000}\NormalTok{)}
  \OperatorTok{.}\FunctionTok{pipe}\NormalTok{(}\FunctionTok{map}\NormalTok{(n }\KeywordTok{=\textgreater{}}\NormalTok{ n }\OperatorTok{*} \DecValTok{2}\NormalTok{))}
  \OperatorTok{.}\FunctionTok{subscribe}\NormalTok{(n }\KeywordTok{=\textgreater{}} \BuiltInTok{console}\OperatorTok{.}\FunctionTok{log}\NormalTok{(n))}
\end{Highlighting}
\end{Shaded}

\textbf{mapTo:}对数据流进行转换,不关心原有值,可以直接传入要转换后的值。

\begin{figure}
\centering
\includegraphics{C:/Users/ZSH/Desktop/ng/ppt/images/14.png}
\caption{}
\end{figure}

\begin{Shaded}
\begin{Highlighting}[]
\ImportTok{import}\NormalTok{ \{ interval \} }\ImportTok{from} \StringTok{"rxjs"}
\ImportTok{import}\NormalTok{ \{ mapTo \} }\ImportTok{from} \StringTok{"rxjs/operators"}

\FunctionTok{interval}\NormalTok{(}\DecValTok{1000}\NormalTok{)}
  \OperatorTok{.}\FunctionTok{pipe}\NormalTok{(}\FunctionTok{mapTo}\NormalTok{(\{ }\DataTypeTok{msg}\OperatorTok{:} \StringTok{"接收到了数据流"}\NormalTok{ \}))}
  \OperatorTok{.}\FunctionTok{subscribe}\NormalTok{(msg }\KeywordTok{=\textgreater{}} \BuiltInTok{console}\OperatorTok{.}\FunctionTok{log}\NormalTok{(msg))}
\end{Highlighting}
\end{Shaded}

\hypertarget{1342-filter}{%
\subparagraph{13.4.2 filter}\label{1342-filter}}

对数据流进行过滤。

\begin{figure}
\centering
\includegraphics{C:/Users/ZSH/Desktop/ng/ppt/images/15.png}
\caption{}
\end{figure}

\begin{Shaded}
\begin{Highlighting}[]
\ImportTok{import}\NormalTok{ \{ range \} }\ImportTok{from} \StringTok{"rxjs"}
\ImportTok{import}\NormalTok{ \{ filter \} }\ImportTok{from} \StringTok{"rxjs/operators"}

\FunctionTok{range}\NormalTok{(}\DecValTok{1}\OperatorTok{,} \DecValTok{10}\NormalTok{)}
  \OperatorTok{.}\FunctionTok{pipe}\NormalTok{(}\FunctionTok{filter}\NormalTok{(n }\KeywordTok{=\textgreater{}}\NormalTok{ n }\OperatorTok{\%} \DecValTok{2} \OperatorTok{===} \DecValTok{0}\NormalTok{))}
  \OperatorTok{.}\FunctionTok{subscribe}\NormalTok{(even }\KeywordTok{=\textgreater{}} \BuiltInTok{console}\OperatorTok{.}\FunctionTok{log}\NormalTok{(even))}
\end{Highlighting}
\end{Shaded}

\hypertarget{1343-pluck}{%
\subparagraph{13.4.3 pluck}\label{1343-pluck}}

获取数据流对象中的属性值。

\begin{figure}
\centering
\includegraphics{C:/Users/ZSH/Desktop/ng/ppt/images/16.png}
\caption{}
\end{figure}

\begin{Shaded}
\begin{Highlighting}[]
\ImportTok{import}\NormalTok{ \{ interval \} }\ImportTok{from} \StringTok{"rxjs"}
\ImportTok{import}\NormalTok{ \{ pluck}\OperatorTok{,}\NormalTok{ mapTo \} }\ImportTok{from} \StringTok{"rxjs/operators"}

\FunctionTok{interval}\NormalTok{(}\DecValTok{1000}\NormalTok{)}
  \OperatorTok{.}\FunctionTok{pipe}\NormalTok{(}
  	\FunctionTok{mapTo}\NormalTok{(\{ }\DataTypeTok{name}\OperatorTok{:} \StringTok{"张三"}\OperatorTok{,} \DataTypeTok{a}\OperatorTok{:}\NormalTok{ \{ }\DataTypeTok{b}\OperatorTok{:} \StringTok{"c"}\NormalTok{ \} \})}\OperatorTok{,} 
  	\FunctionTok{pluck}\NormalTok{(}\StringTok{"a"}\OperatorTok{,} \StringTok{"b"}\NormalTok{)}
\NormalTok{	)}
  \OperatorTok{.}\FunctionTok{subscribe}\NormalTok{(n }\KeywordTok{=\textgreater{}} \BuiltInTok{console}\OperatorTok{.}\FunctionTok{log}\NormalTok{(n))}
\end{Highlighting}
\end{Shaded}

\hypertarget{1344-first}{%
\subparagraph{13.4.4 first}\label{1344-first}}

获取数据流中的第一个值或者查找数据流中第一个符合条件的值,类似数组中的
find 方法。获取到值以后终止行为。

\begin{figure}
\centering
\includegraphics{C:/Users/ZSH/Desktop/ng/ppt/images/17.png}
\caption{}
\end{figure}

\begin{Shaded}
\begin{Highlighting}[]
\ImportTok{import}\NormalTok{ \{ interval \} }\ImportTok{from} \StringTok{"rxjs"}
\ImportTok{import}\NormalTok{ \{ first \} }\ImportTok{from} \StringTok{"rxjs/operators"}

\FunctionTok{interval}\NormalTok{(}\DecValTok{1000}\NormalTok{)}
  \OperatorTok{.}\FunctionTok{pipe}\NormalTok{(}\FunctionTok{first}\NormalTok{())}
  \OperatorTok{.}\FunctionTok{subscribe}\NormalTok{(n }\KeywordTok{=\textgreater{}} \BuiltInTok{console}\OperatorTok{.}\FunctionTok{log}\NormalTok{(n))}

\FunctionTok{interval}\NormalTok{(}\DecValTok{1000}\NormalTok{)}
  \OperatorTok{.}\FunctionTok{pipe}\NormalTok{(}\FunctionTok{first}\NormalTok{(n }\KeywordTok{=\textgreater{}}\NormalTok{ n }\OperatorTok{===} \DecValTok{3}\NormalTok{))}
  \OperatorTok{.}\FunctionTok{subscribe}\NormalTok{(n }\KeywordTok{=\textgreater{}} \BuiltInTok{console}\OperatorTok{.}\FunctionTok{log}\NormalTok{(n))}
\end{Highlighting}
\end{Shaded}

\hypertarget{1345-startwith}{%
\subparagraph{13.4.5 startWith}\label{1345-startwith}}

创建一个新的 observable 对象并将参数值发送出去,然后再发送源 observable
对象发出的值。

在异步编程中提供默认值的时候非常有用。

\begin{figure}
\centering
\includegraphics{C:/Users/ZSH/Desktop/ng/ppt/images/18.png}
\caption{}
\end{figure}

\begin{Shaded}
\begin{Highlighting}[]
\ImportTok{import}\NormalTok{ \{ interval \} }\ImportTok{from} \StringTok{"rxjs"}
\ImportTok{import}\NormalTok{ \{ map}\OperatorTok{,}\NormalTok{ startWith \} }\ImportTok{from} \StringTok{"rxjs/operators"}

\FunctionTok{interval}\NormalTok{(}\DecValTok{1000}\NormalTok{)}
  \OperatorTok{.}\FunctionTok{pipe}\NormalTok{(}
    \FunctionTok{map}\NormalTok{(n }\KeywordTok{=\textgreater{}}\NormalTok{ n }\OperatorTok{+} \DecValTok{100}\NormalTok{)}\OperatorTok{,}
    \FunctionTok{startWith}\NormalTok{(}\DecValTok{505}\NormalTok{)}
\NormalTok{  )}
  \OperatorTok{.}\FunctionTok{subscribe}\NormalTok{(n }\KeywordTok{=\textgreater{}} \BuiltInTok{console}\OperatorTok{.}\FunctionTok{log}\NormalTok{(n))}
\CommentTok{// 505}
\CommentTok{// 100}
\CommentTok{// 101}
\CommentTok{// 102}
\CommentTok{// ...}
\end{Highlighting}
\end{Shaded}

\hypertarget{1346-every}{%
\subparagraph{13.4.6 every}\label{1346-every}}

查看数据流中的每个值是否都符合条件,返回布尔值。类似数组中的 every
方法。

\begin{figure}
\centering
\includegraphics{C:/Users/ZSH/Desktop/ng/ppt/images/28.png}
\caption{}
\end{figure}

\begin{Shaded}
\begin{Highlighting}[]
\ImportTok{import}\NormalTok{ \{ range \} }\ImportTok{from} \StringTok{"rxjs"}
\ImportTok{import}\NormalTok{ \{ every}\OperatorTok{,}\NormalTok{ map \} }\ImportTok{from} \StringTok{"rxjs/operators"}

\FunctionTok{range}\NormalTok{(}\DecValTok{1}\OperatorTok{,} \DecValTok{9}\NormalTok{)}
  \OperatorTok{.}\FunctionTok{pipe}\NormalTok{(}
    \FunctionTok{map}\NormalTok{(n }\KeywordTok{=\textgreater{}}\NormalTok{ n }\OperatorTok{*} \DecValTok{2}\NormalTok{)}\OperatorTok{,}
    \FunctionTok{every}\NormalTok{(n }\KeywordTok{=\textgreater{}}\NormalTok{ n }\OperatorTok{\%} \DecValTok{2} \OperatorTok{===} \DecValTok{0}\NormalTok{)}
\NormalTok{  )}
  \OperatorTok{.}\FunctionTok{subscribe}\NormalTok{(b }\KeywordTok{=\textgreater{}} \BuiltInTok{console}\OperatorTok{.}\FunctionTok{log}\NormalTok{(b))}
\end{Highlighting}
\end{Shaded}

\hypertarget{1347-delaydelaywhen}{%
\subparagraph{13.4.7 delay、delayWhen}\label{1347-delaydelaywhen}}

\textbf{delay:}对上一环节的操作整体进行延迟,只执行一次。

\begin{figure}
\centering
\includegraphics{C:/Users/ZSH/Desktop/ng/ppt/images/19.png}
\caption{}
\end{figure}

\begin{Shaded}
\begin{Highlighting}[]
\ImportTok{import}\NormalTok{ \{ }\ImportTok{from}\NormalTok{ \} }\ImportTok{from} \StringTok{"rxjs"}
\ImportTok{import}\NormalTok{ \{ delay}\OperatorTok{,}\NormalTok{ map}\OperatorTok{,}\NormalTok{ tap \} }\ImportTok{from} \StringTok{"rxjs/operators"}

\ImportTok{from}\NormalTok{([}\DecValTok{1}\OperatorTok{,} \DecValTok{2}\OperatorTok{,} \DecValTok{3}\NormalTok{])}
  \OperatorTok{.}\FunctionTok{pipe}\NormalTok{(}
    \FunctionTok{delay}\NormalTok{(}\DecValTok{1000}\NormalTok{)}\OperatorTok{,}
    \FunctionTok{tap}\NormalTok{(n }\KeywordTok{=\textgreater{}} \BuiltInTok{console}\OperatorTok{.}\FunctionTok{log}\NormalTok{(}\StringTok{"已经延迟 1s"}\OperatorTok{,}\NormalTok{ n))}\OperatorTok{,}
    \FunctionTok{map}\NormalTok{(n }\KeywordTok{=\textgreater{}}\NormalTok{ n }\OperatorTok{*} \DecValTok{2}\NormalTok{)}\OperatorTok{,}
    \FunctionTok{delay}\NormalTok{(}\DecValTok{1000}\NormalTok{)}\OperatorTok{,}
    \FunctionTok{tap}\NormalTok{(() }\KeywordTok{=\textgreater{}} \BuiltInTok{console}\OperatorTok{.}\FunctionTok{log}\NormalTok{(}\StringTok{"又延迟了 1s"}\NormalTok{))}
\NormalTok{  )}
  \OperatorTok{.}\FunctionTok{subscribe}\NormalTok{(}\BuiltInTok{console}\OperatorTok{.}\FunctionTok{log}\NormalTok{)}

\CommentTok{// tap 操作符不会对数据流造成影响, 它被用来执行简单的副作用, 比如输出, 但是复杂的副作用不要在这执行, 比如 Ajax}
\end{Highlighting}
\end{Shaded}

\textbf{delayWhen:}对上一环节的操作进行延迟,上一环节发出多少数据流,传入的回调函数就会执行多次。

\begin{figure}
\centering
\includegraphics{C:/Users/ZSH/Desktop/ng/ppt/images/20.png}
\caption{}
\end{figure}

\begin{Shaded}
\begin{Highlighting}[]
\ImportTok{import}\NormalTok{ \{ range}\OperatorTok{,}\NormalTok{ timer \} }\ImportTok{from} \StringTok{"rxjs"}
\ImportTok{import}\NormalTok{ \{ delayWhen \} }\ImportTok{from} \StringTok{"rxjs/operators"}

\FunctionTok{range}\NormalTok{(}\DecValTok{1}\OperatorTok{,} \DecValTok{10}\NormalTok{)}
  \OperatorTok{.}\FunctionTok{pipe}\NormalTok{(}
    \FunctionTok{delayWhen}\NormalTok{(n }\KeywordTok{=\textgreater{}}\NormalTok{ \{}
      \BuiltInTok{console}\OperatorTok{.}\FunctionTok{log}\NormalTok{(n)}
      \ControlFlowTok{return} \FunctionTok{timer}\NormalTok{(n }\OperatorTok{*} \DecValTok{1000}\NormalTok{)}
\NormalTok{    \})}
\NormalTok{  )}
  \OperatorTok{.}\FunctionTok{subscribe}\NormalTok{(}\BuiltInTok{console}\OperatorTok{.}\FunctionTok{log}\NormalTok{)}
\end{Highlighting}
\end{Shaded}

\hypertarget{1348-taketakewhiletakeutil}{%
\subparagraph{13.4.8
take、takeWhile、takeUtil}\label{1348-taketakewhiletakeutil}}

\textbf{take}:获取数据流中的前几个

\begin{figure}
\centering
\includegraphics{C:/Users/ZSH/Desktop/ng/ppt/images/21.png}
\caption{}
\end{figure}

\begin{Shaded}
\begin{Highlighting}[]
\ImportTok{import}\NormalTok{ \{ range \} }\ImportTok{from} \StringTok{"rxjs"}
\ImportTok{import}\NormalTok{ \{ take \} }\ImportTok{from} \StringTok{"rxjs/operators"}

\FunctionTok{range}\NormalTok{(}\DecValTok{1}\OperatorTok{,} \DecValTok{10}\NormalTok{)}\OperatorTok{.}\FunctionTok{pipe}\NormalTok{(}\FunctionTok{take}\NormalTok{(}\DecValTok{5}\NormalTok{))}\OperatorTok{.}\FunctionTok{subscribe}\NormalTok{(}\BuiltInTok{console}\OperatorTok{.}\FunctionTok{log}\NormalTok{)}
\end{Highlighting}
\end{Shaded}

\textbf{takeWhile:}根据条件从数据源前面开始获取。

\begin{figure}
\centering
\includegraphics{C:/Users/ZSH/Desktop/ng/ppt/images/22.png}
\caption{}
\end{figure}

\begin{Shaded}
\begin{Highlighting}[]
\ImportTok{import}\NormalTok{ \{ range \} }\ImportTok{from} \StringTok{"rxjs"}
\ImportTok{import}\NormalTok{ \{ takeWhile \} }\ImportTok{from} \StringTok{"rxjs/operators"}

\FunctionTok{range}\NormalTok{(}\DecValTok{1}\OperatorTok{,} \DecValTok{10}\NormalTok{)}
  \OperatorTok{.}\FunctionTok{pipe}\NormalTok{(}\FunctionTok{takeWhile}\NormalTok{(n }\KeywordTok{=\textgreater{}}\NormalTok{ n }\OperatorTok{\textless{}} \DecValTok{8}\NormalTok{))}
  \OperatorTok{.}\FunctionTok{subscribe}\NormalTok{(}\BuiltInTok{console}\OperatorTok{.}\FunctionTok{log}\NormalTok{)}
\end{Highlighting}
\end{Shaded}

\textbf{takeUntil:}接收可观察对象,当可观察对象发出值时,终止主数据源。

\begin{figure}
\centering
\includegraphics{C:/Users/ZSH/Desktop/ng/ppt/images/23.png}
\caption{}
\end{figure}

\begin{Shaded}
\begin{Highlighting}[]
\ImportTok{import}\NormalTok{ \{ interval}\OperatorTok{,}\NormalTok{ timer \} }\ImportTok{from} \StringTok{"rxjs"}
\ImportTok{import}\NormalTok{ \{ takeUntil \} }\ImportTok{from} \StringTok{"rxjs/operators"}

\FunctionTok{interval}\NormalTok{(}\DecValTok{100}\NormalTok{)}
  \OperatorTok{.}\FunctionTok{pipe}\NormalTok{(}\FunctionTok{takeUntil}\NormalTok{(}\FunctionTok{timer}\NormalTok{(}\DecValTok{2000}\NormalTok{)))}
  \OperatorTok{.}\FunctionTok{subscribe}\NormalTok{(}\BuiltInTok{console}\OperatorTok{.}\FunctionTok{log}\NormalTok{)}
\CommentTok{// 结果少两个数据流的原因:第一次和最后一次,都需要延迟 100 毫秒。}
\end{Highlighting}
\end{Shaded}

\hypertarget{1349-skipskipwhileskipuntil}{%
\subparagraph{13.4.9
skip、skipWhile、skipUntil}\label{1349-skipskipwhileskipuntil}}

\textbf{skip:}跳过前几个数据流。

\begin{figure}
\centering
\includegraphics{C:/Users/ZSH/Desktop/ng/ppt/images/24.png}
\caption{}
\end{figure}

\begin{Shaded}
\begin{Highlighting}[]
\ImportTok{import}\NormalTok{ \{ range \} }\ImportTok{from} \StringTok{"rxjs"}
\ImportTok{import}\NormalTok{ \{ skip \} }\ImportTok{from} \StringTok{"rxjs/operators"}

\FunctionTok{range}\NormalTok{(}\DecValTok{1}\OperatorTok{,} \DecValTok{10}\NormalTok{)}\OperatorTok{.}\FunctionTok{pipe}\NormalTok{(}\FunctionTok{skip}\NormalTok{(}\DecValTok{5}\NormalTok{))}\OperatorTok{.}\FunctionTok{subscribe}\NormalTok{(}\BuiltInTok{console}\OperatorTok{.}\FunctionTok{log}\NormalTok{)}
\end{Highlighting}
\end{Shaded}

\textbf{skipWhile:}根据条件进行数据流的跳过。

\begin{figure}
\centering
\includegraphics{C:/Users/ZSH/Desktop/ng/ppt/images/25.png}
\caption{}
\end{figure}

\begin{Shaded}
\begin{Highlighting}[]
\ImportTok{import}\NormalTok{ \{ range \} }\ImportTok{from} \StringTok{"rxjs"}
\ImportTok{import}\NormalTok{ \{ skipWhile \} }\ImportTok{from} \StringTok{"rxjs/operators"}

\FunctionTok{range}\NormalTok{(}\DecValTok{1}\OperatorTok{,} \DecValTok{10}\NormalTok{)}
  \OperatorTok{.}\FunctionTok{pipe}\NormalTok{(}\FunctionTok{skipWhile}\NormalTok{(n }\KeywordTok{=\textgreater{}}\NormalTok{ n }\OperatorTok{\textless{}} \DecValTok{5}\NormalTok{))}
  \OperatorTok{.}\FunctionTok{subscribe}\NormalTok{(}\BuiltInTok{console}\OperatorTok{.}\FunctionTok{log}\NormalTok{)}
\end{Highlighting}
\end{Shaded}

\textbf{skipUntil:}跳过数据源中前多少时间发出的数据流,发送从这个时间以后数据源中发送的数据流。

\begin{figure}
\centering
\includegraphics{C:/Users/ZSH/Desktop/ng/ppt/images/26.png}
\caption{}
\end{figure}

\begin{Shaded}
\begin{Highlighting}[]
\ImportTok{import}\NormalTok{ \{ timer}\OperatorTok{,}\NormalTok{ interval \} }\ImportTok{from} \StringTok{"rxjs"}
\ImportTok{import}\NormalTok{ \{ skipUntil \} }\ImportTok{from} \StringTok{"rxjs/operators"}

\FunctionTok{interval}\NormalTok{(}\DecValTok{100}\NormalTok{)}
  \OperatorTok{.}\FunctionTok{pipe}\NormalTok{(}\FunctionTok{skipUntil}\NormalTok{(}\FunctionTok{timer}\NormalTok{(}\DecValTok{2000}\NormalTok{)))}
  \OperatorTok{.}\FunctionTok{subscribe}\NormalTok{(}\BuiltInTok{console}\OperatorTok{.}\FunctionTok{log}\NormalTok{)}
\end{Highlighting}
\end{Shaded}

\hypertarget{13410-last}{%
\subparagraph{13.4.10 last}\label{13410-last}}

获取数据流中的最后一个。

\begin{figure}
\centering
\includegraphics{C:/Users/ZSH/Desktop/ng/ppt/images/27.png}
\caption{}
\end{figure}

\begin{Shaded}
\begin{Highlighting}[]
\ImportTok{import}\NormalTok{ \{ range \} }\ImportTok{from} \StringTok{"rxjs"}
\ImportTok{import}\NormalTok{ \{ last \} }\ImportTok{from} \StringTok{"rxjs/operators"}

\FunctionTok{range}\NormalTok{(}\DecValTok{1}\OperatorTok{,} \DecValTok{10}\NormalTok{)}\OperatorTok{.}\FunctionTok{pipe}\NormalTok{(}\FunctionTok{last}\NormalTok{())}\OperatorTok{.}\FunctionTok{subscribe}\NormalTok{(}\BuiltInTok{console}\OperatorTok{.}\FunctionTok{log}\NormalTok{)}
\end{Highlighting}
\end{Shaded}

如果数据源不变成完成状态,则没有最后一个。

\begin{Shaded}
\begin{Highlighting}[]
\ImportTok{import}\NormalTok{ \{ interval \} }\ImportTok{from} \StringTok{"rxjs"}
\ImportTok{import}\NormalTok{ \{ last}\OperatorTok{,}\NormalTok{ take \} }\ImportTok{from} \StringTok{"rxjs/operators"}

\FunctionTok{interval}\NormalTok{(}\DecValTok{1000}\NormalTok{)}\OperatorTok{.}\FunctionTok{pipe}\NormalTok{(}\FunctionTok{take}\NormalTok{(}\DecValTok{5}\NormalTok{)}\OperatorTok{,} \FunctionTok{last}\NormalTok{())}\OperatorTok{.}\FunctionTok{subscribe}\NormalTok{(}\BuiltInTok{console}\OperatorTok{.}\FunctionTok{log}\NormalTok{)}
\end{Highlighting}
\end{Shaded}

\hypertarget{13411-concatallconcatmap}{%
\subparagraph{13.4.11
concatAll、concatMap}\label{13411-concatallconcatmap}}

\textbf{concatAll:}有时 Observable 发出的又是一个 Obervable,concatAll
的作用就是将新的可观察对象和数据源进行合并。

Observable =\textgreater{} {[}1, 2, 3{]}

Observable =\textgreater{} {[}Observable, Observable{]}

\begin{figure}
\centering
\includegraphics{C:/Users/ZSH/Desktop/ng/ppt/images/29.png}
\caption{}
\end{figure}

\begin{Shaded}
\begin{Highlighting}[]
\ImportTok{import}\NormalTok{ \{ fromEvent}\OperatorTok{,}\NormalTok{ interval \} }\ImportTok{from} \StringTok{"rxjs"}
\ImportTok{import}\NormalTok{ \{ map}\OperatorTok{,}\NormalTok{ take}\OperatorTok{,}\NormalTok{ concatAll \} }\ImportTok{from} \StringTok{"rxjs/operators"}

\FunctionTok{fromEvent}\NormalTok{(}\BuiltInTok{document}\OperatorTok{,} \StringTok{"click"}\NormalTok{)}
  \OperatorTok{.}\FunctionTok{pipe}\NormalTok{(}
    \FunctionTok{map}\NormalTok{(}\BuiltInTok{event} \KeywordTok{=\textgreater{}} \FunctionTok{interval}\NormalTok{(}\DecValTok{1000}\NormalTok{)}\OperatorTok{.}\FunctionTok{pipe}\NormalTok{(}\FunctionTok{take}\NormalTok{(}\DecValTok{2}\NormalTok{)))}\OperatorTok{,}
    \FunctionTok{concatAll}\NormalTok{()}
\NormalTok{  )}
  \OperatorTok{.}\FunctionTok{subscribe}\NormalTok{(}\BuiltInTok{console}\OperatorTok{.}\FunctionTok{log}\NormalTok{)}
\end{Highlighting}
\end{Shaded}

\begin{Shaded}
\begin{Highlighting}[]
\ImportTok{import}\NormalTok{ \{ map}\OperatorTok{,}\NormalTok{ concatAll \} }\ImportTok{from} \StringTok{"rxjs/operators"}
\ImportTok{import}\NormalTok{ \{ }\KeywordTok{of}\OperatorTok{,}\NormalTok{ interval \} }\ImportTok{from} \StringTok{"rxjs"}

\FunctionTok{interval}\NormalTok{(}\DecValTok{1000}\NormalTok{)}
  \OperatorTok{.}\FunctionTok{pipe}\NormalTok{(}
    \FunctionTok{map}\NormalTok{(val }\KeywordTok{=\textgreater{}} \KeywordTok{of}\NormalTok{(val }\OperatorTok{+} \DecValTok{10}\NormalTok{))}\OperatorTok{,}
    \FunctionTok{concatAll}\NormalTok{()}
\NormalTok{  )}
  \OperatorTok{.}\FunctionTok{subscribe}\NormalTok{(}\BuiltInTok{console}\OperatorTok{.}\FunctionTok{log}\NormalTok{)}
\end{Highlighting}
\end{Shaded}

\textbf{concatMap:}合并可观察对象并处理其发出的数据流。

\begin{figure}
\centering
\includegraphics{C:/Users/ZSH/Desktop/ng/ppt/images/30.png}
\caption{}
\end{figure}

\hypertarget{13413-reducescan}{%
\subparagraph{13.4.13 reduce、scan}\label{13413-reducescan}}

\textbf{reduce}: 类似 JavaScript 数组中的
reduce,对数数据进行累计操作。reduce
会等待数据源中的数据流发送完成后再执行,执行时 reduce
内部遍历每一个数据流进行累计操作,操作完成得到结果将结果作为数据流发出。

\begin{figure}
\centering
\includegraphics{C:/Users/ZSH/Desktop/ng/ppt/images/31.png}
\caption{}
\end{figure}

\begin{Shaded}
\begin{Highlighting}[]
\ImportTok{import}\NormalTok{ \{ interval \} }\ImportTok{from} \StringTok{"rxjs"}
\ImportTok{import}\NormalTok{ \{ take}\OperatorTok{,}\NormalTok{ reduce \} }\ImportTok{from} \StringTok{"rxjs/operators"}

\FunctionTok{interval}\NormalTok{(}\DecValTok{500}\NormalTok{)}
  \OperatorTok{.}\FunctionTok{pipe}\NormalTok{(}
    \FunctionTok{take}\NormalTok{(}\DecValTok{5}\NormalTok{)}\OperatorTok{,}
    \FunctionTok{reduce}\NormalTok{((acc}\OperatorTok{,}\NormalTok{ value) }\KeywordTok{=\textgreater{}}\NormalTok{ acc }\OperatorTok{+=}\NormalTok{ value}\OperatorTok{,} \DecValTok{0}\NormalTok{)}
\NormalTok{  )}
  \OperatorTok{.}\FunctionTok{subscribe}\NormalTok{(v }\KeywordTok{=\textgreater{}} \BuiltInTok{console}\OperatorTok{.}\FunctionTok{log}\NormalTok{())}
\end{Highlighting}
\end{Shaded}

\textbf{scan}:类似
reduce,进行累计操作,但执行时机不同,数据源每次发出数据流 scan
都会执行。reduce 是发送出最终计算的结果,而 scan 是发出每次计算的结果。

\begin{figure}
\centering
\includegraphics{C:/Users/ZSH/Desktop/ng/ppt/images/32.png}
\caption{}
\end{figure}

\begin{Shaded}
\begin{Highlighting}[]
\ImportTok{import}\NormalTok{ \{ interval \} }\ImportTok{from} \StringTok{"rxjs"}
\ImportTok{import}\NormalTok{ \{ take}\OperatorTok{,}\NormalTok{ scan \} }\ImportTok{from} \StringTok{"rxjs/operators"}

\FunctionTok{interval}\NormalTok{(}\DecValTok{500}\NormalTok{)}
  \OperatorTok{.}\FunctionTok{pipe}\NormalTok{(}
    \FunctionTok{take}\NormalTok{(}\DecValTok{5}\NormalTok{)}\OperatorTok{,}
    \FunctionTok{scan}\NormalTok{((acc}\OperatorTok{,}\NormalTok{ value) }\KeywordTok{=\textgreater{}}\NormalTok{ acc }\OperatorTok{+=}\NormalTok{ value}\OperatorTok{,} \DecValTok{0}\NormalTok{)}
\NormalTok{  )}
  \OperatorTok{.}\FunctionTok{subscribe}\NormalTok{(v }\KeywordTok{=\textgreater{}} \BuiltInTok{console}\OperatorTok{.}\FunctionTok{log}\NormalTok{())}
\end{Highlighting}
\end{Shaded}

\hypertarget{13414-mergeallmergemap}{%
\subparagraph{13.4.14 mergeAll、mergeMap}\label{13414-mergeallmergemap}}

\textbf{mergeAll:}交叉合并可观察对象。

\begin{figure}
\centering
\includegraphics{C:/Users/ZSH/Desktop/ng/ppt/images/34.png}
\caption{}
\end{figure}

\begin{Shaded}
\begin{Highlighting}[]
\ImportTok{import}\NormalTok{ \{ fromEvent}\OperatorTok{,}\NormalTok{ interval \} }\ImportTok{from} \StringTok{"rxjs"}
\ImportTok{import}\NormalTok{ \{ map}\OperatorTok{,}\NormalTok{ mergeAll \} }\ImportTok{from} \StringTok{"rxjs/operators"}

\FunctionTok{fromEvent}\NormalTok{(}\BuiltInTok{document}\OperatorTok{,} \StringTok{"click"}\NormalTok{)}
  \OperatorTok{.}\FunctionTok{pipe}\NormalTok{(}
    \FunctionTok{map}\NormalTok{(() }\KeywordTok{=\textgreater{}} \FunctionTok{interval}\NormalTok{(}\DecValTok{1000}\NormalTok{))}\OperatorTok{,}
    \FunctionTok{mergeAll}\NormalTok{()}
\NormalTok{  )}
  \OperatorTok{.}\FunctionTok{subscribe}\NormalTok{(}\BuiltInTok{console}\OperatorTok{.}\FunctionTok{log}\NormalTok{)}
\end{Highlighting}
\end{Shaded}

\textbf{mergeMap}:交叉合并可观察对象以后对可观察对象发出的数据流进行转换。

\begin{figure}
\centering
\includegraphics{C:/Users/ZSH/Desktop/ng/ppt/images/35.png}
\caption{}
\end{figure}

\begin{Shaded}
\begin{Highlighting}[]
\ImportTok{import}\NormalTok{ \{ }\KeywordTok{of}\OperatorTok{,}\NormalTok{ interval \} }\ImportTok{from} \StringTok{"rxjs"}
\ImportTok{import}\NormalTok{ \{ mergeMap}\OperatorTok{,}\NormalTok{ map \} }\ImportTok{from} \StringTok{"rxjs/operators"}

\KeywordTok{of}\NormalTok{(}\StringTok{"a"}\OperatorTok{,} \StringTok{"b"}\OperatorTok{,} \StringTok{"c"}\NormalTok{)}
  \OperatorTok{.}\FunctionTok{pipe}\NormalTok{(}\FunctionTok{mergeMap}\NormalTok{(x }\KeywordTok{=\textgreater{}} \FunctionTok{interval}\NormalTok{(}\DecValTok{1000}\NormalTok{)}\OperatorTok{.}\FunctionTok{pipe}\NormalTok{(}\FunctionTok{map}\NormalTok{(i }\KeywordTok{=\textgreater{}}\NormalTok{ x }\OperatorTok{+}\NormalTok{ i))))}
  \OperatorTok{.}\FunctionTok{subscribe}\NormalTok{(x }\KeywordTok{=\textgreater{}} \BuiltInTok{console}\OperatorTok{.}\FunctionTok{log}\NormalTok{(x))}
\end{Highlighting}
\end{Shaded}

\hypertarget{13415-throttletime}{%
\subparagraph{13.4.15 throttleTime}\label{13415-throttletime}}

节流,可观察对象高频次向外部发出数据流,通过 throttleTime
限制在规定时间内每次只向订阅者传递一次数据流。

\begin{figure}
\centering
\includegraphics{C:/Users/ZSH/Desktop/ng/ppt/images/36.png}
\caption{}
\end{figure}

\begin{Shaded}
\begin{Highlighting}[]
\ImportTok{import}\NormalTok{ \{ fromEvent \} }\ImportTok{from} \StringTok{"rxjs"}
\ImportTok{import}\NormalTok{ \{ throttleTime \} }\ImportTok{from} \StringTok{"rxjs/operators"}

\FunctionTok{fromEvent}\NormalTok{(}\BuiltInTok{document}\OperatorTok{,} \StringTok{"click"}\NormalTok{)}
  \OperatorTok{.}\FunctionTok{pipe}\NormalTok{(}\FunctionTok{throttleTime}\NormalTok{(}\DecValTok{2000}\NormalTok{))}
  \OperatorTok{.}\FunctionTok{subscribe}\NormalTok{(x }\KeywordTok{=\textgreater{}} \BuiltInTok{console}\OperatorTok{.}\FunctionTok{log}\NormalTok{(x))}
\end{Highlighting}
\end{Shaded}

\hypertarget{13416-debouncetime}{%
\subparagraph{13.4.16 debounceTime}\label{13416-debouncetime}}

防抖,触发高频事件,只响应最后一次。

\begin{figure}
\centering
\includegraphics{C:/Users/ZSH/Desktop/ng/ppt/images/37.png}
\caption{}
\end{figure}

\begin{Shaded}
\begin{Highlighting}[]
\ImportTok{import}\NormalTok{ \{ fromEvent \} }\ImportTok{from} \StringTok{"rxjs"}
\ImportTok{import}\NormalTok{ \{ debounceTime \} }\ImportTok{from} \StringTok{"rxjs/operators"}

\FunctionTok{fromEvent}\NormalTok{(}\BuiltInTok{document}\OperatorTok{,} \StringTok{"click"}\NormalTok{)}
  \OperatorTok{.}\FunctionTok{pipe}\NormalTok{(}\FunctionTok{debounceTime}\NormalTok{(}\DecValTok{1000}\NormalTok{))}
  \OperatorTok{.}\FunctionTok{subscribe}\NormalTok{(x }\KeywordTok{=\textgreater{}} \BuiltInTok{console}\OperatorTok{.}\FunctionTok{log}\NormalTok{(x))}
\end{Highlighting}
\end{Shaded}

\hypertarget{13417-distinctuntilchanged}{%
\subparagraph{13.4.17
distinctUntilChanged}\label{13417-distinctuntilchanged}}

检测数据源当前发出的数据流是否和上次发出的相同,如相同,跳过,不相同,发出。

\begin{figure}
\centering
\includegraphics{C:/Users/ZSH/Desktop/ng/ppt/images/38.png}
\caption{}
\end{figure}

\begin{Shaded}
\begin{Highlighting}[]
\ImportTok{import}\NormalTok{ \{ }\KeywordTok{of}\NormalTok{ \} }\ImportTok{from} \StringTok{"rxjs"}
\ImportTok{import}\NormalTok{ \{ distinctUntilChanged \} }\ImportTok{from} \StringTok{"rxjs/operators"}

\KeywordTok{of}\NormalTok{(}\DecValTok{1}\OperatorTok{,} \DecValTok{1}\OperatorTok{,} \DecValTok{2}\OperatorTok{,} \DecValTok{2}\OperatorTok{,} \DecValTok{2}\OperatorTok{,} \DecValTok{1}\OperatorTok{,} \DecValTok{1}\OperatorTok{,} \DecValTok{2}\OperatorTok{,} \DecValTok{3}\OperatorTok{,} \DecValTok{3}\OperatorTok{,} \DecValTok{4}\NormalTok{)}
  \OperatorTok{.}\FunctionTok{pipe}\NormalTok{(}\FunctionTok{distinctUntilChanged}\NormalTok{())}
  \OperatorTok{.}\FunctionTok{subscribe}\NormalTok{(x }\KeywordTok{=\textgreater{}} \BuiltInTok{console}\OperatorTok{.}\FunctionTok{log}\NormalTok{(x)) }\CommentTok{// 1, 2, 1, 2, 3, 4}
\end{Highlighting}
\end{Shaded}

\hypertarget{13418-groupby}{%
\subparagraph{13.4.18 groupBy}\label{13418-groupby}}

对数据流进行分组。

\begin{figure}
\centering
\includegraphics{C:/Users/ZSH/Desktop/ng/ppt/images/44.png}
\caption{}
\end{figure}

\begin{Shaded}
\begin{Highlighting}[]
\ImportTok{import}\NormalTok{ \{ }\KeywordTok{of}\NormalTok{ \} }\ImportTok{from} \StringTok{"rxjs"}
\ImportTok{import}\NormalTok{ \{ mergeMap}\OperatorTok{,}\NormalTok{ groupBy}\OperatorTok{,}\NormalTok{ toArray \} }\ImportTok{from} \StringTok{"rxjs/operators"}

\KeywordTok{of}\NormalTok{(}
\NormalTok{  \{ }\DataTypeTok{name}\OperatorTok{:} \StringTok{"Sue"}\OperatorTok{,} \DataTypeTok{age}\OperatorTok{:} \DecValTok{25}\NormalTok{ \}}\OperatorTok{,}
\NormalTok{  \{ }\DataTypeTok{name}\OperatorTok{:} \StringTok{"Joe"}\OperatorTok{,} \DataTypeTok{age}\OperatorTok{:} \DecValTok{30}\NormalTok{ \}}\OperatorTok{,}
\NormalTok{  \{ }\DataTypeTok{name}\OperatorTok{:} \StringTok{"Frank"}\OperatorTok{,} \DataTypeTok{age}\OperatorTok{:} \DecValTok{25}\NormalTok{ \}}\OperatorTok{,}
\NormalTok{  \{ }\DataTypeTok{name}\OperatorTok{:} \StringTok{"Sarah"}\OperatorTok{,} \DataTypeTok{age}\OperatorTok{:} \DecValTok{35}\NormalTok{ \}}
\NormalTok{)}
  \OperatorTok{.}\FunctionTok{pipe}\NormalTok{(}
    \FunctionTok{groupBy}\NormalTok{(person }\KeywordTok{=\textgreater{}}\NormalTok{ person}\OperatorTok{.}\AttributeTok{age}\NormalTok{)}\OperatorTok{,}
    \FunctionTok{mergeMap}\NormalTok{(group }\KeywordTok{=\textgreater{}}\NormalTok{ group}\OperatorTok{.}\FunctionTok{pipe}\NormalTok{(}\FunctionTok{toArray}\NormalTok{()))}
\NormalTok{  )}
  \OperatorTok{.}\FunctionTok{subscribe}\NormalTok{(}\BuiltInTok{console}\OperatorTok{.}\FunctionTok{log}\NormalTok{)}

\CommentTok{// [\{name: "Sue", age: 25\}, \{ name: "Frank", age: 25 \}]}
\CommentTok{// [\{ name: "Joe", age: 30 \}]}
\CommentTok{// [\{ name: "Sarah", age: 35 \}]}
\end{Highlighting}
\end{Shaded}

\hypertarget{13419-withlatestfrom}{%
\subparagraph{13.4.19 withLatestFrom}\label{13419-withlatestfrom}}

主数据源发出的数据流总是和支数据源中的最新数据流进行结合,返回数组。

\begin{figure}
\centering
\includegraphics{C:/Users/ZSH/Desktop/ng/ppt/images/45.png}
\caption{}
\end{figure}

\begin{Shaded}
\begin{Highlighting}[]
\ImportTok{import}\NormalTok{ \{ fromEvent}\OperatorTok{,}\NormalTok{ interval \} }\ImportTok{from} \StringTok{"rxjs"}
\ImportTok{import}\NormalTok{ \{ withLatestFrom \} }\ImportTok{from} \StringTok{"rxjs/operators"}

\KeywordTok{const}\NormalTok{ clicks }\OperatorTok{=} \FunctionTok{fromEvent}\NormalTok{(}\BuiltInTok{document}\OperatorTok{,} \StringTok{"click"}\NormalTok{)}
\KeywordTok{const}\NormalTok{ timer }\OperatorTok{=} \FunctionTok{interval}\NormalTok{(}\DecValTok{1000}\NormalTok{)}
\NormalTok{clicks}\OperatorTok{.}\FunctionTok{pipe}\NormalTok{(}\FunctionTok{withLatestFrom}\NormalTok{(timer))}\OperatorTok{.}\FunctionTok{subscribe}\NormalTok{(}\BuiltInTok{console}\OperatorTok{.}\FunctionTok{log}\NormalTok{)}
\end{Highlighting}
\end{Shaded}

\hypertarget{13420-switchmap}{%
\subparagraph{13.4.20 switchMap}\label{13420-switchmap}}

切换可观察对象。

\begin{figure}
\centering
\includegraphics{C:/Users/ZSH/Desktop/ng/ppt/images/46.png}
\caption{}
\end{figure}

\begin{Shaded}
\begin{Highlighting}[]
\ImportTok{import}\NormalTok{ \{ fromEvent}\OperatorTok{,}\NormalTok{ interval \} }\ImportTok{from} \StringTok{"rxjs"}
\ImportTok{import}\NormalTok{ \{ switchMap \} }\ImportTok{from} \StringTok{"rxjs/operators"}

\FunctionTok{fromEvent}\NormalTok{(}\BuiltInTok{document}\OperatorTok{,} \StringTok{"click"}\NormalTok{)}
  \OperatorTok{.}\FunctionTok{pipe}\NormalTok{(}\FunctionTok{switchMap}\NormalTok{(ev }\KeywordTok{=\textgreater{}} \FunctionTok{interval}\NormalTok{(}\DecValTok{1000}\NormalTok{)))}
  \OperatorTok{.}\FunctionTok{subscribe}\NormalTok{(x }\KeywordTok{=\textgreater{}} \BuiltInTok{console}\OperatorTok{.}\FunctionTok{log}\NormalTok{(x))}
\end{Highlighting}
\end{Shaded}

\hypertarget{135-ux7ec3ux4e60}{%
\paragraph{13.5 练习}\label{135-ux7ec3ux4e60}}

\hypertarget{1351-ux5143ux7d20ux62d6ux62fd}{%
\subparagraph{13.5.1 元素拖拽}\label{1351-ux5143ux7d20ux62d6ux62fd}}

\begin{Shaded}
\begin{Highlighting}[]
\KeywordTok{\textless{}style\textgreater{}}
  \PreprocessorTok{\#box}\NormalTok{ \{}
    \KeywordTok{width}\NormalTok{: }\DecValTok{200}\DataTypeTok{px}\OperatorTok{;}
    \KeywordTok{height}\NormalTok{: }\DecValTok{200}\DataTypeTok{px}\OperatorTok{;}
    \KeywordTok{background}\NormalTok{: }\ConstantTok{skyblue}\OperatorTok{;}
    \KeywordTok{position}\NormalTok{: }\DecValTok{absolute}\OperatorTok{;}
    \KeywordTok{left}\NormalTok{: }\DecValTok{0}\OperatorTok{;}
    \KeywordTok{top}\NormalTok{: }\DecValTok{0}\OperatorTok{;}
\NormalTok{  \}}
\KeywordTok{\textless{}/style\textgreater{}}
\KeywordTok{\textless{}div}\OtherTok{ id=}\StringTok{"box"}\KeywordTok{\textgreater{}\textless{}/div\textgreater{}}
\end{Highlighting}
\end{Shaded}

\begin{Shaded}
\begin{Highlighting}[]
\CommentTok{// 原生 JavaScript}
\NormalTok{box}\OperatorTok{.}\AttributeTok{onmousedown} \OperatorTok{=} \KeywordTok{function}\NormalTok{ (}\BuiltInTok{event}\NormalTok{) \{}
  \KeywordTok{let}\NormalTok{ distanceX }\OperatorTok{=} \BuiltInTok{event}\OperatorTok{.}\AttributeTok{clientX} \OperatorTok{{-}} \BuiltInTok{event}\OperatorTok{.}\AttributeTok{target}\OperatorTok{.}\AttributeTok{offsetLeft}
  \KeywordTok{let}\NormalTok{ distanceY }\OperatorTok{=} \BuiltInTok{event}\OperatorTok{.}\AttributeTok{clientY} \OperatorTok{{-}} \BuiltInTok{event}\OperatorTok{.}\AttributeTok{target}\OperatorTok{.}\AttributeTok{offsetTop}
  \BuiltInTok{document}\OperatorTok{.}\AttributeTok{onmousemove} \OperatorTok{=} \KeywordTok{function}\NormalTok{ (}\BuiltInTok{event}\NormalTok{) \{}
    \KeywordTok{let}\NormalTok{ positionX }\OperatorTok{=} \BuiltInTok{event}\OperatorTok{.}\AttributeTok{clientX} \OperatorTok{{-}}\NormalTok{ distanceX}
    \KeywordTok{let}\NormalTok{ positionY }\OperatorTok{=} \BuiltInTok{event}\OperatorTok{.}\AttributeTok{clientY} \OperatorTok{{-}}\NormalTok{ distanceY}
\NormalTok{    box}\OperatorTok{.}\AttributeTok{style}\OperatorTok{.}\AttributeTok{left} \OperatorTok{=}\NormalTok{ positionX }\OperatorTok{+} \StringTok{"px"}
\NormalTok{    box}\OperatorTok{.}\AttributeTok{style}\OperatorTok{.}\AttributeTok{top} \OperatorTok{=}\NormalTok{ positionY }\OperatorTok{+} \StringTok{"px"}
\NormalTok{  \}}
\NormalTok{  box}\OperatorTok{.}\AttributeTok{onmouseup} \OperatorTok{=} \KeywordTok{function}\NormalTok{ () \{}
    \BuiltInTok{document}\OperatorTok{.}\AttributeTok{onmousemove} \OperatorTok{=} \KeywordTok{null}
\NormalTok{  \}}
\NormalTok{\}}
\end{Highlighting}
\end{Shaded}

\begin{Shaded}
\begin{Highlighting}[]
\CommentTok{// RxJS}
\ImportTok{import}\NormalTok{ \{ fromEvent \} }\ImportTok{from} \StringTok{"rxjs"}
\ImportTok{import}\NormalTok{ \{ map}\OperatorTok{,}\NormalTok{ switchMap}\OperatorTok{,}\NormalTok{ takeUntil \} }\ImportTok{from} \StringTok{"rxjs/operators"}

\KeywordTok{const}\NormalTok{ box }\OperatorTok{=} \BuiltInTok{document}\OperatorTok{.}\FunctionTok{getElementById}\NormalTok{(}\StringTok{"box"}\NormalTok{)}

\FunctionTok{fromEvent}\NormalTok{(box}\OperatorTok{,} \StringTok{"mousedown"}\NormalTok{)}
  \OperatorTok{.}\FunctionTok{pipe}\NormalTok{(}
    \FunctionTok{map}\NormalTok{(}\BuiltInTok{event} \KeywordTok{=\textgreater{}}\NormalTok{ (\{}
      \DataTypeTok{distanceX}\OperatorTok{:} \BuiltInTok{event}\OperatorTok{.}\AttributeTok{clientX} \OperatorTok{{-}} \BuiltInTok{event}\OperatorTok{.}\AttributeTok{target}\OperatorTok{.}\AttributeTok{offsetLeft}\OperatorTok{,}
      \DataTypeTok{distanceY}\OperatorTok{:} \BuiltInTok{event}\OperatorTok{.}\AttributeTok{clientY} \OperatorTok{{-}} \BuiltInTok{event}\OperatorTok{.}\AttributeTok{target}\OperatorTok{.}\AttributeTok{offsetTop}
\NormalTok{    \}))}\OperatorTok{,}
    \FunctionTok{switchMap}\NormalTok{((\{ distanceX}\OperatorTok{,}\NormalTok{ distanceY \}) }\KeywordTok{=\textgreater{}}
      \FunctionTok{fromEvent}\NormalTok{(}\BuiltInTok{document}\OperatorTok{,} \StringTok{"mousemove"}\NormalTok{)}\OperatorTok{.}\FunctionTok{pipe}\NormalTok{(}
        \FunctionTok{map}\NormalTok{(}\BuiltInTok{event} \KeywordTok{=\textgreater{}}\NormalTok{ (\{}
          \DataTypeTok{positionX}\OperatorTok{:} \BuiltInTok{event}\OperatorTok{.}\AttributeTok{clientX} \OperatorTok{{-}}\NormalTok{ distanceX}\OperatorTok{,}
          \DataTypeTok{positionY}\OperatorTok{:} \BuiltInTok{event}\OperatorTok{.}\AttributeTok{clientY} \OperatorTok{{-}}\NormalTok{ distanceY}
\NormalTok{        \}))}\OperatorTok{,}
        \FunctionTok{takeUntil}\NormalTok{(}\FunctionTok{fromEvent}\NormalTok{(}\BuiltInTok{document}\OperatorTok{,} \StringTok{"mouseup"}\NormalTok{))}
\NormalTok{      )}
\NormalTok{    )}
\NormalTok{  )}
  \OperatorTok{.}\FunctionTok{subscribe}\NormalTok{((\{ positionX}\OperatorTok{,}\NormalTok{ positionY \}) }\KeywordTok{=\textgreater{}}\NormalTok{ \{}
\NormalTok{    box}\OperatorTok{.}\AttributeTok{style}\OperatorTok{.}\AttributeTok{left} \OperatorTok{=}\NormalTok{ positionX }\OperatorTok{+} \StringTok{"px"}
\NormalTok{    box}\OperatorTok{.}\AttributeTok{style}\OperatorTok{.}\AttributeTok{top} \OperatorTok{=}\NormalTok{ positionY }\OperatorTok{+} \StringTok{"px"}
\NormalTok{  \})}
\end{Highlighting}
\end{Shaded}

\hypertarget{1352-ux641cux7d22}{%
\subparagraph{13.5.2 搜索}\label{1352-ux641cux7d22}}

\begin{Shaded}
\begin{Highlighting}[]
\KeywordTok{\textless{}input}\OtherTok{ id=}\StringTok{"search"}\OtherTok{ type=}\StringTok{"text"}\OtherTok{ placeholder=}\StringTok{"请输入搜索内容..."} \KeywordTok{/\textgreater{}}
\end{Highlighting}
\end{Shaded}

\begin{Shaded}
\begin{Highlighting}[]
\ImportTok{import}\NormalTok{ \{ fromEvent}\OperatorTok{,} \ImportTok{from}\OperatorTok{,}\NormalTok{ throwError \} }\ImportTok{from} \StringTok{"rxjs"}
\ImportTok{import}\NormalTok{ \{ debounceTime}\OperatorTok{,}\NormalTok{ distinctUntilChanged}\OperatorTok{,}\NormalTok{ map}\OperatorTok{,}\NormalTok{ switchMap}\OperatorTok{,}\NormalTok{ catchError \} }\ImportTok{from} \StringTok{"rxjs/operators"}
\ImportTok{import}\NormalTok{ axios }\ImportTok{from} \StringTok{"axios"}

\KeywordTok{const}\NormalTok{ search }\OperatorTok{=} \BuiltInTok{document}\OperatorTok{.}\FunctionTok{getElementById}\NormalTok{(}\StringTok{"search"}\NormalTok{)}

\FunctionTok{fromEvent}\NormalTok{(search}\OperatorTok{,} \StringTok{"keyup"}\NormalTok{)}
  \OperatorTok{.}\FunctionTok{pipe}\NormalTok{(}
    \FunctionTok{debounceTime}\NormalTok{(}\DecValTok{700}\NormalTok{)}\OperatorTok{,}
    \FunctionTok{map}\NormalTok{(}\BuiltInTok{event} \KeywordTok{=\textgreater{}} \BuiltInTok{event}\OperatorTok{.}\AttributeTok{target}\OperatorTok{.}\AttributeTok{value}\NormalTok{)}\OperatorTok{,}
    \FunctionTok{distinctUntilChanged}\NormalTok{()}\OperatorTok{,}
    \FunctionTok{switchMap}\NormalTok{(keyword }\KeywordTok{=\textgreater{}}
      \ImportTok{from}\NormalTok{(}
\NormalTok{        axios}\OperatorTok{.}\FunctionTok{get}\NormalTok{(}\VerbatimStringTok{\textasciigrave{}https://j1sonplaceholder.typicode.com/posts?q=}\SpecialCharTok{$\{}\NormalTok{keyword}\SpecialCharTok{\}}\VerbatimStringTok{\textasciigrave{}}\NormalTok{)}
\NormalTok{      )}\OperatorTok{.}\FunctionTok{pipe}\NormalTok{(}
        \FunctionTok{map}\NormalTok{(response }\KeywordTok{=\textgreater{}}\NormalTok{ response}\OperatorTok{.}\AttributeTok{data}\NormalTok{)}\OperatorTok{,}
        \FunctionTok{catchError}\NormalTok{(error }\KeywordTok{=\textgreater{}} \FunctionTok{throwError}\NormalTok{(}\VerbatimStringTok{\textasciigrave{}发生了错误: }\SpecialCharTok{$\{}\NormalTok{error}\OperatorTok{.}\AttributeTok{message}\SpecialCharTok{\}}\VerbatimStringTok{\textasciigrave{}}\NormalTok{))}
\NormalTok{      )}
\NormalTok{    )}
\NormalTok{  )}
  \OperatorTok{.}\FunctionTok{subscribe}\NormalTok{(\{}
    \DataTypeTok{next}\OperatorTok{:}\NormalTok{ value }\KeywordTok{=\textgreater{}}\NormalTok{ \{}
      \BuiltInTok{console}\OperatorTok{.}\FunctionTok{log}\NormalTok{(value)}
\NormalTok{    \}}\OperatorTok{,}
    \DataTypeTok{error}\OperatorTok{:}\NormalTok{ error }\KeywordTok{=\textgreater{}}\NormalTok{ \{}
      \BuiltInTok{console}\OperatorTok{.}\FunctionTok{log}\NormalTok{(error)}
\NormalTok{    \}}
\NormalTok{  \})}
\end{Highlighting}
\end{Shaded}

\hypertarget{1353-ux4e32ux8054ux8bf7ux6c42}{%
\subparagraph{13.5.3 串联请求}\label{1353-ux4e32ux8054ux8bf7ux6c42}}

先获取token,再根据token获取用户信息

\begin{Shaded}
\begin{Highlighting}[]
\KeywordTok{\textless{}button}\OtherTok{ id=}\StringTok{"btn"}\KeywordTok{\textgreater{}}\NormalTok{获取用户信息}\KeywordTok{\textless{}/button\textgreater{}}
\end{Highlighting}
\end{Shaded}

\begin{Shaded}
\begin{Highlighting}[]
\ImportTok{import}\NormalTok{ axios }\ImportTok{from} \StringTok{"axios"}
\ImportTok{import}\NormalTok{ \{ }\ImportTok{from}\OperatorTok{,}\NormalTok{ fromEvent \} }\ImportTok{from} \StringTok{"rxjs"}
\ImportTok{import}\NormalTok{ \{ pluck}\OperatorTok{,}\NormalTok{ concatMap \} }\ImportTok{from} \StringTok{"rxjs/operators"}

\KeywordTok{const}\NormalTok{ button }\OperatorTok{=} \BuiltInTok{document}\OperatorTok{.}\FunctionTok{getElementById}\NormalTok{(}\StringTok{"btn"}\NormalTok{)}

\FunctionTok{fromEvent}\NormalTok{(button}\OperatorTok{,} \StringTok{"click"}\NormalTok{)}
  \OperatorTok{.}\FunctionTok{pipe}\NormalTok{(}
    \FunctionTok{concatMap}\NormalTok{(}\BuiltInTok{event} \KeywordTok{=\textgreater{}}
      \ImportTok{from}\NormalTok{(axios}\OperatorTok{.}\FunctionTok{get}\NormalTok{(}\StringTok{"http://localhost:3005/token"}\NormalTok{))}\OperatorTok{.}\FunctionTok{pipe}\NormalTok{(}
        \FunctionTok{pluck}\NormalTok{(}\StringTok{"data"}\OperatorTok{,} \StringTok{"token"}\NormalTok{)}
\NormalTok{      )}
\NormalTok{    )}\OperatorTok{,}
    \FunctionTok{concatMap}\NormalTok{(token }\KeywordTok{=\textgreater{}}
      \ImportTok{from}\NormalTok{(axios}\OperatorTok{.}\FunctionTok{get}\NormalTok{(}\StringTok{"http://localhost:3005/userInfo"}\NormalTok{))}\OperatorTok{.}\FunctionTok{pipe}\NormalTok{(}\FunctionTok{pluck}\NormalTok{(}\StringTok{"data"}\NormalTok{))}
\NormalTok{    )}
\NormalTok{  )}
  \OperatorTok{.}\FunctionTok{subscribe}\NormalTok{(}\BuiltInTok{console}\OperatorTok{.}\FunctionTok{log}\NormalTok{)}
\end{Highlighting}
\end{Shaded}

\hypertarget{14-httpclientmodule}{%
\subsubsection{14. HttpClientModule}\label{14-httpclientmodule}}

该模块用于发送 Http 请求,用于发送请求的方法都返回 Observable 对象。

\hypertarget{141-ux5febux901fux5f00ux59cb}{%
\paragraph{14.1 快速开始}\label{141-ux5febux901fux5f00ux59cb}}

\begin{enumerate}
\def\labelenumi{\arabic{enumi}.}
\item
  引入 HttpClientModule 模块

\begin{Shaded}
\begin{Highlighting}[]
\CommentTok{// app.module.ts}
\ImportTok{import}\NormalTok{ \{ httpClientModule \} }\ImportTok{from} \StringTok{\textquotesingle{}@angular/common/http\textquotesingle{}}\OperatorTok{;}
\NormalTok{imports}\OperatorTok{:}\NormalTok{ [}
\NormalTok{  httpClientModule}
\NormalTok{]}
\end{Highlighting}
\end{Shaded}
\item
  注入 HttpClient 服务实例对象,用于发送请求

\begin{Shaded}
\begin{Highlighting}[]
\CommentTok{// app.component.ts}
\ImportTok{import}\NormalTok{ \{ HttpClient \} }\ImportTok{from} \StringTok{\textquotesingle{}@angular/common/http\textquotesingle{}}\OperatorTok{;}

\ImportTok{export} \KeywordTok{class}\NormalTok{ AppComponent \{}
	\FunctionTok{constructor}\NormalTok{(}\KeywordTok{private} \DataTypeTok{http}\OperatorTok{:}\NormalTok{ HttpClient) \{\}}
\NormalTok{\}}
\end{Highlighting}
\end{Shaded}
\item
  发送请求

\begin{Shaded}
\begin{Highlighting}[]
\ImportTok{import}\NormalTok{ \{ HttpClient \} }\ImportTok{from} \StringTok{"@angular/common/http"}

\ImportTok{export} \KeywordTok{class}\NormalTok{ AppComponent }\KeywordTok{implements}\NormalTok{ OnInit \{}
  \FunctionTok{constructor}\NormalTok{(}\KeywordTok{private} \DataTypeTok{http}\OperatorTok{:}\NormalTok{ HttpClient) \{\}}
  \FunctionTok{ngOnInit}\NormalTok{() \{}
    \KeywordTok{this}\OperatorTok{.}\FunctionTok{getUsers}\NormalTok{()}\OperatorTok{.}\FunctionTok{subscribe}\NormalTok{(}\BuiltInTok{console}\OperatorTok{.}\FunctionTok{log}\NormalTok{)}
\NormalTok{  \}}
  \FunctionTok{getUsers}\NormalTok{() \{}
    \ControlFlowTok{return} \KeywordTok{this}\OperatorTok{.}\AttributeTok{http}\OperatorTok{.}\FunctionTok{get}\NormalTok{(}\StringTok{"https://jsonplaceholder.typicode.com/users"}\NormalTok{)}
\NormalTok{  \}}
\NormalTok{\}}
\end{Highlighting}
\end{Shaded}
\end{enumerate}

\hypertarget{142--ux8bf7ux6c42ux65b9ux6cd5}{%
\paragraph{14.2 请求方法}\label{142--ux8bf7ux6c42ux65b9ux6cd5}}

\begin{Shaded}
\begin{Highlighting}[]
\KeywordTok{this}\OperatorTok{.}\AttributeTok{http}\OperatorTok{.}\FunctionTok{get}\NormalTok{(url [}\OperatorTok{,}\NormalTok{ options])}\OperatorTok{;}
\KeywordTok{this}\OperatorTok{.}\AttributeTok{http}\OperatorTok{.}\FunctionTok{post}\NormalTok{(url}\OperatorTok{,}\NormalTok{ data [}\OperatorTok{,}\NormalTok{ options])}\OperatorTok{;}
\KeywordTok{this}\OperatorTok{.}\AttributeTok{http}\OperatorTok{.}\FunctionTok{delete}\NormalTok{(url [}\OperatorTok{,}\NormalTok{ options])}\OperatorTok{;}
\KeywordTok{this}\OperatorTok{.}\AttributeTok{http}\OperatorTok{.}\FunctionTok{put}\NormalTok{(url}\OperatorTok{,}\NormalTok{ data [}\OperatorTok{,}\NormalTok{ options])}\OperatorTok{;}
\end{Highlighting}
\end{Shaded}

\begin{Shaded}
\begin{Highlighting}[]
\KeywordTok{this}\OperatorTok{.}\AttributeTok{http}\OperatorTok{.}\AttributeTok{get}\OperatorTok{\textless{}}\NormalTok{Post[]}\OperatorTok{\textgreater{}}\NormalTok{(}\StringTok{\textquotesingle{}/getAllPosts\textquotesingle{}}\NormalTok{)}
  \OperatorTok{.}\FunctionTok{subscribe}\NormalTok{(response }\KeywordTok{=\textgreater{}} \BuiltInTok{console}\OperatorTok{.}\FunctionTok{log}\NormalTok{(response))}
\end{Highlighting}
\end{Shaded}

\hypertarget{143-ux8bf7ux6c42ux53c2ux6570}{%
\paragraph{14.3 请求参数}\label{143-ux8bf7ux6c42ux53c2ux6570}}

\begin{enumerate}
\def\labelenumi{\arabic{enumi}.}
\item
  HttpParams 类

\begin{Shaded}
\begin{Highlighting}[]
\ImportTok{export}\NormalTok{ declare }\KeywordTok{class}\NormalTok{ HttpParams \{}
    \FunctionTok{constructor}\NormalTok{(options}\OperatorTok{?:}\NormalTok{ HttpParamsOptions)}\OperatorTok{;}
    \FunctionTok{has}\NormalTok{(}\DataTypeTok{param}\OperatorTok{:}\NormalTok{ string)}\OperatorTok{:}\NormalTok{ boolean}\OperatorTok{;}
    \FunctionTok{get}\NormalTok{(}\DataTypeTok{param}\OperatorTok{:}\NormalTok{ string)}\OperatorTok{:}\NormalTok{ string }\OperatorTok{|} \KeywordTok{null}\OperatorTok{;}
    \FunctionTok{getAll}\NormalTok{(}\DataTypeTok{param}\OperatorTok{:}\NormalTok{ string)}\OperatorTok{:}\NormalTok{ string[] }\OperatorTok{|} \KeywordTok{null}\OperatorTok{;}
    \FunctionTok{keys}\NormalTok{()}\OperatorTok{:}\NormalTok{ string[]}\OperatorTok{;}
    \FunctionTok{append}\NormalTok{(}\DataTypeTok{param}\OperatorTok{:}\NormalTok{ string}\OperatorTok{,} \DataTypeTok{value}\OperatorTok{:}\NormalTok{ string)}\OperatorTok{:}\NormalTok{ HttpParams}\OperatorTok{;}
    \FunctionTok{set}\NormalTok{(}\DataTypeTok{param}\OperatorTok{:}\NormalTok{ string}\OperatorTok{,} \DataTypeTok{value}\OperatorTok{:}\NormalTok{ string)}\OperatorTok{:}\NormalTok{ HttpParams}\OperatorTok{;}
    \KeywordTok{delete}\NormalTok{(}\DataTypeTok{param}\OperatorTok{:}\NormalTok{ string}\OperatorTok{,}\NormalTok{ value}\OperatorTok{?:}\NormalTok{ string)}\OperatorTok{:}\NormalTok{ HttpParams}\OperatorTok{;}
    \FunctionTok{toString}\NormalTok{()}\OperatorTok{:}\NormalTok{ string}\OperatorTok{;}
\NormalTok{\}}
\end{Highlighting}
\end{Shaded}
\item
  HttpParamsOptions 接口

\begin{Shaded}
\begin{Highlighting}[]
\NormalTok{declare }\KeywordTok{interface}\NormalTok{ HttpParamsOptions \{}
\NormalTok{    fromString}\OperatorTok{?:}\NormalTok{ string}\OperatorTok{;}
\NormalTok{    fromObject}\OperatorTok{?:}\NormalTok{ \{}
\NormalTok{        [}\DataTypeTok{param}\OperatorTok{:}\NormalTok{ string]}\OperatorTok{:}\NormalTok{ string }\OperatorTok{|}\NormalTok{ ReadonlyArray}\OperatorTok{\textless{}}\NormalTok{string}\OperatorTok{\textgreater{};}
\NormalTok{    \}}\OperatorTok{;}
\NormalTok{    encoder}\OperatorTok{?:}\NormalTok{ HttpParameterCodec}\OperatorTok{;}
\NormalTok{\}}
\end{Highlighting}
\end{Shaded}
\item
  使用示例

\begin{Shaded}
\begin{Highlighting}[]
\ImportTok{import}\NormalTok{ \{ HttpParams \} }\ImportTok{from} \StringTok{\textquotesingle{}@angular/common/http\textquotesingle{}}\OperatorTok{;}

\KeywordTok{let}\NormalTok{ params }\OperatorTok{=} \KeywordTok{new} \FunctionTok{HttpParams}\NormalTok{(\{ }\DataTypeTok{fromObject}\OperatorTok{:}\NormalTok{ \{}\DataTypeTok{name}\OperatorTok{:} \StringTok{"zhangsan"}\OperatorTok{,} \DataTypeTok{age}\OperatorTok{:} \StringTok{"20"}\NormalTok{\}\})}
\NormalTok{params }\OperatorTok{=}\NormalTok{ params}\OperatorTok{.}\FunctionTok{append}\NormalTok{(}\StringTok{"sex"}\OperatorTok{,} \StringTok{"male"}\NormalTok{)}
\KeywordTok{let}\NormalTok{ params }\OperatorTok{=} \KeywordTok{new} \FunctionTok{HttpParams}\NormalTok{(\{ }\DataTypeTok{fromString}\OperatorTok{:} \StringTok{"name=zhangsan\&age=20"}\NormalTok{\})}
\end{Highlighting}
\end{Shaded}
\end{enumerate}

\hypertarget{144-ux8bf7ux6c42ux5934}{%
\paragraph{14.4 请求头}\label{144-ux8bf7ux6c42ux5934}}

请求头字段的创建需要使用 HttpHeaders
类,在类实例对象下面有各种操作请求头的方法。

\begin{Shaded}
\begin{Highlighting}[]
\ImportTok{export}\NormalTok{ declare }\KeywordTok{class}\NormalTok{ HttpHeaders \{}
    \FunctionTok{constructor}\NormalTok{(headers}\OperatorTok{?:}\NormalTok{ string }\OperatorTok{|}\NormalTok{ \{}
\NormalTok{        [}\DataTypeTok{name}\OperatorTok{:}\NormalTok{ string]}\OperatorTok{:}\NormalTok{ string }\OperatorTok{|}\NormalTok{ string[]}\OperatorTok{;}
\NormalTok{    \})}\OperatorTok{;}
    \FunctionTok{has}\NormalTok{(}\DataTypeTok{name}\OperatorTok{:}\NormalTok{ string)}\OperatorTok{:}\NormalTok{ boolean}\OperatorTok{;}
    \FunctionTok{get}\NormalTok{(}\DataTypeTok{name}\OperatorTok{:}\NormalTok{ string)}\OperatorTok{:}\NormalTok{ string }\OperatorTok{|} \KeywordTok{null}\OperatorTok{;}
    \FunctionTok{keys}\NormalTok{()}\OperatorTok{:}\NormalTok{ string[]}\OperatorTok{;}
    \FunctionTok{getAll}\NormalTok{(}\DataTypeTok{name}\OperatorTok{:}\NormalTok{ string)}\OperatorTok{:}\NormalTok{ string[] }\OperatorTok{|} \KeywordTok{null}\OperatorTok{;}
    \FunctionTok{append}\NormalTok{(}\DataTypeTok{name}\OperatorTok{:}\NormalTok{ string}\OperatorTok{,} \DataTypeTok{value}\OperatorTok{:}\NormalTok{ string }\OperatorTok{|}\NormalTok{ string[])}\OperatorTok{:}\NormalTok{ HttpHeaders}\OperatorTok{;}
    \FunctionTok{set}\NormalTok{(}\DataTypeTok{name}\OperatorTok{:}\NormalTok{ string}\OperatorTok{,} \DataTypeTok{value}\OperatorTok{:}\NormalTok{ string }\OperatorTok{|}\NormalTok{ string[])}\OperatorTok{:}\NormalTok{ HttpHeaders}\OperatorTok{;}
    \KeywordTok{delete}\NormalTok{(}\DataTypeTok{name}\OperatorTok{:}\NormalTok{ string}\OperatorTok{,}\NormalTok{ value}\OperatorTok{?:}\NormalTok{ string }\OperatorTok{|}\NormalTok{ string[])}\OperatorTok{:}\NormalTok{ HttpHeaders}\OperatorTok{;}
\NormalTok{\}}
\end{Highlighting}
\end{Shaded}

\begin{Shaded}
\begin{Highlighting}[]
\KeywordTok{let}\NormalTok{ headers }\OperatorTok{=} \KeywordTok{new} \FunctionTok{HttpHeaders}\NormalTok{(\{ }\DataTypeTok{test}\OperatorTok{:} \StringTok{"Hello"}\NormalTok{ \})}
\end{Highlighting}
\end{Shaded}

\hypertarget{145-ux54cdux5e94ux5185ux5bb9}{%
\paragraph{14.5 响应内容}\label{145-ux54cdux5e94ux5185ux5bb9}}

\begin{Shaded}
\begin{Highlighting}[]
\NormalTok{declare type HttpObserve }\OperatorTok{=} \StringTok{\textquotesingle{}body\textquotesingle{}} \OperatorTok{|} \StringTok{\textquotesingle{}response\textquotesingle{}}\OperatorTok{;}
\CommentTok{// response 读取完整响应体}
\CommentTok{// body 读取服务器端返回的数据}
\end{Highlighting}
\end{Shaded}

\begin{Shaded}
\begin{Highlighting}[]
\KeywordTok{this}\OperatorTok{.}\AttributeTok{http}\OperatorTok{.}\FunctionTok{get}\NormalTok{(}
  \StringTok{"https://jsonplaceholder.typicode.com/users"}\OperatorTok{,} 
\NormalTok{  \{ }\DataTypeTok{observe}\OperatorTok{:} \StringTok{"body"}\NormalTok{ \}}
\NormalTok{)}\OperatorTok{.}\FunctionTok{subscribe}\NormalTok{(}\BuiltInTok{console}\OperatorTok{.}\FunctionTok{log}\NormalTok{)}
\end{Highlighting}
\end{Shaded}

\hypertarget{146-ux62e6ux622aux5668}{%
\paragraph{14.6 拦截器}\label{146-ux62e6ux622aux5668}}

拦截器是 Angular 应用中全局捕获和修改 HTTP
请求和响应的方式。(Token、Error)

拦截器将只拦截使用 HttpClientModule 模块发出的请求。

\texttt{ng\ g\ interceptor\ \textless{}name\textgreater{}}

\begin{figure}
\centering
\includegraphics{C:/Users/ZSH/Desktop/ng/ppt/images/47.png}
\caption{}
\end{figure}

\begin{figure}
\centering
\includegraphics{C:/Users/ZSH/Desktop/ng/ppt/images/48.png}
\caption{}
\end{figure}

\hypertarget{1461-ux8bf7ux6c42ux62e6ux622a}{%
\subparagraph{14.6.1 请求拦截}\label{1461-ux8bf7ux6c42ux62e6ux622a}}

\begin{Shaded}
\begin{Highlighting}[]
\NormalTok{@}\FunctionTok{Injectable}\NormalTok{()}
\ImportTok{export} \KeywordTok{class}\NormalTok{ AuthInterceptor }\KeywordTok{implements}\NormalTok{ HttpInterceptor \{}
  \FunctionTok{constructor}\NormalTok{() \{\}}
	\CommentTok{// 拦截方法}
  \FunctionTok{intercept}\NormalTok{(}
  	\CommentTok{// unknown 指定请求体 (body) 的类型}
    \DataTypeTok{request}\OperatorTok{:}\NormalTok{ HttpRequest}\OperatorTok{\textless{}}\NormalTok{unknown}\OperatorTok{\textgreater{},}
    \DataTypeTok{next}\OperatorTok{:}\NormalTok{ HttpHandler}
     \CommentTok{// unknown 指定响应内容 (body) 的类型}
\NormalTok{  )}\OperatorTok{:}\NormalTok{ Observable}\OperatorTok{\textless{}}\NormalTok{HttpEvent}\OperatorTok{\textless{}}\NormalTok{unknown}\OperatorTok{\textgreater{}\textgreater{}}\NormalTok{ \{}
    \CommentTok{// 克隆并修改请求头}
    \KeywordTok{const}\NormalTok{ req }\OperatorTok{=}\NormalTok{ request}\OperatorTok{.}\FunctionTok{clone}\NormalTok{(\{}
      \DataTypeTok{setHeaders}\OperatorTok{:}\NormalTok{ \{}
        \DataTypeTok{Authorization}\OperatorTok{:} \StringTok{"Bearer xxxxxxx"}
\NormalTok{      \}}
\NormalTok{    \})}
    \CommentTok{// 通过回调函数将修改后的请求头回传给应用}
    \ControlFlowTok{return}\NormalTok{ next}\OperatorTok{.}\FunctionTok{handle}\NormalTok{(req)}
\NormalTok{  \}}
\NormalTok{\}}
\end{Highlighting}
\end{Shaded}

\hypertarget{1462-ux54cdux5e94ux62e6ux622a}{%
\subparagraph{14.6.2 响应拦截}\label{1462-ux54cdux5e94ux62e6ux622a}}

\begin{Shaded}
\begin{Highlighting}[]
\NormalTok{@}\FunctionTok{Injectable}\NormalTok{()}
\ImportTok{export} \KeywordTok{class}\NormalTok{ AuthInterceptor }\KeywordTok{implements}\NormalTok{ HttpInterceptor \{}
  \FunctionTok{constructor}\NormalTok{() \{\}}
	\CommentTok{// 拦截方法}
  \FunctionTok{intercept}\NormalTok{(}
    \DataTypeTok{request}\OperatorTok{:}\NormalTok{ HttpRequest}\OperatorTok{\textless{}}\NormalTok{unknown}\OperatorTok{\textgreater{},}
    \DataTypeTok{next}\OperatorTok{:}\NormalTok{ HttpHandler}
\NormalTok{  )}\OperatorTok{:}\NormalTok{ Observable}\OperatorTok{\textless{}}\NormalTok{any}\OperatorTok{\textgreater{}}\NormalTok{ \{}
    \ControlFlowTok{return}\NormalTok{ next}\OperatorTok{.}\FunctionTok{handle}\NormalTok{(request)}\OperatorTok{.}\FunctionTok{pipe}\NormalTok{(}
      \FunctionTok{retry}\NormalTok{(}\DecValTok{2}\NormalTok{)}\OperatorTok{,}
      \FunctionTok{catchError}\NormalTok{((}\DataTypeTok{error}\OperatorTok{:}\NormalTok{ HttpErrorResponse) }\KeywordTok{=\textgreater{}} \FunctionTok{throwError}\NormalTok{(error))}
\NormalTok{    )}
\NormalTok{  \}}
\NormalTok{\}}
\end{Highlighting}
\end{Shaded}

\hypertarget{1453-ux62e6ux622aux5668ux6ce8ux5165}{%
\subparagraph{14.5.3
拦截器注入}\label{1453-ux62e6ux622aux5668ux6ce8ux5165}}

\begin{Shaded}
\begin{Highlighting}[]
\ImportTok{import}\NormalTok{ \{ AuthInterceptor \} }\ImportTok{from} \StringTok{"./auth.interceptor"}
\ImportTok{import}\NormalTok{ \{ HTTP\_INTERCEPTORS \} }\ImportTok{from} \StringTok{"@angular/common/http"}

\NormalTok{@}\FunctionTok{NgModule}\NormalTok{(\{}
  \DataTypeTok{providers}\OperatorTok{:}\NormalTok{ [}
\NormalTok{    \{}
      \DataTypeTok{provide}\OperatorTok{:}\NormalTok{ HTTP\_INTERCEPTORS}\OperatorTok{,}
      \DataTypeTok{useClass}\OperatorTok{:}\NormalTok{ AuthInterceptor}\OperatorTok{,}
      \DataTypeTok{multi}\OperatorTok{:} \KeywordTok{true}
\NormalTok{    \}}
\NormalTok{  ]}
\NormalTok{\})}
\end{Highlighting}
\end{Shaded}

\hypertarget{147-angular-proxy}{%
\paragraph{14.7 Angular Proxy}\label{147-angular-proxy}}

\begin{enumerate}
\def\labelenumi{\arabic{enumi}.}
\item
  在项目的根目录下创建 proxy.conf.json 文件并加入如下代码

\begin{Shaded}
\begin{Highlighting}[]
\FunctionTok{\{}
 	\DataTypeTok{"/api/*"}\FunctionTok{:} \FunctionTok{\{}
    \DataTypeTok{"target"}\FunctionTok{:} \StringTok{"http://localhost:3070"}\FunctionTok{,}
    \DataTypeTok{"secure"}\FunctionTok{:} \KeywordTok{false}\FunctionTok{,}
    \DataTypeTok{"changeOrigin"}\FunctionTok{:} \KeywordTok{true}
  \FunctionTok{\}}
\FunctionTok{\}}
\end{Highlighting}
\end{Shaded}

  \begin{enumerate}
  \def\labelenumii{\arabic{enumii}.}
  \item
    /api/*:在应用中发出的以 /api 开头的请求走此代理
  \item
    target:服务器端 URL
  \item
    secure:如果服务器端 URL 的协议是 https,此项需要为 true
  \item
    changeOrigin:如果服务器端不是 localhost, 此项需要为 true
  \end{enumerate}
\item
  指定 proxy 配置文件 (方式一)

\begin{Shaded}
\begin{Highlighting}[]
\StringTok{"scripts"}\OperatorTok{:}\NormalTok{ \{}
  \StringTok{"start"}\OperatorTok{:} \StringTok{"ng serve {-}{-}proxy{-}config proxy.conf.json"}\OperatorTok{,}
\NormalTok{\}}
\end{Highlighting}
\end{Shaded}
\item
  指定 proxy 配置文件 (方式二)

\begin{Shaded}
\begin{Highlighting}[]
\ErrorTok{"serve":} \FunctionTok{\{}
  \DataTypeTok{"options"}\FunctionTok{:} \FunctionTok{\{}
    \DataTypeTok{"proxyConfig"}\FunctionTok{:} \StringTok{"proxy.conf.json"}
  \FunctionTok{\},}
\end{Highlighting}
\end{Shaded}
\end{enumerate}

\hypertarget{15-ngrx}{%
\subsubsection{15. NgRx}\label{15-ngrx}}

\hypertarget{151-ux6982ux8ff0}{%
\paragraph{15.1 概述}\label{151-ux6982ux8ff0}}

NgRx 是 Angular 应用中实现全局状态管理的 Redux 架构解决方案。

\begin{figure}
\centering
\includegraphics{C:/Users/ZSH/Desktop/ng/ppt/images/49.png}
\caption{}
\end{figure}

\begin{enumerate}
\def\labelenumi{\arabic{enumi}.}
\item
  @ngrx/store:全局状态管理模块
\item
  @ngrx/effects:处理副作用
\item
  @ngrx/store-devtools:浏览器调试工具,需要依赖
  \href{https://github.com/zalmoxisus/redux-devtools-extension/}{Redux
  Devtools Extension}
\item
  @ngrx/schematics:命令行工具,快速生成 NgRx 文件
\item
  @ngrx/entity:提高开发者在 Reducer 中操作数据的效率
\item
  @ngrx/router-store:将路由状态同步到全局 Store
\end{enumerate}

\hypertarget{152-ux5febux901fux5f00ux59cb}{%
\paragraph{15.2 快速开始}\label{152-ux5febux901fux5f00ux59cb}}

\begin{enumerate}
\def\labelenumi{\arabic{enumi}.}
\item
  下载 NgRx

  \texttt{npm\ install\ @ngrx/store\ @ngrx/effects\ @ngrx/entity\ @ngrx/router-store\ @ngrx/store-devtools\ @ngrx/schematics}
\item
  配置 NgRx CLI

  \texttt{ng\ config\ cli.defaultCollection\ @ngrx/schematics}

\begin{Shaded}
\begin{Highlighting}[]
\CommentTok{// angular.json}
\StringTok{"cli"}\OperatorTok{:}\NormalTok{ \{}
  \StringTok{"defaultCollection"}\OperatorTok{:} \StringTok{"@ngrx/schematics"}
\NormalTok{\}}
\end{Highlighting}
\end{Shaded}
\item
  创建 Store

  \texttt{ng\ g\ store\ State\ -\/-root\ -\/-module\ app.module.ts\ -\/-statePath\ store\ -\/-stateInterface\ AppState}
\item
  创建 Action

  \texttt{ng\ g\ action\ store/actions/counter\ -\/-skipTests}

\begin{Shaded}
\begin{Highlighting}[]
\ImportTok{import}\NormalTok{ \{ createAction \} }\ImportTok{from} \StringTok{"@ngrx/store"}

\ImportTok{export} \KeywordTok{const}\NormalTok{ increment }\OperatorTok{=} \FunctionTok{createAction}\NormalTok{(}\StringTok{"increment"}\NormalTok{)}
\ImportTok{export} \KeywordTok{const}\NormalTok{ decrement }\OperatorTok{=} \FunctionTok{createAction}\NormalTok{(}\StringTok{"decrement"}\NormalTok{)}
\end{Highlighting}
\end{Shaded}
\item
  创建 Reducer

  \texttt{ng\ g\ reducer\ store/reducers/counter\ -\/-skipTests\ -\/-reducers=../index.ts}

\begin{Shaded}
\begin{Highlighting}[]
\ImportTok{import}\NormalTok{ \{ createReducer}\OperatorTok{,}\NormalTok{ on \} }\ImportTok{from} \StringTok{"@ngrx/store"}
\ImportTok{import}\NormalTok{ \{ decrement}\OperatorTok{,}\NormalTok{ increment \} }\ImportTok{from} \StringTok{"../actions/counter.actions"}

\ImportTok{export} \KeywordTok{const}\NormalTok{ counterFeatureKey }\OperatorTok{=} \StringTok{"counter"}

\ImportTok{export} \KeywordTok{interface}\NormalTok{ State \{}
  \DataTypeTok{count}\OperatorTok{:}\NormalTok{ number}
\NormalTok{\}}

\ImportTok{export} \KeywordTok{const}\NormalTok{ initialState}\OperatorTok{:}\NormalTok{ State }\OperatorTok{=}\NormalTok{ \{}
  \DataTypeTok{count}\OperatorTok{:} \DecValTok{0}
\NormalTok{\}}

\ImportTok{export} \KeywordTok{const}\NormalTok{ reducer }\OperatorTok{=} \FunctionTok{createReducer}\NormalTok{(}
\NormalTok{  initialState}\OperatorTok{,}
  \FunctionTok{on}\NormalTok{(increment}\OperatorTok{,}\NormalTok{ state }\KeywordTok{=\textgreater{}}\NormalTok{ (\{ }\DataTypeTok{count}\OperatorTok{:}\NormalTok{ state}\OperatorTok{.}\AttributeTok{count} \OperatorTok{+} \DecValTok{1}\NormalTok{ \}))}\OperatorTok{,}
  \FunctionTok{on}\NormalTok{(decrement}\OperatorTok{,}\NormalTok{ state }\KeywordTok{=\textgreater{}}\NormalTok{ (\{ }\DataTypeTok{count}\OperatorTok{:}\NormalTok{ state}\OperatorTok{.}\AttributeTok{count} \OperatorTok{{-}} \DecValTok{1}\NormalTok{ \}))}
\NormalTok{)}
\end{Highlighting}
\end{Shaded}
\item
  创建 Selector

  \texttt{ng\ g\ selector\ store/selectors/counter\ -\/-skipTests}

\begin{Shaded}
\begin{Highlighting}[]
\ImportTok{import}\NormalTok{ \{ createFeatureSelector}\OperatorTok{,}\NormalTok{ createSelector \} }\ImportTok{from} \StringTok{"@ngrx/store"}
\ImportTok{import}\NormalTok{ \{ counterFeatureKey}\OperatorTok{,}\NormalTok{ State \} }\ImportTok{from} \StringTok{"../reducers/counter.reducer"}
\ImportTok{import}\NormalTok{ \{ AppState \} }\ImportTok{from} \StringTok{".."}

\ImportTok{export} \KeywordTok{const}\NormalTok{ selectCounter }\OperatorTok{=}\NormalTok{ createFeatureSelector}\OperatorTok{\textless{}}\NormalTok{AppState}\OperatorTok{,}\NormalTok{ State}\OperatorTok{\textgreater{}}\NormalTok{(counterFeatureKey)}
\ImportTok{export} \KeywordTok{const}\NormalTok{ selectCount }\OperatorTok{=} \FunctionTok{createSelector}\NormalTok{(selectCounter}\OperatorTok{,}\NormalTok{ state }\KeywordTok{=\textgreater{}}\NormalTok{ state}\OperatorTok{.}\AttributeTok{count}\NormalTok{)}
\end{Highlighting}
\end{Shaded}
\item
  组件类触发 Action、获取状态

\begin{Shaded}
\begin{Highlighting}[]
\ImportTok{import}\NormalTok{ \{ select}\OperatorTok{,}\NormalTok{ Store \} }\ImportTok{from} \StringTok{"@ngrx/store"}
\ImportTok{import}\NormalTok{ \{ Observable \} }\ImportTok{from} \StringTok{"rxjs"}
\ImportTok{import}\NormalTok{ \{ AppState \} }\ImportTok{from} \StringTok{"./store"}
\ImportTok{import}\NormalTok{ \{ decrement}\OperatorTok{,}\NormalTok{ increment \} }\ImportTok{from} \StringTok{"./store/actions/counter.actions"}
\ImportTok{import}\NormalTok{ \{ selectCount \} }\ImportTok{from} \StringTok{"./store/selectors/counter.selectors"}

\ImportTok{export} \KeywordTok{class}\NormalTok{ AppComponent \{}
  \DataTypeTok{count}\OperatorTok{:}\NormalTok{ Observable}\OperatorTok{\textless{}}\NormalTok{number}\OperatorTok{\textgreater{}}
  \FunctionTok{constructor}\NormalTok{(}\KeywordTok{private} \DataTypeTok{store}\OperatorTok{:}\NormalTok{ Store}\OperatorTok{\textless{}}\NormalTok{AppState}\OperatorTok{\textgreater{}}\NormalTok{) \{}
    \KeywordTok{this}\OperatorTok{.}\AttributeTok{count} \OperatorTok{=} \KeywordTok{this}\OperatorTok{.}\AttributeTok{store}\OperatorTok{.}\FunctionTok{pipe}\NormalTok{(}\FunctionTok{select}\NormalTok{(selectCount))}
\NormalTok{  \}}
  \FunctionTok{increment}\NormalTok{() \{}
    \KeywordTok{this}\OperatorTok{.}\AttributeTok{store}\OperatorTok{.}\FunctionTok{dispatch}\NormalTok{(}\FunctionTok{increment}\NormalTok{())}
\NormalTok{  \}}
  \FunctionTok{decrement}\NormalTok{() \{}
    \KeywordTok{this}\OperatorTok{.}\AttributeTok{store}\OperatorTok{.}\FunctionTok{dispatch}\NormalTok{(}\FunctionTok{decrement}\NormalTok{())}
\NormalTok{  \}}
\NormalTok{\}}
\end{Highlighting}
\end{Shaded}
\item
  组件模板显示状态

\begin{Shaded}
\begin{Highlighting}[]
\KeywordTok{\textless{}button}\OtherTok{ (click)=}\StringTok{"increment()"}\KeywordTok{\textgreater{}}\NormalTok{+}\KeywordTok{\textless{}/button\textgreater{}}
\KeywordTok{\textless{}span\textgreater{}}\NormalTok{\{\{ count | async \}\}}\KeywordTok{\textless{}/span\textgreater{}}
\KeywordTok{\textless{}button}\OtherTok{ (click)=}\StringTok{"decrement()"}\KeywordTok{\textgreater{}}\NormalTok{{-}}\KeywordTok{\textless{}/button\textgreater{}}
\end{Highlighting}
\end{Shaded}
\end{enumerate}

\hypertarget{153-action-payload}{%
\paragraph{15.3 Action Payload}\label{153-action-payload}}

\begin{enumerate}
\def\labelenumi{\arabic{enumi}.}
\item
  在组件中使用 dispatch 触发 Action 时传递参数,参数最终会被放置在
  Action 对象中。

\begin{Shaded}
\begin{Highlighting}[]
\KeywordTok{this}\OperatorTok{.}\AttributeTok{store}\OperatorTok{.}\FunctionTok{dispatch}\NormalTok{(}\FunctionTok{increment}\NormalTok{(\{ }\DataTypeTok{count}\OperatorTok{:} \DecValTok{5}\NormalTok{ \}))}
\end{Highlighting}
\end{Shaded}
\item
  在创建 Action Creator 函数时,获取参数并指定参数类型。

\begin{Shaded}
\begin{Highlighting}[]
\ImportTok{import}\NormalTok{ \{ createAction}\OperatorTok{,}\NormalTok{ props \} }\ImportTok{from} \StringTok{"@ngrx/store"}
\ImportTok{export} \KeywordTok{const}\NormalTok{ increment }\OperatorTok{=} \FunctionTok{createAction}\NormalTok{(}\StringTok{"increment"}\OperatorTok{,}\NormalTok{ props}\OperatorTok{\textless{}}\NormalTok{\{ }\DataTypeTok{count}\OperatorTok{:}\NormalTok{ number \}}\OperatorTok{\textgreater{}}\NormalTok{())}
\end{Highlighting}
\end{Shaded}

\begin{Shaded}
\begin{Highlighting}[]
\ImportTok{export}\NormalTok{ declare }\KeywordTok{function}\NormalTok{ props}\OperatorTok{\textless{}}\NormalTok{P }\KeywordTok{extends}\NormalTok{ object}\OperatorTok{\textgreater{}}\NormalTok{()}\OperatorTok{:}\NormalTok{ Props}\OperatorTok{\textless{}}\NormalTok{P}\OperatorTok{\textgreater{};}
\end{Highlighting}
\end{Shaded}
\item
  在 Reducer 中通过 Action 对象获取参数。

\begin{Shaded}
\begin{Highlighting}[]
\ImportTok{export} \KeywordTok{const}\NormalTok{ reducer }\OperatorTok{=} \FunctionTok{createReducer}\NormalTok{(}
\NormalTok{  initialState}\OperatorTok{,}
  \FunctionTok{on}\NormalTok{(increment}\OperatorTok{,}\NormalTok{ (state}\OperatorTok{,}\NormalTok{ action) }\KeywordTok{=\textgreater{}}\NormalTok{ (\{ }\DataTypeTok{count}\OperatorTok{:}\NormalTok{ state}\OperatorTok{.}\AttributeTok{count} \OperatorTok{+}\NormalTok{ action}\OperatorTok{.}\AttributeTok{count}\NormalTok{ \}))}
\NormalTok{)}
\end{Highlighting}
\end{Shaded}
\end{enumerate}

\hypertarget{154-metareducer}{%
\paragraph{15.4 MetaReducer}\label{154-metareducer}}

metaReducer 是 Action -\textgreater{} Reducer 之间的钩子,允许开发者对
Action 进行预处理 (在普通 Reducer 函数调用之前调用)。

\begin{Shaded}
\begin{Highlighting}[]
\KeywordTok{function} \FunctionTok{debug}\NormalTok{(reducer}\OperatorTok{:}\NormalTok{ ActionReducer}\OperatorTok{\textless{}}\NormalTok{any}\OperatorTok{\textgreater{}}\NormalTok{)}\OperatorTok{:}\NormalTok{ ActionReducer}\OperatorTok{\textless{}}\NormalTok{any}\OperatorTok{\textgreater{}}\NormalTok{ \{}
  \ControlFlowTok{return} \KeywordTok{function}\NormalTok{ (state}\OperatorTok{,}\NormalTok{ action) \{}
    \ControlFlowTok{return} \FunctionTok{reducer}\NormalTok{(state}\OperatorTok{,}\NormalTok{ action)}
\NormalTok{  \}}
\NormalTok{\}}

\ImportTok{export} \KeywordTok{const}\NormalTok{ metaReducers}\OperatorTok{:}\NormalTok{ MetaReducer}\OperatorTok{\textless{}}\NormalTok{AppState}\OperatorTok{\textgreater{}}\NormalTok{[] }\OperatorTok{=} \OperatorTok{!}\NormalTok{environment}\OperatorTok{.}\AttributeTok{production}
  \OperatorTok{?}\NormalTok{ [debug]}
  \OperatorTok{:}\NormalTok{ []}
\end{Highlighting}
\end{Shaded}

\hypertarget{155-effect}{%
\paragraph{15.5 Effect}\label{155-effect}}

需求:在页面中新增一个按钮,点击按钮后延迟一秒让数值增加。

\begin{enumerate}
\def\labelenumi{\arabic{enumi}.}
\item
  在组件模板中新增一个用于异步数值增加的按钮,按钮被点击后执行
  \texttt{increment\_async} 方法

\begin{Shaded}
\begin{Highlighting}[]
\KeywordTok{\textless{}button}\OtherTok{ (click)=}\StringTok{"increment\_async()"}\KeywordTok{\textgreater{}}\NormalTok{async}\KeywordTok{\textless{}/button\textgreater{}}
\end{Highlighting}
\end{Shaded}
\item
  在组件类中新增 \texttt{increment\_async}
  方法,并在方法中触发执行异步操作的 Action

\begin{Shaded}
\begin{Highlighting}[]
\FunctionTok{increment\_async}\NormalTok{() \{}
  \KeywordTok{this}\OperatorTok{.}\AttributeTok{store}\OperatorTok{.}\FunctionTok{dispatch}\NormalTok{(}\FunctionTok{increment\_async}\NormalTok{())}
\NormalTok{\}}
\end{Highlighting}
\end{Shaded}
\item
  在 Action 文件中新增执行异步操作的 Action

\begin{Shaded}
\begin{Highlighting}[]
\ImportTok{export} \KeywordTok{const}\NormalTok{ increment\_async }\OperatorTok{=} \FunctionTok{createAction}\NormalTok{(}\StringTok{"increment\_async"}\NormalTok{)}
\end{Highlighting}
\end{Shaded}
\item
  创建 Effect,接收 Action 并执行副作用,继续触发 Action

  \texttt{ng\ g\ effect\ store/effects/counter\ -\/-root\ -\/-module\ app.module.ts\ -\/-skipTests}

  Effect 功能由 @ngrx/effects
  模块提供,所以在根模块中需要导入相关的模块依赖

\begin{Shaded}
\begin{Highlighting}[]
\ImportTok{import}\NormalTok{ \{ Injectable \} }\ImportTok{from} \StringTok{"@angular/core"}
\ImportTok{import}\NormalTok{ \{ Actions}\OperatorTok{,}\NormalTok{ createEffect}\OperatorTok{,}\NormalTok{ ofType \} }\ImportTok{from} \StringTok{"@ngrx/effects"}
\ImportTok{import}\NormalTok{ \{ increment}\OperatorTok{,}\NormalTok{ increment\_async \} }\ImportTok{from} \StringTok{"../actions/counter.actions"}
\ImportTok{import}\NormalTok{ \{ mergeMap}\OperatorTok{,}\NormalTok{ map \} }\ImportTok{from} \StringTok{"rxjs/operators"}
\ImportTok{import}\NormalTok{ \{ timer \} }\ImportTok{from} \StringTok{"rxjs"}

\CommentTok{// createEffect}
\CommentTok{// 用于创建 Effect, Effect 用于执行副作用.}
\CommentTok{// 调用方法时传递回调函数, 回调函数中返回 Observable 对象, 对象中要发出副作用执行完成后要触发的 Action 对象}
\CommentTok{// 回调函数的返回值在 createEffect 方法内部被继续返回, 最终返回值被存储在了 Effect 类的属性中}
\CommentTok{// NgRx 在实例化 Effect 类后, 会订阅 Effect 类属性, 当副作用执行完成后它会获取到要触发的 Action 对象并触发这个 Action}

\CommentTok{// Actions}
\CommentTok{// 当组件触发 Action 时, Effect 需要通过 Actions 服务接收 Action, 所以在 Effect 类中通过 constructor 构造函数参数的方式将 Actions 服务类的实例对象注入到 Effect 类中}
\CommentTok{// Actions 服务类的实例对象为 Observable 对象, 当有 Action 被触发时, Action 对象本身会作为数据流被发出}

\CommentTok{// ofType}
\CommentTok{// 对目标 Action 对象进行过滤.}
\CommentTok{// 参数为目标 Action 的 Action Creator 函数}
\CommentTok{// 如果未过滤出目标 Action 对象, 本次不会继续发送数据流}
\CommentTok{// 如果过滤出目标 Action 对象, 会将 Action 对象作为数据流继续发出}

\NormalTok{@}\FunctionTok{Injectable}\NormalTok{()}
\ImportTok{export} \KeywordTok{class}\NormalTok{ CounterEffects \{}
  \FunctionTok{constructor}\NormalTok{(}\KeywordTok{private} \DataTypeTok{actions}\OperatorTok{:}\NormalTok{ Actions) \{}
    \CommentTok{// this.loadCount.subscribe(console.log)}
\NormalTok{  \}}
\NormalTok{  loadCount }\OperatorTok{=} \FunctionTok{createEffect}\NormalTok{(() }\KeywordTok{=\textgreater{}}\NormalTok{ \{}
    \ControlFlowTok{return} \KeywordTok{this}\OperatorTok{.}\AttributeTok{actions}\OperatorTok{.}\FunctionTok{pipe}\NormalTok{(}
      \FunctionTok{ofType}\NormalTok{(increment\_async)}\OperatorTok{,}
      \FunctionTok{mergeMap}\NormalTok{(() }\KeywordTok{=\textgreater{}} \FunctionTok{timer}\NormalTok{(}\DecValTok{1000}\NormalTok{)}\OperatorTok{.}\FunctionTok{pipe}\NormalTok{(}\FunctionTok{map}\NormalTok{(() }\KeywordTok{=\textgreater{}} \FunctionTok{increment}\NormalTok{(\{ }\DataTypeTok{count}\OperatorTok{:} \DecValTok{10}\NormalTok{ \}))))}
\NormalTok{    )}
\NormalTok{  \})}
\NormalTok{\}}
\end{Highlighting}
\end{Shaded}
\end{enumerate}

\hypertarget{156-entity}{%
\paragraph{15.6 Entity}\label{156-entity}}

\hypertarget{1561-ux6982ux8ff0}{%
\subparagraph{15.6.1 概述}\label{1561-ux6982ux8ff0}}

Entity 译为实体,实体就是集合中的一条数据。

NgRx
中提供了实体适配器对象,在实体适配器对象下面提供了各种操作集合中实体的方法,目的就是提高开发者操作实体的效率。

\hypertarget{1562-ux6838ux5fc3}{%
\subparagraph{15.6.2 核心}\label{1562-ux6838ux5fc3}}

\begin{enumerate}
\def\labelenumi{\arabic{enumi}.}
\item
  EntityState:实体类型接口

\begin{Shaded}
\begin{Highlighting}[]
\CommentTok{/*}
\CommentTok{	\{}
\CommentTok{		ids: [1, 2],}
\CommentTok{		entities: \{}
\CommentTok{			1: \{ id: 1, title: "Hello Angular" \},}
\CommentTok{			2: \{ id: 2, title: "Hello NgRx" \}}
\CommentTok{		\}}
\CommentTok{	\}}
\CommentTok{*/}
\ImportTok{export} \KeywordTok{interface}\NormalTok{ State }\KeywordTok{extends}\NormalTok{ EntityState}\OperatorTok{\textless{}}\NormalTok{Todo}\OperatorTok{\textgreater{}}\NormalTok{ \{\}}
\end{Highlighting}
\end{Shaded}
\item
  createEntityAdapter: 创建实体适配器对象
\item
  EntityAdapter:实体适配器对象类型接口

\begin{Shaded}
\begin{Highlighting}[]
\ImportTok{export} \KeywordTok{const}\NormalTok{ adapter}\OperatorTok{:}\NormalTok{ EntityAdapter}\OperatorTok{\textless{}}\NormalTok{Todo}\OperatorTok{\textgreater{}} \OperatorTok{=}\NormalTok{ createEntityAdapter}\OperatorTok{\textless{}}\NormalTok{Todo}\OperatorTok{\textgreater{}}\NormalTok{()}
\CommentTok{// 获取初始状态 可以传递对象参数 也可以不传}
\CommentTok{// \{ids: [], entities: \{\}\}}
\ImportTok{export} \KeywordTok{const}\NormalTok{ initialState}\OperatorTok{:}\NormalTok{ State }\OperatorTok{=}\NormalTok{ adapter}\OperatorTok{.}\FunctionTok{getInitialState}\NormalTok{()}
\end{Highlighting}
\end{Shaded}
\end{enumerate}

\hypertarget{1563-ux5b9eux4f8bux65b9ux6cd5}{%
\subparagraph{15.6.3 实例方法}\label{1563-ux5b9eux4f8bux65b9ux6cd5}}

\url{https://ngrx.io/guide/entity/adapter\#adapter-collection-methods}

\hypertarget{1564-ux9009ux62e9ux5668}{%
\subparagraph{15.6.4 选择器}\label{1564-ux9009ux62e9ux5668}}

\begin{Shaded}
\begin{Highlighting}[]
\CommentTok{// selectTotal 获取数据条数}
\CommentTok{// selectAll 获取所有数据 以数组形式呈现}
\CommentTok{// selectEntities 获取实体集合 以字典形式呈现}
\CommentTok{// selectIds 获取id集合, 以数组形式呈现}
\KeywordTok{const}\NormalTok{ \{ selectIds}\OperatorTok{,}\NormalTok{ selectEntities}\OperatorTok{,}\NormalTok{ selectAll}\OperatorTok{,}\NormalTok{ selectTotal \} }\OperatorTok{=}\NormalTok{ adapter}\OperatorTok{.}\FunctionTok{getSelectors}\NormalTok{()}\OperatorTok{;}
\end{Highlighting}
\end{Shaded}

\begin{Shaded}
\begin{Highlighting}[]
\ImportTok{export} \KeywordTok{const}\NormalTok{ selectTodo }\OperatorTok{=}\NormalTok{ createFeatureSelector}\OperatorTok{\textless{}}\NormalTok{AppState}\OperatorTok{,}\NormalTok{ State}\OperatorTok{\textgreater{}}\NormalTok{(todoFeatureKey)}
\ImportTok{export} \KeywordTok{const}\NormalTok{ selectTodos }\OperatorTok{=} \FunctionTok{createSelector}\NormalTok{(selectTodo}\OperatorTok{,}\NormalTok{ selectAll)}
\end{Highlighting}
\end{Shaded}

\hypertarget{157-router-store}{%
\paragraph{15.7 Router Store}\label{157-router-store}}

\hypertarget{1571-ux540cux6b65ux8defux7531ux72b6ux6001}{%
\subparagraph{15.7.1
同步路由状态}\label{1571-ux540cux6b65ux8defux7531ux72b6ux6001}}

\begin{enumerate}
\def\labelenumi{\arabic{enumi}.}
\item
  引入模块

\begin{Shaded}
\begin{Highlighting}[]
\ImportTok{import}\NormalTok{ \{ StoreRouterConnectingModule \} }\ImportTok{from} \StringTok{"@ngrx/router{-}store"}

\NormalTok{@}\FunctionTok{NgModule}\NormalTok{(\{}
  \DataTypeTok{imports}\OperatorTok{:}\NormalTok{ [}
\NormalTok{    StoreRouterConnectingModule}\OperatorTok{.}\FunctionTok{forRoot}\NormalTok{()}
\NormalTok{  ]}
\NormalTok{\})}
\ImportTok{export} \KeywordTok{class}\NormalTok{ AppModule \{\}}
\end{Highlighting}
\end{Shaded}
\item
  将路由状态集成到 Store

\begin{Shaded}
\begin{Highlighting}[]
\ImportTok{import} \OperatorTok{*} \ImportTok{as}\NormalTok{ fromRouter }\ImportTok{from} \StringTok{"@ngrx/router{-}store"}

\ImportTok{export} \KeywordTok{interface}\NormalTok{ AppState \{}
  \DataTypeTok{router}\OperatorTok{:}\NormalTok{ fromRouter}\OperatorTok{.}\AttributeTok{RouterReducerState}
\NormalTok{\}}
\ImportTok{export} \KeywordTok{const}\NormalTok{ reducers}\OperatorTok{:}\NormalTok{ ActionReducerMap}\OperatorTok{\textless{}}\NormalTok{AppState}\OperatorTok{\textgreater{}} \OperatorTok{=}\NormalTok{ \{}
  \DataTypeTok{router}\OperatorTok{:}\NormalTok{ fromRouter}\OperatorTok{.}\AttributeTok{routerReducer}
\NormalTok{\}}
\end{Highlighting}
\end{Shaded}
\end{enumerate}

\hypertarget{1572-ux521bux5efaux83b7ux53d6ux8defux7531ux72b6ux6001ux7684-selector}{%
\subparagraph{15.7.2 创建获取路由状态的
Selector}\label{1572-ux521bux5efaux83b7ux53d6ux8defux7531ux72b6ux6001ux7684-selector}}

\begin{Shaded}
\begin{Highlighting}[]
\CommentTok{// router.selectors.ts}
\ImportTok{import}\NormalTok{ \{ createFeatureSelector \} }\ImportTok{from} \StringTok{"@ngrx/store"}
\ImportTok{import}\NormalTok{ \{ AppState \} }\ImportTok{from} \StringTok{".."}
\ImportTok{import}\NormalTok{ \{ RouterReducerState}\OperatorTok{,}\NormalTok{ getSelectors \} }\ImportTok{from} \StringTok{"@ngrx/router{-}store"}

\KeywordTok{const}\NormalTok{ selectRouter }\OperatorTok{=}\NormalTok{ createFeatureSelector}\OperatorTok{\textless{}}\NormalTok{AppState}\OperatorTok{,}\NormalTok{ RouterReducerState}\OperatorTok{\textgreater{}}\NormalTok{(}
  \StringTok{"router"}
\NormalTok{)}

\ImportTok{export} \KeywordTok{const}\NormalTok{ \{}
  \CommentTok{// 获取和当前路由相关的信息 (路由参数、路由配置等)}
\NormalTok{  selectCurrentRoute}\OperatorTok{,}
  \CommentTok{// 获取地址栏中 \# 号后面的内容}
\NormalTok{  selectFragment}\OperatorTok{,}
  \CommentTok{// 获取路由查询参数}
\NormalTok{  selectQueryParams}\OperatorTok{,}
  \CommentTok{// 获取具体的某一个查询参数 selectQueryParam(\textquotesingle{}name\textquotesingle{})}
\NormalTok{  selectQueryParam}\OperatorTok{,}
  \CommentTok{// 获取动态路由参数}
\NormalTok{  selectRouteParams}\OperatorTok{,}
 	\CommentTok{// 获取某一个具体的动态路由参数 selectRouteParam(\textquotesingle{}name\textquotesingle{})}
\NormalTok{  selectRouteParam}\OperatorTok{,}
  \CommentTok{// 获取路由自定义数据}
\NormalTok{  selectRouteData}\OperatorTok{,}
  \CommentTok{// 获取路由的实际访问地址}
\NormalTok{  selectUrl}
\NormalTok{\} }\OperatorTok{=} \FunctionTok{getSelectors}\NormalTok{(selectRouter)}
\end{Highlighting}
\end{Shaded}

\begin{Shaded}
\begin{Highlighting}[]
\CommentTok{// home.component.ts}
\ImportTok{import}\NormalTok{ \{ select}\OperatorTok{,}\NormalTok{ Store \} }\ImportTok{from} \StringTok{"@ngrx/store"}
\ImportTok{import}\NormalTok{ \{ AppState \} }\ImportTok{from} \StringTok{"src/app/store"}
\ImportTok{import}\NormalTok{ \{ selectQueryParams \} }\ImportTok{from} \StringTok{"src/app/store/selectors/router.selectors"}

\ImportTok{export} \KeywordTok{class}\NormalTok{ AboutComponent \{}
  \FunctionTok{constructor}\NormalTok{(}\KeywordTok{private} \DataTypeTok{store}\OperatorTok{:}\NormalTok{ Store}\OperatorTok{\textless{}}\NormalTok{AppState}\OperatorTok{\textgreater{}}\NormalTok{) \{}
    \KeywordTok{this}\OperatorTok{.}\AttributeTok{store}\OperatorTok{.}\FunctionTok{pipe}\NormalTok{(}\FunctionTok{select}\NormalTok{(selectQueryParams))}\OperatorTok{.}\FunctionTok{subscribe}\NormalTok{(}\BuiltInTok{console}\OperatorTok{.}\FunctionTok{log}\NormalTok{)}
\NormalTok{  \}}
\NormalTok{\}}
\end{Highlighting}
\end{Shaded}

\hypertarget{16-ux52a8ux753b}{%
\subsubsection{16. 动画}\label{16-ux52a8ux753b}}

\begin{figure}
\centering
\includegraphics{C:/Users/ZSH/Desktop/ng/ppt/images/55.gif}
\caption{}
\end{figure}

\hypertarget{161-ux72b6ux6001}{%
\paragraph{16.1 状态}\label{161-ux72b6ux6001}}

\hypertarget{1661-ux4ec0ux4e48ux662fux72b6ux6001}{%
\subparagraph{16.6.1
什么是状态}\label{1661-ux4ec0ux4e48ux662fux72b6ux6001}}

状态表示的是要进行运动的元素在运动的不同时期所呈现的样式。

\begin{figure}
\centering
\includegraphics{C:/Users/ZSH/Desktop/ng/ppt/images/50.png}
\caption{}
\end{figure}

\hypertarget{1662-ux72b6ux6001ux7684ux79cdux7c7b}{%
\subparagraph{16.6.2
状态的种类}\label{1662-ux72b6ux6001ux7684ux79cdux7c7b}}

在 Angular
中,有三种类型的状态,分别为:\texttt{void}、\texttt{*}、\texttt{custom}

\begin{figure}
\centering
\includegraphics{C:/Users/ZSH/Desktop/ng/ppt/images/51.png}
\caption{}
\end{figure}

void:当元素在内存中创建好但尚未被添加到 DOM 中或将元素从 DOM
中删除时会发生此状态

*:元素被插入到 DOM
树之后的状态,或者是已经在DOM树中的元素的状态,也叫默认状态

custom:自定义状态,元素默认就在页面之中,从一个状态运动到另一个状态,比如面板的折叠和展开。

\hypertarget{1663-ux8fdbux51faux573aux52a8ux753b}{%
\subparagraph{16.6.3
进出场动画}\label{1663-ux8fdbux51faux573aux52a8ux753b}}

进场动画是指元素被创建后以动画的形式出现在用户面前,进场动画的状态用
\texttt{void\ =\textgreater{}\ *} 表示,别名为 \texttt{:enter}

\begin{figure}
\centering
\includegraphics{C:/Users/ZSH/Desktop/ng/ppt/images/52.png}
\caption{}
\end{figure}

出场动画是指元素在被删除前执行的一段告别动画,出场动画的状态用
\texttt{*\ =\textgreater{}\ void},别名为 \texttt{:leave}

\begin{figure}
\centering
\includegraphics{C:/Users/ZSH/Desktop/ng/ppt/images/53.png}
\caption{}
\end{figure}

\hypertarget{162-ux5febux901fux4e0aux624b}{%
\paragraph{16.2 快速上手}\label{162-ux5febux901fux4e0aux624b}}

\begin{enumerate}
\def\labelenumi{\arabic{enumi}.}
\item
  在使用动画功能之前,需要引入动画模块,即
  \texttt{BrowserAnimationsModule}

\begin{Shaded}
\begin{Highlighting}[]
\ImportTok{import}\NormalTok{ \{ BrowserAnimationsModule \} }\ImportTok{from} \StringTok{"@angular/platform{-}browser/animations"}

\NormalTok{@}\FunctionTok{NgModule}\NormalTok{(\{}
  \DataTypeTok{imports}\OperatorTok{:}\NormalTok{ [BrowserAnimationsModule]}\OperatorTok{,}
\NormalTok{\})}
\ImportTok{export} \KeywordTok{class}\NormalTok{ AppModule \{\}}
\end{Highlighting}
\end{Shaded}
\item
  默认代码解析,todo 之删除任务和添加任务

\begin{Shaded}
\begin{Highlighting}[]
\CommentTok{\textless{}!{-}{-} 在 index.html 文件中引入 bootstrap.min.css {-}{-}\textgreater{}}
\KeywordTok{\textless{}link}\OtherTok{ rel=}\StringTok{"stylesheet"}\OtherTok{ href=}\StringTok{"https://cdn.jsdelivr.net/npm/bootstrap@3.3.7/dist/css/bootstrap.min.css"} \KeywordTok{/\textgreater{}}
\end{Highlighting}
\end{Shaded}

\begin{Shaded}
\begin{Highlighting}[]
\KeywordTok{\textless{}div}\OtherTok{ class=}\StringTok{"container"}\KeywordTok{\textgreater{}}
  \KeywordTok{\textless{}h2\textgreater{}}\NormalTok{Todos}\KeywordTok{\textless{}/h2\textgreater{}}
  \KeywordTok{\textless{}div}\OtherTok{ class=}\StringTok{"form{-}group"}\KeywordTok{\textgreater{}}
    \KeywordTok{\textless{}input}\OtherTok{ (keyup.enter)=}\StringTok{"addItem(input)"}\OtherTok{ \#input type=}\StringTok{"text"}\OtherTok{ class=}\StringTok{"form{-}control"}\OtherTok{ placeholder=}\StringTok{"add todos"} \KeywordTok{/\textgreater{}}
  \KeywordTok{\textless{}/div\textgreater{}}
  \KeywordTok{\textless{}ul}\OtherTok{ class=}\StringTok{"list{-}group"}\KeywordTok{\textgreater{}}
    \KeywordTok{\textless{}li}\OtherTok{ (click)=}\StringTok{"removeItem(i)"}\OtherTok{ *ngFor=}\StringTok{"let item of todos; let i = index"}\OtherTok{ class=}\StringTok{"list{-}group{-}item"}\KeywordTok{\textgreater{}}
\NormalTok{      \{\{ item \}\}}
    \KeywordTok{\textless{}/li\textgreater{}}
  \KeywordTok{\textless{}/ul\textgreater{}}
\KeywordTok{\textless{}/div\textgreater{}}
\end{Highlighting}
\end{Shaded}

\begin{Shaded}
\begin{Highlighting}[]
\ImportTok{import}\NormalTok{ \{ Component \} }\ImportTok{from} \StringTok{"@angular/core"}

\NormalTok{@}\FunctionTok{Component}\NormalTok{(\{}
  \DataTypeTok{selector}\OperatorTok{:} \StringTok{"app{-}root"}\OperatorTok{,}
  \DataTypeTok{templateUrl}\OperatorTok{:} \StringTok{"./app.component.html"}\OperatorTok{,}
  \DataTypeTok{styles}\OperatorTok{:}\NormalTok{ []}
\NormalTok{\})}
\ImportTok{export} \KeywordTok{class}\NormalTok{ AppComponent \{}
  \CommentTok{// todo 列表}
  \DataTypeTok{todos}\OperatorTok{:}\NormalTok{ string[] }\OperatorTok{=}\NormalTok{ [}\StringTok{"Learn Angular"}\OperatorTok{,} \StringTok{"Learn RxJS"}\OperatorTok{,} \StringTok{"Learn NgRx"}\NormalTok{]}
	\CommentTok{// 添加 todo}
  \FunctionTok{addItem}\NormalTok{(}\DataTypeTok{input}\OperatorTok{:} \BuiltInTok{HTMLInputElement}\NormalTok{) \{}
    \KeywordTok{this}\OperatorTok{.}\AttributeTok{todos}\OperatorTok{.}\FunctionTok{push}\NormalTok{(input}\OperatorTok{.}\AttributeTok{value}\NormalTok{)}
\NormalTok{    input}\OperatorTok{.}\AttributeTok{value} \OperatorTok{=} \StringTok{""}
\NormalTok{  \}}
	\CommentTok{// 删除 todo}
  \FunctionTok{removeItem}\NormalTok{(}\DataTypeTok{index}\OperatorTok{:}\NormalTok{ number) \{}
    \KeywordTok{this}\OperatorTok{.}\AttributeTok{todos}\OperatorTok{.}\FunctionTok{splice}\NormalTok{(index}\OperatorTok{,} \DecValTok{1}\NormalTok{)}
\NormalTok{  \}}
\NormalTok{\}}
\end{Highlighting}
\end{Shaded}
\item
  创建动画

  \begin{enumerate}
  \def\labelenumii{\arabic{enumii}.}
  \item
    trigger 方法用于创建动画,指定动画名称
  \item
    transition
    方法用于指定动画的运动状态,出场动画或者入场动画,或者自定义状态动画。
  \item
    style 方法用于设置元素在不同的状态下所对应的样式
  \item
    animate 方法用于设置运动参数,比如动画运动时间,延迟事件,运动形式
  \end{enumerate}

\begin{Shaded}
\begin{Highlighting}[]
\NormalTok{@}\FunctionTok{Component}\NormalTok{(\{}
  \DataTypeTok{animations}\OperatorTok{:}\NormalTok{ [}
    \CommentTok{// 创建动画, 动画名称为 slide}
    \FunctionTok{trigger}\NormalTok{(}\StringTok{"slide"}\OperatorTok{,}\NormalTok{ [}
      \CommentTok{// 指定入场动画 注意: 字符串两边不能有空格, 箭头两边可以有也可以没有空格}
      \CommentTok{// void =\textgreater{} * 可以替换为 :enter}
      \FunctionTok{transition}\NormalTok{(}\StringTok{"void =\textgreater{} *"}\OperatorTok{,}\NormalTok{ [}
        \CommentTok{// 指定元素未入场前的样式}
        \FunctionTok{style}\NormalTok{(\{ }\DataTypeTok{opacity}\OperatorTok{:} \DecValTok{0}\OperatorTok{,} \DataTypeTok{transform}\OperatorTok{:} \StringTok{"translateY(40px)"}\NormalTok{ \})}\OperatorTok{,}
        \CommentTok{// 指定元素入场后的样式及运动参数}
        \FunctionTok{animate}\NormalTok{(}\DecValTok{250}\OperatorTok{,} \FunctionTok{style}\NormalTok{(\{ }\DataTypeTok{opacity}\OperatorTok{:} \DecValTok{1}\OperatorTok{,} \DataTypeTok{transform}\OperatorTok{:} \StringTok{"translateY(0)"}\NormalTok{ \}))}
\NormalTok{      ])}\OperatorTok{,}
      \CommentTok{// 指定出场动画}
      \CommentTok{// * =\textgreater{} void 可以替换为 :leave}
      \FunctionTok{transition}\NormalTok{(}\StringTok{"* =\textgreater{} void"}\OperatorTok{,}\NormalTok{ [}
        \CommentTok{// 指定元素出场后的样式和运动参数}
        \FunctionTok{animate}\NormalTok{(}\DecValTok{600}\OperatorTok{,} \FunctionTok{style}\NormalTok{(\{ }\DataTypeTok{opacity}\OperatorTok{:} \DecValTok{0}\OperatorTok{,} \DataTypeTok{transform}\OperatorTok{:} \StringTok{"translateX(100\%)"}\NormalTok{ \}))}
\NormalTok{      ])}
\NormalTok{    ])}
\NormalTok{  ]}
\NormalTok{\})}
\end{Highlighting}
\end{Shaded}

\begin{Shaded}
\begin{Highlighting}[]
\KeywordTok{\textless{}li} \ErrorTok{@slide}\KeywordTok{\textgreater{}\textless{}/li\textgreater{}}
\end{Highlighting}
\end{Shaded}

  注意:入场动画中可以不指定元素的默认状态,Angular 会将 void
  状态清空作为默认状态

\begin{Shaded}
\begin{Highlighting}[]
\FunctionTok{trigger}\NormalTok{(}\StringTok{"slide"}\OperatorTok{,}\NormalTok{ [}
  \FunctionTok{transition}\NormalTok{(}\StringTok{":enter"}\OperatorTok{,}\NormalTok{ [}
    \FunctionTok{style}\NormalTok{(\{ }\DataTypeTok{opacity}\OperatorTok{:} \DecValTok{0}\OperatorTok{,} \DataTypeTok{transform}\OperatorTok{:} \StringTok{"translateY(40px)"}\NormalTok{ \})}\OperatorTok{,}
    \FunctionTok{animate}\NormalTok{(}\DecValTok{250}\NormalTok{)}
\NormalTok{  ])}\OperatorTok{,}
  \FunctionTok{transition}\NormalTok{(}\StringTok{":leave"}\OperatorTok{,}\NormalTok{ [}
 		\FunctionTok{animate}\NormalTok{(}\DecValTok{600}\OperatorTok{,} \FunctionTok{style}\NormalTok{(\{ }\DataTypeTok{opacity}\OperatorTok{:} \DecValTok{0}\OperatorTok{,} \DataTypeTok{transform}\OperatorTok{:} \StringTok{"translateX(100\%)"}\NormalTok{ \}))}
\NormalTok{  ])}
\NormalTok{])}
\end{Highlighting}
\end{Shaded}

  注意:要设置动画的运动参数,需要将 animate
  方法的一个参数更改为字符串类型

\begin{Shaded}
\begin{Highlighting}[]
\CommentTok{// 动画执行总时间 延迟时间 (可选) 运动形式 (可选)}
\FunctionTok{animate}\NormalTok{(}\StringTok{"600ms 1s ease{-}out"}\OperatorTok{,} \FunctionTok{style}\NormalTok{(\{ }\DataTypeTok{opacity}\OperatorTok{:} \DecValTok{0}\OperatorTok{,} \DataTypeTok{transform}\OperatorTok{:} \StringTok{"translateX(100\%)"}\NormalTok{ \}))}
\end{Highlighting}
\end{Shaded}
\end{enumerate}

\hypertarget{163--ux5173ux952eux5e27ux52a8ux753b}{%
\paragraph{16.3 关键帧动画}\label{163--ux5173ux952eux5e27ux52a8ux753b}}

关键帧动画使用 \texttt{keyframes} 方法定义

\begin{Shaded}
\begin{Highlighting}[]
\FunctionTok{transition}\NormalTok{(}\StringTok{":leave"}\OperatorTok{,}\NormalTok{ [}
  \FunctionTok{animate}\NormalTok{(}
    \DecValTok{600}\OperatorTok{,}
    \FunctionTok{keyframes}\NormalTok{([}
      \FunctionTok{style}\NormalTok{(\{ }\DataTypeTok{offset}\OperatorTok{:} \FloatTok{0.3}\OperatorTok{,} \DataTypeTok{transform}\OperatorTok{:} \StringTok{"translateX({-}80px)"}\NormalTok{ \})}\OperatorTok{,}
      \FunctionTok{style}\NormalTok{(\{ }\DataTypeTok{offset}\OperatorTok{:} \DecValTok{1}\OperatorTok{,} \DataTypeTok{transform}\OperatorTok{:} \StringTok{"translateX(100\%)"}\NormalTok{ \})}
\NormalTok{    ])}
\NormalTok{  )}
\NormalTok{])}
\end{Highlighting}
\end{Shaded}

\hypertarget{164-ux52a8ux753bux56deux8c03}{%
\paragraph{16.4 动画回调}\label{164-ux52a8ux753bux56deux8c03}}

Angular
提供了和动画相关的两个回调函数,分别为动画开始执行时和动画执行完成后

\begin{Shaded}
\begin{Highlighting}[]
\KeywordTok{\textless{}li} \ErrorTok{@slide}\OtherTok{ (}\ErrorTok{@slide.start)}\OtherTok{=}\StringTok{"start($event)"}\OtherTok{ (}\ErrorTok{@slide.done)}\OtherTok{=}\StringTok{"done($event)"}\KeywordTok{\textgreater{}\textless{}/li\textgreater{}}
\end{Highlighting}
\end{Shaded}

\begin{Shaded}
\begin{Highlighting}[]
\ImportTok{import}\NormalTok{ \{ AnimationEvent \} }\ImportTok{from} \StringTok{"@angular/animations"}

\FunctionTok{start}\NormalTok{(}\BuiltInTok{event}\OperatorTok{:}\NormalTok{ AnimationEvent) \{}
  \BuiltInTok{console}\OperatorTok{.}\FunctionTok{log}\NormalTok{(}\BuiltInTok{event}\NormalTok{)}
\NormalTok{\}}
\FunctionTok{done}\NormalTok{(}\BuiltInTok{event}\OperatorTok{:}\NormalTok{ AnimationEvent) \{}
  \BuiltInTok{console}\OperatorTok{.}\FunctionTok{log}\NormalTok{(}\BuiltInTok{event}\NormalTok{)}
\NormalTok{\}}
\end{Highlighting}
\end{Shaded}

\hypertarget{165-ux521bux5efaux53efux91cdux7528ux52a8ux753b}{%
\paragraph{16.5
创建可重用动画}\label{165-ux521bux5efaux53efux91cdux7528ux52a8ux753b}}

\begin{enumerate}
\def\labelenumi{\arabic{enumi}.}
\item
  将动画的定义放置在单独的文件中,方便多组件调用。

\begin{Shaded}
\begin{Highlighting}[]
\ImportTok{import}\NormalTok{ \{ animate}\OperatorTok{,}\NormalTok{ keyframes}\OperatorTok{,}\NormalTok{ style}\OperatorTok{,}\NormalTok{ transition}\OperatorTok{,}\NormalTok{ trigger \} }\ImportTok{from} \StringTok{"@angular/animations"}

\ImportTok{export} \KeywordTok{const}\NormalTok{ slide }\OperatorTok{=} \FunctionTok{trigger}\NormalTok{(}\StringTok{"slide"}\OperatorTok{,}\NormalTok{ [}
  \FunctionTok{transition}\NormalTok{(}\StringTok{":enter"}\OperatorTok{,}\NormalTok{ [}
    \FunctionTok{style}\NormalTok{(\{ }\DataTypeTok{opacity}\OperatorTok{:} \DecValTok{0}\OperatorTok{,} \DataTypeTok{transform}\OperatorTok{:} \StringTok{"translateY(40px)"}\NormalTok{ \})}\OperatorTok{,}
    \FunctionTok{animate}\NormalTok{(}\DecValTok{250}\NormalTok{)}
\NormalTok{  ])}\OperatorTok{,}
  \FunctionTok{transition}\NormalTok{(}\StringTok{":leave"}\OperatorTok{,}\NormalTok{ [}
    \FunctionTok{animate}\NormalTok{(}
      \DecValTok{600}\OperatorTok{,}
      \FunctionTok{keyframes}\NormalTok{([}
        \FunctionTok{style}\NormalTok{(\{ }\DataTypeTok{offset}\OperatorTok{:} \FloatTok{0.3}\OperatorTok{,} \DataTypeTok{transform}\OperatorTok{:} \StringTok{"translateX({-}80px)"}\NormalTok{ \})}\OperatorTok{,}
        \FunctionTok{style}\NormalTok{(\{ }\DataTypeTok{offset}\OperatorTok{:} \DecValTok{1}\OperatorTok{,} \DataTypeTok{transform}\OperatorTok{:} \StringTok{"translateX(100\%)"}\NormalTok{ \})}
\NormalTok{      ])}
\NormalTok{    )}
\NormalTok{  ])}
\NormalTok{])}
\end{Highlighting}
\end{Shaded}

\begin{Shaded}
\begin{Highlighting}[]
\ImportTok{import}\NormalTok{ \{ slide \} }\ImportTok{from} \StringTok{"./animations"}

\NormalTok{@}\FunctionTok{Component}\NormalTok{(\{}
  \DataTypeTok{animations}\OperatorTok{:}\NormalTok{ [slide]}
\NormalTok{\})}
\end{Highlighting}
\end{Shaded}
\item
  抽取具体的动画定义,方便多动画调用。

\begin{Shaded}
\begin{Highlighting}[]
\ImportTok{import}\NormalTok{ \{animate}\OperatorTok{,}\NormalTok{ animation}\OperatorTok{,}\NormalTok{ keyframes}\OperatorTok{,}\NormalTok{ style}\OperatorTok{,}\NormalTok{ transition}\OperatorTok{,}\NormalTok{ trigger}\OperatorTok{,}\NormalTok{ useAnimation\} }\ImportTok{from} \StringTok{"@angular/animations"}

\ImportTok{export} \KeywordTok{const}\NormalTok{ slideInUp }\OperatorTok{=} \FunctionTok{animation}\NormalTok{([}
  \FunctionTok{style}\NormalTok{(\{ }\DataTypeTok{opacity}\OperatorTok{:} \DecValTok{0}\OperatorTok{,} \DataTypeTok{transform}\OperatorTok{:} \StringTok{"translateY(40px)"}\NormalTok{ \})}\OperatorTok{,}
  \FunctionTok{animate}\NormalTok{(}\DecValTok{250}\NormalTok{)}
\NormalTok{])}

\ImportTok{export} \KeywordTok{const}\NormalTok{ slideOutLeft }\OperatorTok{=} \FunctionTok{animation}\NormalTok{([}
  \FunctionTok{animate}\NormalTok{(}
    \DecValTok{600}\OperatorTok{,}
    \FunctionTok{keyframes}\NormalTok{([}
      \FunctionTok{style}\NormalTok{(\{ }\DataTypeTok{offset}\OperatorTok{:} \FloatTok{0.3}\OperatorTok{,} \DataTypeTok{transform}\OperatorTok{:} \StringTok{"translateX({-}80px)"}\NormalTok{ \})}\OperatorTok{,}
      \FunctionTok{style}\NormalTok{(\{ }\DataTypeTok{offset}\OperatorTok{:} \DecValTok{1}\OperatorTok{,} \DataTypeTok{transform}\OperatorTok{:} \StringTok{"translateX(100\%)"}\NormalTok{ \})}
\NormalTok{    ])}
\NormalTok{  )}
\NormalTok{])}

\ImportTok{export} \KeywordTok{const}\NormalTok{ slide }\OperatorTok{=} \FunctionTok{trigger}\NormalTok{(}\StringTok{"slide"}\OperatorTok{,}\NormalTok{ [}
  \FunctionTok{transition}\NormalTok{(}\StringTok{":enter"}\OperatorTok{,} \FunctionTok{useAnimation}\NormalTok{(slideInUp))}\OperatorTok{,}
  \FunctionTok{transition}\NormalTok{(}\StringTok{":leave"}\OperatorTok{,} \FunctionTok{useAnimation}\NormalTok{(slideOutLeft))}
\NormalTok{])}
\end{Highlighting}
\end{Shaded}
\item
  调用动画时传递运动总时间,延迟时间,运动形式

\begin{Shaded}
\begin{Highlighting}[]
\ImportTok{export} \KeywordTok{const}\NormalTok{ slideInUp }\OperatorTok{=} \FunctionTok{animation}\NormalTok{(}
\NormalTok{  [}
    \FunctionTok{style}\NormalTok{(\{ }\DataTypeTok{opacity}\OperatorTok{:} \DecValTok{0}\OperatorTok{,} \DataTypeTok{transform}\OperatorTok{:} \StringTok{"translateY(40px)"}\NormalTok{ \})}\OperatorTok{,}
    \FunctionTok{animate}\NormalTok{(}\StringTok{"\{\{ duration \}\} \{\{ delay \}\} \{\{ easing \}\}"}\NormalTok{)}
\NormalTok{  ]}\OperatorTok{,}
\NormalTok{  \{}
    \DataTypeTok{params}\OperatorTok{:}\NormalTok{ \{}
      \DataTypeTok{duration}\OperatorTok{:} \StringTok{"400ms"}\OperatorTok{,}
      \DataTypeTok{delay}\OperatorTok{:} \StringTok{"0s"}\OperatorTok{,}
      \DataTypeTok{easing}\OperatorTok{:} \StringTok{"ease{-}out"}
\NormalTok{    \}}
\NormalTok{  \}}
\NormalTok{)}
\end{Highlighting}
\end{Shaded}

\begin{Shaded}
\begin{Highlighting}[]
\FunctionTok{transition}\NormalTok{(}\StringTok{":enter"}\OperatorTok{,} \FunctionTok{useAnimation}\NormalTok{(slideInUp}\OperatorTok{,}\NormalTok{ \{}\DataTypeTok{params}\OperatorTok{:}\NormalTok{ \{}\DataTypeTok{delay}\OperatorTok{:} \StringTok{"1s"}\NormalTok{\}\}))}
\end{Highlighting}
\end{Shaded}
\end{enumerate}

\hypertarget{166-ux67e5ux8be2ux5143ux7d20ux6267ux884cux52a8ux753b}{%
\paragraph{16.6
查询元素执行动画}\label{166-ux67e5ux8be2ux5143ux7d20ux6267ux884cux52a8ux753b}}

Angular 中提供了 \texttt{query} 方法查找元素并为元素创建动画

\begin{Shaded}
\begin{Highlighting}[]
\ImportTok{import}\NormalTok{ \{ slide \} }\ImportTok{from} \StringTok{"./animations"}

\NormalTok{animations}\OperatorTok{:}\NormalTok{ [}
\NormalTok{  slide}\OperatorTok{,}
  \FunctionTok{trigger}\NormalTok{(}\StringTok{"todoAnimations"}\OperatorTok{,}\NormalTok{ [}
    \FunctionTok{transition}\NormalTok{(}\StringTok{":enter"}\OperatorTok{,}\NormalTok{ [}
      \FunctionTok{query}\NormalTok{(}\StringTok{"h2"}\OperatorTok{,}\NormalTok{ [}
        \FunctionTok{style}\NormalTok{(\{ }\DataTypeTok{transform}\OperatorTok{:} \StringTok{"translateY({-}30px)"}\NormalTok{ \})}\OperatorTok{,}
        \FunctionTok{animate}\NormalTok{(}\DecValTok{300}\NormalTok{)}
\NormalTok{      ])}\OperatorTok{,}
      \CommentTok{// 查询子级动画 使其执行}
      \FunctionTok{query}\NormalTok{(}\StringTok{"@slide"}\OperatorTok{,} \FunctionTok{animateChild}\NormalTok{())}
\NormalTok{    ])}
\NormalTok{  ])}
\NormalTok{]}
\end{Highlighting}
\end{Shaded}

\begin{Shaded}
\begin{Highlighting}[]
\KeywordTok{\textless{}div}\OtherTok{ class=}\StringTok{"container"} \ErrorTok{@todoAnimations}\KeywordTok{\textgreater{}}
  \KeywordTok{\textless{}h2\textgreater{}}\NormalTok{Todos}\KeywordTok{\textless{}/h2\textgreater{}}
  \KeywordTok{\textless{}ul}\OtherTok{ class=}\StringTok{"list{-}group"}\KeywordTok{\textgreater{}}
    \KeywordTok{\textless{}li}
      \ErrorTok{@slide}
\OtherTok{      (click)=}\StringTok{"removeItem(i)"}
\OtherTok{      *ngFor=}\StringTok{"let item of todos; let i = index"}
\OtherTok{      class=}\StringTok{"list{-}group{-}item"}
    \KeywordTok{\textgreater{}}
\NormalTok{      \{\{ item \}\}}
    \KeywordTok{\textless{}/li\textgreater{}}
  \KeywordTok{\textless{}/ul\textgreater{}}
\KeywordTok{\textless{}/div\textgreater{}}
\end{Highlighting}
\end{Shaded}

默认情况下,父级动画和子级动画按照顺序执行,先执行父级动画,再执行子级动画,可以使用
\texttt{group} 方法让其并行

\begin{Shaded}
\begin{Highlighting}[]
\FunctionTok{trigger}\NormalTok{(}\StringTok{"todoAnimations"}\OperatorTok{,}\NormalTok{ [}
  \FunctionTok{transition}\NormalTok{(}\StringTok{":enter"}\OperatorTok{,}\NormalTok{ [}
    \FunctionTok{group}\NormalTok{([}
      \FunctionTok{query}\NormalTok{(}\StringTok{"h2"}\OperatorTok{,}\NormalTok{ [}
        \FunctionTok{style}\NormalTok{(\{ }\DataTypeTok{transform}\OperatorTok{:} \StringTok{"translateY({-}30px)"}\NormalTok{ \})}\OperatorTok{,}
        \FunctionTok{animate}\NormalTok{(}\DecValTok{300}\NormalTok{)}
\NormalTok{      ])}\OperatorTok{,}
      \FunctionTok{query}\NormalTok{(}\StringTok{"@slide"}\OperatorTok{,} \FunctionTok{animateChild}\NormalTok{())}
\NormalTok{    ])}
\NormalTok{  ])}
\NormalTok{])}
\end{Highlighting}
\end{Shaded}

\hypertarget{167-ux4ea4ux9519ux52a8ux753b}{%
\paragraph{16.7 交错动画}\label{167-ux4ea4ux9519ux52a8ux753b}}

Angular 提供了 stagger
方法,在多个元素同时执行同一个动画时,让每个元素动画的执行依次延迟。

\begin{Shaded}
\begin{Highlighting}[]
\FunctionTok{transition}\NormalTok{(}\StringTok{":enter"}\OperatorTok{,}\NormalTok{ [}
  \FunctionTok{group}\NormalTok{([}
    \FunctionTok{query}\NormalTok{(}\StringTok{"h2"}\OperatorTok{,}\NormalTok{ [}
      \FunctionTok{style}\NormalTok{(\{ }\DataTypeTok{transform}\OperatorTok{:} \StringTok{"translateY({-}30px)"}\NormalTok{ \})}\OperatorTok{,}
      \FunctionTok{animate}\NormalTok{(}\DecValTok{300}\NormalTok{)}
\NormalTok{    ])}\OperatorTok{,}
    \FunctionTok{query}\NormalTok{(}\StringTok{"@slide"}\OperatorTok{,} \FunctionTok{stagger}\NormalTok{(}\DecValTok{200}\OperatorTok{,} \FunctionTok{animateChild}\NormalTok{()))}
\NormalTok{  ])}
\NormalTok{])}
\end{Highlighting}
\end{Shaded}

注意:stagger 方法只能在 query 方法内部使用

\hypertarget{168-ux81eaux5b9aux4e49ux72b6ux6001ux52a8ux753b}{%
\paragraph{16.8
自定义状态动画}\label{168-ux81eaux5b9aux4e49ux72b6ux6001ux52a8ux753b}}

Angular 提供了 \texttt{state} 方法用于定义状态。

\begin{figure}
\centering
\includegraphics{C:/Users/ZSH/Desktop/ng/ppt/images/56.gif}
\caption{}
\end{figure}

\begin{enumerate}
\def\labelenumi{\arabic{enumi}.}
\item
  默认代码解析

\begin{Shaded}
\begin{Highlighting}[]
\KeywordTok{\textless{}div}\OtherTok{ class=}\StringTok{"container"}\KeywordTok{\textgreater{}}
  \KeywordTok{\textless{}div}\OtherTok{ class=}\StringTok{"panel panel{-}default"}\KeywordTok{\textgreater{}}
    \KeywordTok{\textless{}div}\OtherTok{ class=}\StringTok{"panel{-}heading"}\OtherTok{ (click)=}\StringTok{"toggle()"}\KeywordTok{\textgreater{}}
\NormalTok{      一套框架, 多种平台, 移动端 }\ErrorTok{\&}\NormalTok{ 桌面端}
    \KeywordTok{\textless{}/div\textgreater{}}
    \KeywordTok{\textless{}div}\OtherTok{ class=}\StringTok{"panel{-}body"}\KeywordTok{\textgreater{}}
      \KeywordTok{\textless{}p\textgreater{}}
\NormalTok{        使用简单的声明式模板,快速实现各种特性。使用自定义组件和大量现有组件,扩展模板语言。在几乎所有的}
\NormalTok{        IDE 中获得针对 Angular}
\NormalTok{        的即时帮助和反馈。所有这一切,都是为了帮助你编写漂亮的应用,而不是绞尽脑汁的让代码“能用”。}
      \KeywordTok{\textless{}/p\textgreater{}}
      \KeywordTok{\textless{}p\textgreater{}}
\NormalTok{        从原型到全球部署,Angular 都能带给你支撑 Google}
\NormalTok{        大型应用的那些高延展性基础设施与技术。}
      \KeywordTok{\textless{}/p\textgreater{}}
      \KeywordTok{\textless{}p\textgreater{}}
\NormalTok{        通过 Web Worker 和服务端渲染,达到在如今(以及未来)的 Web}
\NormalTok{        平台上所能达到的最高速度。 Angular 让你有效掌控可伸缩性。基于}
\NormalTok{        RxJS、Immutable.js 和其它推送模型,能适应海量数据需求。}
      \KeywordTok{\textless{}/p\textgreater{}}
      \KeywordTok{\textless{}p\textgreater{}}
\NormalTok{        学会用 Angular}
\NormalTok{        构建应用,然后把这些代码和能力复用在多种多种不同平台的应用上 ——}
\NormalTok{        Web、移动 Web、移动应用、原生应用和桌面原生应用。}
      \KeywordTok{\textless{}/p\textgreater{}}
    \KeywordTok{\textless{}/div\textgreater{}}
  \KeywordTok{\textless{}/div\textgreater{}}
\KeywordTok{\textless{}/div\textgreater{}}
\KeywordTok{\textless{}style\textgreater{}}
  \FunctionTok{.container}\NormalTok{ \{}
    \KeywordTok{margin{-}top}\NormalTok{: }\DecValTok{100}\DataTypeTok{px}\OperatorTok{;}
\NormalTok{  \}}
  \FunctionTok{.panel{-}heading}\NormalTok{ \{}
    \KeywordTok{cursor}\NormalTok{: }\DecValTok{pointer}\OperatorTok{;}
\NormalTok{  \}}
\KeywordTok{\textless{}/style\textgreater{}}
\end{Highlighting}
\end{Shaded}

\begin{Shaded}
\begin{Highlighting}[]
\ImportTok{import}\NormalTok{ \{ Component \} }\ImportTok{from} \StringTok{"@angular/core"}

\NormalTok{@}\FunctionTok{Component}\NormalTok{(\{}
  \DataTypeTok{selector}\OperatorTok{:} \StringTok{"app{-}root"}\OperatorTok{,}
  \DataTypeTok{templateUrl}\OperatorTok{:} \StringTok{"./app.component.html"}\OperatorTok{,}
  \DataTypeTok{styles}\OperatorTok{:}\NormalTok{ []}
\NormalTok{\})}
\ImportTok{export} \KeywordTok{class}\NormalTok{ AppComponent \{}
  \DataTypeTok{isExpended}\OperatorTok{:}\NormalTok{ boolean }\OperatorTok{=} \KeywordTok{false}
  \FunctionTok{toggle}\NormalTok{() \{}
    \KeywordTok{this}\OperatorTok{.}\AttributeTok{isExpended} \OperatorTok{=} \OperatorTok{!}\KeywordTok{this}\OperatorTok{.}\AttributeTok{isExpended}
\NormalTok{  \}}
\NormalTok{\}}
\end{Highlighting}
\end{Shaded}
\item
  创建动画

\begin{Shaded}
\begin{Highlighting}[]
\FunctionTok{trigger}\NormalTok{(}\StringTok{"expandCollapse"}\OperatorTok{,}\NormalTok{ [}
  \CommentTok{// 使用 state 方法定义折叠状态元素对应的样式}
  \FunctionTok{state}\NormalTok{(}
    \StringTok{"collapsed"}\OperatorTok{,}
    \FunctionTok{style}\NormalTok{(\{}
      \DataTypeTok{height}\OperatorTok{:} \DecValTok{0}\OperatorTok{,}
      \DataTypeTok{overflow}\OperatorTok{:} \StringTok{"hidden"}\OperatorTok{,}
      \DataTypeTok{paddingTop}\OperatorTok{:} \DecValTok{0}\OperatorTok{,}
      \DataTypeTok{paddingBottom}\OperatorTok{:} \DecValTok{0}
\NormalTok{    \})}
\NormalTok{  )}\OperatorTok{,}
  \CommentTok{// 使用 state 方法定义展开状态元素对应的样式}
  \FunctionTok{state}\NormalTok{(}\StringTok{"expanded"}\OperatorTok{,} \FunctionTok{style}\NormalTok{(\{ }\DataTypeTok{height}\OperatorTok{:} \StringTok{"*"}\OperatorTok{,} \DataTypeTok{overflow}\OperatorTok{:} \StringTok{"auto"}\NormalTok{ \}))}\OperatorTok{,}
  \CommentTok{// 定义展开动画}
  \FunctionTok{transition}\NormalTok{(}\StringTok{"collapsed =\textgreater{} expanded"}\OperatorTok{,} \FunctionTok{animate}\NormalTok{(}\StringTok{"400ms ease{-}out"}\NormalTok{))}\OperatorTok{,}
  \CommentTok{// 定义折叠动画}
  \FunctionTok{transition}\NormalTok{(}\StringTok{"expanded =\textgreater{} collapsed"}\OperatorTok{,} \FunctionTok{animate}\NormalTok{(}\StringTok{"400ms ease{-}in"}\NormalTok{))}
\NormalTok{])}
\end{Highlighting}
\end{Shaded}

\begin{Shaded}
\begin{Highlighting}[]
\KeywordTok{\textless{}div}\OtherTok{ class=}\StringTok{"panel{-}body"}\OtherTok{ [}\ErrorTok{@expandCollapse]}\OtherTok{=}\StringTok{"isExpended ? \textquotesingle{}expanded\textquotesingle{} : \textquotesingle{}collapsed\textquotesingle{}"}\KeywordTok{\textgreater{}\textless{}/div\textgreater{}}
\end{Highlighting}
\end{Shaded}
\end{enumerate}

\hypertarget{169-ux8defux7531ux52a8ux753b}{%
\paragraph{16.9 路由动画}\label{169-ux8defux7531ux52a8ux753b}}

\begin{figure}
\centering
\includegraphics{C:/Users/ZSH/Desktop/ng/ppt/images/57.gif}
\caption{}
\end{figure}

\begin{enumerate}
\def\labelenumi{\arabic{enumi}.}
\item
  为路由添加状态标识,此标识即为路由执行动画时的自定义状态

\begin{Shaded}
\begin{Highlighting}[]
\KeywordTok{const}\NormalTok{ routes}\OperatorTok{:}\NormalTok{ Routes }\OperatorTok{=}\NormalTok{ [}
\NormalTok{  \{}
    \DataTypeTok{path}\OperatorTok{:} \StringTok{""}\OperatorTok{,}
    \DataTypeTok{component}\OperatorTok{:}\NormalTok{ HomeComponent}\OperatorTok{,}
    \DataTypeTok{pathMatch}\OperatorTok{:} \StringTok{"full"}\OperatorTok{,}
    \DataTypeTok{data}\OperatorTok{:}\NormalTok{ \{}
      \DataTypeTok{animation}\OperatorTok{:} \StringTok{"one"} 
\NormalTok{    \}}
\NormalTok{  \}}\OperatorTok{,}
\NormalTok{  \{}
    \DataTypeTok{path}\OperatorTok{:} \StringTok{"about"}\OperatorTok{,}
    \DataTypeTok{component}\OperatorTok{:}\NormalTok{ AboutComponent}\OperatorTok{,}
    \DataTypeTok{data}\OperatorTok{:}\NormalTok{ \{}
      \DataTypeTok{animation}\OperatorTok{:} \StringTok{"two"}
\NormalTok{    \}}
\NormalTok{  \}}\OperatorTok{,}
\NormalTok{  \{}
    \DataTypeTok{path}\OperatorTok{:} \StringTok{"news"}\OperatorTok{,}
    \DataTypeTok{component}\OperatorTok{:}\NormalTok{ NewsComponent}\OperatorTok{,}
    \DataTypeTok{data}\OperatorTok{:}\NormalTok{ \{}
      \DataTypeTok{animation}\OperatorTok{:} \StringTok{"three"}
\NormalTok{    \}}
\NormalTok{  \}}
\NormalTok{]}
\end{Highlighting}
\end{Shaded}
\item
  通过路由插座对象获取路由状态标识,并将标识传递给动画的调用者,让动画执行当前要执行的状态是什么

\begin{Shaded}
\begin{Highlighting}[]
\KeywordTok{\textless{}div}\OtherTok{ class=}\StringTok{"routerContainer"}\OtherTok{ [}\ErrorTok{@routerAnimations]}\OtherTok{=}\StringTok{"prepareRoute(outlet)"}\KeywordTok{\textgreater{}}
  \KeywordTok{\textless{}router{-}outlet}\OtherTok{ \#outlet=}\StringTok{"outlet"}\KeywordTok{\textgreater{}\textless{}/router{-}outlet\textgreater{}}
\KeywordTok{\textless{}/div\textgreater{}}
\end{Highlighting}
\end{Shaded}

\begin{Shaded}
\begin{Highlighting}[]
\ImportTok{import}\NormalTok{ \{ RouterOutlet \} }\ImportTok{from} \StringTok{"@angular/router"}

\ImportTok{export} \KeywordTok{class}\NormalTok{ AppComponent \{}
  \FunctionTok{prepareRoute}\NormalTok{(}\DataTypeTok{outlet}\OperatorTok{:}\NormalTok{ RouterOutlet) \{}
    \ControlFlowTok{return}\NormalTok{ (}
\NormalTok{      outlet }\OperatorTok{\&\&}
\NormalTok{      outlet}\OperatorTok{.}\AttributeTok{activatedRouteData} \OperatorTok{\&\&}
\NormalTok{      outlet}\OperatorTok{.}\AttributeTok{activatedRouteData}\OperatorTok{.}\AttributeTok{animation}
\NormalTok{    )}
\NormalTok{  \}}
\NormalTok{\}}
\end{Highlighting}
\end{Shaded}
\item
  将 routerContainer 设置为相对定位,将它的直接一级子元素设置成绝对定位

\begin{Shaded}
\begin{Highlighting}[]
\CommentTok{/* styles.css */}
\FunctionTok{.routerContainer}\NormalTok{ \{}
  \KeywordTok{position}\NormalTok{: }\DecValTok{relative}\OperatorTok{;}
\NormalTok{\}}

\FunctionTok{.routerContainer} \OperatorTok{\textgreater{}} \OperatorTok{*}\NormalTok{ \{}
  \KeywordTok{position}\NormalTok{: }\DecValTok{absolute}\OperatorTok{;}
  \KeywordTok{left}\NormalTok{: }\DecValTok{0}\OperatorTok{;}
  \KeywordTok{top}\NormalTok{: }\DecValTok{0}\OperatorTok{;}
  \KeywordTok{width}\NormalTok{: }\DecValTok{100}\DataTypeTok{\%}\OperatorTok{;}
\NormalTok{\}}
\end{Highlighting}
\end{Shaded}
\item
  创建动画

\begin{Shaded}
\begin{Highlighting}[]
\FunctionTok{trigger}\NormalTok{(}\StringTok{"routerAnimations"}\OperatorTok{,}\NormalTok{ [}
  \FunctionTok{transition}\NormalTok{(}\StringTok{"one =\textgreater{} two, one =\textgreater{} three, two =\textgreater{} three"}\OperatorTok{,}\NormalTok{ [}
    \FunctionTok{query}\NormalTok{(}\StringTok{":enter"}\OperatorTok{,} \FunctionTok{style}\NormalTok{(\{ }\DataTypeTok{transform}\OperatorTok{:} \StringTok{"translateX(100\%)"}\OperatorTok{,} \DataTypeTok{opacity}\OperatorTok{:} \DecValTok{0}\NormalTok{ \}))}\OperatorTok{,}
    \FunctionTok{group}\NormalTok{([}
      \FunctionTok{query}\NormalTok{(}
        \StringTok{":enter"}\OperatorTok{,}
        \FunctionTok{animate}\NormalTok{(}
          \StringTok{"0.4s ease{-}in"}\OperatorTok{,}
          \FunctionTok{style}\NormalTok{(\{ }\DataTypeTok{transform}\OperatorTok{:} \StringTok{"translateX(0)"}\OperatorTok{,} \DataTypeTok{opacity}\OperatorTok{:} \DecValTok{1}\NormalTok{ \})}
\NormalTok{        )}
\NormalTok{      )}\OperatorTok{,}
      \FunctionTok{query}\NormalTok{(}
        \StringTok{":leave"}\OperatorTok{,}
        \FunctionTok{animate}\NormalTok{(}
          \StringTok{"0.4s ease{-}out"}\OperatorTok{,}
          \FunctionTok{style}\NormalTok{(\{}
            \DataTypeTok{transform}\OperatorTok{:} \StringTok{"translateX({-}100\%)"}\OperatorTok{,}
            \DataTypeTok{opacity}\OperatorTok{:} \DecValTok{0}
\NormalTok{          \})}
\NormalTok{        )}
\NormalTok{      )}
\NormalTok{    ])}
\NormalTok{  ])}\OperatorTok{,}
  \FunctionTok{transition}\NormalTok{(}\StringTok{"three =\textgreater{} two, three =\textgreater{} one, two =\textgreater{} one"}\OperatorTok{,}\NormalTok{ [}
    \FunctionTok{query}\NormalTok{(}
      \StringTok{":enter"}\OperatorTok{,}
      \FunctionTok{style}\NormalTok{(\{ }\DataTypeTok{transform}\OperatorTok{:} \StringTok{"translateX({-}100\%)"}\OperatorTok{,} \DataTypeTok{opacity}\OperatorTok{:} \DecValTok{0}\NormalTok{ \})}
\NormalTok{    )}\OperatorTok{,}
    \FunctionTok{group}\NormalTok{([}
      \FunctionTok{query}\NormalTok{(}
        \StringTok{":enter"}\OperatorTok{,}
        \FunctionTok{animate}\NormalTok{(}
          \StringTok{"0.4s ease{-}in"}\OperatorTok{,}
          \FunctionTok{style}\NormalTok{(\{ }\DataTypeTok{transform}\OperatorTok{:} \StringTok{"translateX(0)"}\OperatorTok{,} \DataTypeTok{opacity}\OperatorTok{:} \DecValTok{1}\NormalTok{ \})}
\NormalTok{        )}
\NormalTok{      )}\OperatorTok{,}
      \FunctionTok{query}\NormalTok{(}
        \StringTok{":leave"}\OperatorTok{,}
        \FunctionTok{animate}\NormalTok{(}
          \StringTok{"0.4s ease{-}out"}\OperatorTok{,}
          \FunctionTok{style}\NormalTok{(\{}
            \DataTypeTok{transform}\OperatorTok{:} \StringTok{"translateX(100\%)"}\OperatorTok{,}
            \DataTypeTok{opacity}\OperatorTok{:} \DecValTok{0}
\NormalTok{          \})}
\NormalTok{        )}
\NormalTok{      )}
\NormalTok{    ])}
\NormalTok{  ])}
\NormalTok{])}
\end{Highlighting}
\end{Shaded}
\end{enumerate}

\end{document}
